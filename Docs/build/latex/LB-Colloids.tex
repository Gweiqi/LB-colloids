%% Generated by Sphinx.
\def\sphinxdocclass{report}
\documentclass[letterpaper,10pt,english]{sphinxmanual}
\ifdefined\pdfpxdimen
   \let\sphinxpxdimen\pdfpxdimen\else\newdimen\sphinxpxdimen
\fi \sphinxpxdimen=.75bp\relax
\PassOptionsToPackage{svgnames}{xcolor}
\usepackage[utf8]{inputenc}
\ifdefined\DeclareUnicodeCharacter
 \ifdefined\DeclareUnicodeCharacterAsOptional
  \DeclareUnicodeCharacter{"00A0}{\nobreakspace}
  \DeclareUnicodeCharacter{"2500}{\sphinxunichar{2500}}
  \DeclareUnicodeCharacter{"2502}{\sphinxunichar{2502}}
  \DeclareUnicodeCharacter{"2514}{\sphinxunichar{2514}}
  \DeclareUnicodeCharacter{"251C}{\sphinxunichar{251C}}
  \DeclareUnicodeCharacter{"2572}{\textbackslash}
 \else
  \DeclareUnicodeCharacter{00A0}{\nobreakspace}
  \DeclareUnicodeCharacter{2500}{\sphinxunichar{2500}}
  \DeclareUnicodeCharacter{2502}{\sphinxunichar{2502}}
  \DeclareUnicodeCharacter{2514}{\sphinxunichar{2514}}
  \DeclareUnicodeCharacter{251C}{\sphinxunichar{251C}}
  \DeclareUnicodeCharacter{2572}{\textbackslash}
 \fi
\fi
\usepackage{cmap}
\usepackage[T1]{fontenc}
\usepackage{amsmath,amssymb,amstext}
\usepackage{babel}
\usepackage{palatino}
\usepackage[Bjarne]{fncychap}
\usepackage[dontkeepoldnames]{sphinx}

\usepackage{geometry}

% Include hyperref last.
\usepackage{hyperref}
% Fix anchor placement for figures with captions.
\usepackage{hypcap}% it must be loaded after hyperref.
% Set up styles of URL: it should be placed after hyperref.
\urlstyle{same}
\addto\captionsenglish{\renewcommand{\contentsname}{Contents:}}

\addto\captionsenglish{\renewcommand{\figurename}{Fig.}}
\addto\captionsenglish{\renewcommand{\tablename}{Table}}
\addto\captionsenglish{\renewcommand{\literalblockname}{Listing}}

\addto\captionsenglish{\renewcommand{\literalblockcontinuedname}{continued from previous page}}
\addto\captionsenglish{\renewcommand{\literalblockcontinuesname}{continues on next page}}

\addto\extrasenglish{\def\pageautorefname{page}}

\setcounter{tocdepth}{1}



\title{LB-Colloids Documentation}
\date{Oct 13, 2018}
\release{0.1}
\author{Joshua D Larsen}
\newcommand{\sphinxlogo}{\vbox{}}
\renewcommand{\releasename}{Release}
\makeindex

\begin{document}

\maketitle
\sphinxtableofcontents
\phantomsection\label{\detokenize{index::doc}}



\chapter{Introduction to LB-Colloids}
\label{\detokenize{index:welcome-to-the-lb-colloids-documentation}}\label{\detokenize{index:introduction-to-lb-colloids}}
Lattice Boltzmann colloids is an object oriented python package that builds and simulates two dimensional, nine fluid node (D2Q9) computational fluid dynamic models. Lattice Boltzmann is able to represent fluid domains defined by complex geometries ehich are observed in natural porous media. Simulated fluid flow is slightly compressible and has been shown to return an approximation of the Navier-Stokes equation (\sphinxstyleemphasis{Benzi et al 1992}). The lattice Boltzmann code used for this software is based on the single relaxation time Bhantager Gross Krook formulation of lattice Boltzmann. Simple bounceback boundary conditions are implemented at fluid boundaries to preserve mass balance. As a result of the bounceback boundary conditions no-slip conditions develop at pore-solid interfaces.

Binarized imagery is used to represent the fluid flow domain in this implementation of lattice Boltzmann. Distinct advantages of using binarized imagery are: fluid flow in natural porous media can be simulated through the use of segmented thin sections, and synthetic porous media can easily be generated as an image and used for fluid dynamic simulations.

Colloid particle tracking simulations can be parameterized and simulated for steady state lattice Boltzmann computational fluid dynamic models using this software. Colloid transport through the soil environment is of great interest and importance to soil development processes through the translocation of clays, contaminant transport (\sphinxstyleemphasis{Saiers 1996, Jaisi et al 2008}), filtration and transport of bio-colloids (\sphinxstyleemphasis{Harter 2000, Redman 2004, Foppen et al 2005}), and soil nutrient dynamics. Microscale colloid particle tracking simulations allow the user to gain additional insight into physio-chemical processes influencing colloid transport. Coupling macroscopic breakthrough curve information with the microscopic distribution of colloids during a simulation provides insight into attachment and colloid retention mechanisms for simulations. These insights have the potential to illuminate retention and transport mechanisms for column and field scale studies of colloid and contaminant transport.

Parameterization of an LB-Colloids model is possible in two ways. Formatted text documents can be supplied to the LB-Colloids simulation software and results will be stored in an hdf5 file and a human readable text document. Advanced users may prefer to parameterize simulations through a python interface. This gives the advanced user the ability to easily calibrate models by adjusting parameters within python code blocks, run multiple models, and perform sensitivity analysis. Currently output readers and built in analysis methods are only available to the advanced user. API documentation serves as a guide to the advanced user on importing and using these python objects.

This document serves as the user guide to the LB-Colloids python package. For more details pertaining to the development of this package see \sphinxstyleemphasis{Larsen and Schaap TBD}. The user guide follows the structure:

Chapter 2: Table of mathematical symbols

Chapter 3: Installation of LB-Colloids

Chapter 4: Formatted text input

Chapter 5: Parameterization of LB-Colloids

Chapter 6: Lattice Boltzmann API

Chapter 7: Colloid Simulation API

Chapter 8: Penetrable-Sphere (PSHPERE) API


\chapter{Table of Mathematical Symbols}
\label{\detokenize{index:table-of-mathematical-symbols}}

\section{Lattice Boltzmann}
\label{\detokenize{index:lattice-boltzmann}}
\(\pmb{e_{i}}\): lattice Boltzmann eigenvector array

\(f_{i}\) :  distribution function

\(f_{eq}\) : equilibrium distribution function

\(\rho\) :  macroscopic fluid density

\(\tau\) :  relaxation time

\(\pmb{u}\) :  macroscopic fluid velocity

\(v\) : fluid viscosity

\(w_{i}\) : fluid link weights


\section{Colloid Simulation}
\label{\detokenize{index:colloid-simulation}}
\(A_{H}\) : Hamaker constant

\(a_{c}\) : colloid radius

\(D_{0}\) : diffusivity

\(\epsilon_{0}\) : dielectric permativity of a vacuum

\(\epsilon_{r}\) : dielectric permativity of water

\(e\) : electron charge

\(F^{b}\) : bouyancy force

\(F^{B}\) : brownian force

\(F^{D}\) : drag force

\(F^{G}\) : gravity force

\(f_{1}\) : hydrodynamic correction factor

\(f_{2}\) : hydrodynamic correction factor

\(f_{3}\) : hydrodynamic correction factor

\(f_{4}\) : hydrodynamic correction factor

\(G(0, 1)\) : random gaussian distribution

\(g\) : acceleration due to gravity

\(h\) : gap distance

\(\bar{h}\) : non-dimensional gap distance

\(I^{*}\) : Two times fluid ionic strength.

\(k\) : Boltzmann constant

\(\kappa\) : Debye length

\(M\) : molarity

\(N_{A}\) : Avagadro’s number

\(\Phi^{A}\) : Attractive interaction energy

\(\Phi^{EDL}\) : Electric double layer interaction energy

\(\psi_{c}\) : colloid potential

\(\psi_{s}\) : surface potential

\(\rho_{c}\) : colloid density

\(\rho_{w}\) : fluid density

\(u\) : fluid velocity

\(\mu\) : fluid viscosity

\(V\) : colloid velocity

\(T\) : Fluid temperature

\(Z\) : cation charge

\(\xi\) : \(6\pi \mu a_{c}\)


\chapter{Installation of LB-Colloids}
\label{\detokenize{index:installation-of-lb-colloids}}
LB-Colloids is currently avaialable for python 2.7.6 - 2.7.12 installations. Support is currently not available for python 3, however most source code can be converted to python 3 with little trouble. Future support is planned for python 3.5

Recommended installation of LB-Colloids is performed using the python tool pip. If pip is not included with your python distribution an installation script can be found at \sphinxurl{https://bootstrap.pypa.io/get-pip.py}. It is recommended that the user adds pip as a path variable for easy installation of python packages.

An installation of gfortran must be present on the user’s computer to properly install LB-Colloid’s. Mathematic modules are compiled locally in FORTRAN upon install and called by python for computational efficiency. Gfortran should be added as a path variable to your computer.

LB-Colloids can be downloade from \sphinxstyleemphasis{pypi} and/ort \sphinxstyleemphasis{tbd}. Move the LB-Colloids package to your prefered source code location.

Open a terminal and navigate to the base directory of the LB-Colloids source code. You should see setup.py in this directory. Install LB-Colloids using pip.

\begin{sphinxVerbatim}[commandchars=\\\{\}]
\PYG{g+gp}{\PYGZgt{}\PYGZgt{}\PYGZgt{} }\PYG{n}{cd} \PYG{n}{Desktop}\PYG{o}{/}\PYG{n}{LB}\PYG{o}{\PYGZhy{}}\PYG{n}{Colloids}
\PYG{g+gp}{\PYGZgt{}\PYGZgt{}\PYGZgt{} }\PYG{n}{pip} \PYG{n}{install} \PYG{o}{.}
\PYG{g+gp}{\PYGZgt{}\PYGZgt{}\PYGZgt{} }\PYG{c+c1}{\PYGZsh{} or for the developer use}
\PYG{g+gp}{\PYGZgt{}\PYGZgt{}\PYGZgt{} }\PYG{n}{pip} \PYG{n}{install} \PYG{o}{\PYGZhy{}}\PYG{n}{e} \PYG{o}{.}
\end{sphinxVerbatim}

Congratulations! LB-Colloids is now installed on your machine.


\chapter{FORMATED TEXT INPUT}
\label{\detokenize{index:formated-text-input}}
\sphinxstyleemphasis{LB Colloids} computational fluid dynamic models can be parameterized
through the use of formatted text files that are parsed by internal
input control modules. Five separate data types are supported in
the input reader modules. Data types used during the
\sphinxstyleemphasis{LB-Colloids} input parameterization are strictly enforced. It is highly
recommended that the user familiarize themselves with data types before
attempting to parameterize and run \sphinxstyleemphasis{LB-Colloids.}


\section{Description of Data Types}
\label{\detokenize{index:description-of-data-types}}
\sphinxstylestrong{STRING:} String type data within \sphinxstyleemphasis{LB-Colloids} is limited to file
names.

\sphinxstyleemphasis{Example usage:}

LBMODEL: mymodel.png

\sphinxstylestrong{INTEGER:} Integer type data.

\sphinxstyleemphasis{Example usage:}

NCOLS: 200

\sphinxstylestrong{FLOAT:} A floating point value or real number must be supplied.

\sphinxstyleemphasis{Example usage:}

AC: 1e-6

\sphinxstylestrong{BOOLEAN:} A value of True or False must be supplied.

\sphinxstyleemphasis{Example usage:}

PLOT: True

\sphinxstylestrong{LIST:} List type data combines data types to allow the user to
specify multiple values in the parameterization process. This data type
is used to set image boundary conditions using the solid and void
keywords.

\sphinxstyleemphasis{Example usage:}

SOLID: 255 223 200

\sphinxstylestrong{DICTIONARY:} Dictionary type data combines data types to use related
pairs of data in the parameterization process. This data type is limited
in usage to the optional concentration and valence keywords.

\sphinxstyleemphasis{Example usage:}

VALENCE: Na 1 Ca 2 Mg 2 Al 3


\section{NAM File Inputs}
\label{\detokenize{index:nam-file-inputs}}
The \sphinxstyleemphasis{LB-Colloid NAM file} is a simple formatted text file that is read
by run\_model.py. The NAM file contains relative or absolute file paths
to a Lattice Boltzmann configuration file and colloid transport
configuration files. The program run\_model.py looks within its current
directory for files ending with the suffix .nam, reads them, and
delegates the input and output control to the appropriate \sphinxstyleemphasis{LB-Colloids}
modules at runtime.


\subsection{Input Structure:}
\label{\detokenize{index:input-structure}}
Input uses a block type structure. Block keywords begin a
parameterization section. When the parameterization section is complete,
the keyword end must be supplied for \sphinxstyleemphasis{LB-}Colloids to close the block
input reader. Within the input structure description, brackets
surrounding a parameter indicates that it is an optional parameter.

\sphinxstylestrong{LBMODEL (STRING):} Block keyword to designate the beginning of a
lattice Boltzmann parameterization block. This keyword is used to inform
run\_model.py that a lattice Boltzmann simulation will be performed.
\begin{quote}

\sphinxstylestrong{LBCONFIG (STRING):} File name of the lattice Boltzmann
configuration file that is used for simulation.
\end{quote}

\sphinxstylestrong{END:} Block keyword to inform run\_model.py that it has finished
gathering parameters for the LBMODEL data block.

\sphinxstylestrong{{[}COLLOIDMODEL{]} (STRING):} Optional block keyword to designate the
beginning of a colloid transport parameterization block. The keyword is
used to inform run\_model.py that a colloid transport simulation will be
performed.
\begin{quote}

\sphinxstylestrong{{[}COLLOIDCONFIG{]}*n (STRING):} Optional file name of the colloid
transport configuration file that is used for the simulation.
Multiple colloid transport configuration files can be supplied for a
multiple ionic strength model (MISM) and each filename must occupy a
new line within the NAM file. This parameter is required if the
COLLOIDMODEL keyword is used
\end{quote}

\sphinxstylestrong{{[}END{]} (STRING):} Optional block keyword to inform run\_model.py that
it has finished gathering parameters for the COLLOIDMODEL data block.
This parameter is required if the COLLOIDMODEL keyword is used


\subsection{Example NAM file:}
\label{\detokenize{index:example-nam-file}}
\sphinxincludegraphics[width=6.50000in,height=5.31250in]{{NAM}.png}


\section{Lattice Boltzmann Configuration File}
\label{\detokenize{index:lattice-boltzmann-configuration-file}}
The lattice Boltzmann configuration file uses a block input format to
parameterize D2Q9 lattice Boltzmann simulations. Four separate
blocks are available to the user. The MODEL PARAMETERS and IMAGE
PARAMETERS block must be supplied for basic model configuration. The
PERMEABILITY PARAMETERS block is optional, however the NITERS value
must be updated for a permeability model to be able to run to steady
state conditions. Configuration blocks can be added to the lattice
Boltzmann configuration file in any order as long as the proper
formatting requirements are fulfilled. Parameters within a configuration
block may be listed in any order as long as parameter requirements are
fulfilled. Only one parameter may be listed per line within the
configuration file.


\subsection{Input Structure:}
\label{\detokenize{index:id1}}
\sphinxstylestrong{START MODEL PARAMETERS (STRING):} Required block keyword to begin
parameterization of the lattice Boltzmann model parameters data block.
\begin{quote}

\sphinxstylestrong{LBMODEL (STRING):} Hdf5 file path and name corresponding to the
lattice Boltzmann model. Lattice Boltzmann simulation results will
be written to this Hdf5 file.

\sphinxstylestrong{LBRES (FLOAT):} The lattice Boltzmann simulation resolution
parameter corresponds to the image resolution, in meters, of the
synthetic porous media or soil thin section for D2Q9 lattice
Boltzmann simulations.

\sphinxstylestrong{{[}KERNEL{]} (STRING):} Lattice Boltzmann debug option that allows
users to switch between simulating models with Python or FORTRAN as
the base computational code. This option defaults to FORTRAN
(recommended) if it is not supplied by the user. The FORTRAN kernel
is approximately 100x faster than the Python kernel. Valid options
are ‘python’ and ‘fortran’.

\sphinxstylestrong{{[}PHYSICAL\_VISCOSITY{]} (FLOAT):} Optional parameter relating to
the physical viscosity of the simulated fluid within a model run.
This parameter is used in the dimensionalization process from
non-dimensional lattice units to SI units. Default value is 8.9e-4
Pa S which corresponds to the viscosity of water at 25$^{\text{o}}$ C.

\sphinxstylestrong{{[}PHYSICAL\_RHO{]} (FLOAT):} Optional parameter relating to the
physical density of the simulated fluid within a model run. This
parameter is used in the dimensionalization process from
non-dimensional lattice units to SI units. Default value is 997
Kg/m$^{\text{3}}$ which corresponds to the viscosity of water at
25$^{\text{o}}$ C.
\end{quote}

\sphinxstylestrong{END MODEL PARAMETERS (STRING):} Required block keyword to end
parameterization of the lattice Boltzmann model parameters block

\sphinxstylestrong{START IMAGE PARAMETERS (STRING):} Required block keyword to begin the
parameterization of model domain boundary conditions using image
properties of the supplied thin section.
\begin{quote}

\sphinxstylestrong{IMAGE (STRING):} Filename of the image representing the lattice Boltzmann simulation domain

\sphinxstylestrong{SOLID (LIST {[}INTEGER{]}):} List of solid phase greyscale values as
integers corresponding to the supplied model thin section

\sphinxstylestrong{VOID (LIST {[}INTEGER{]}):} List of pore/void phase greyscale values
as integers corresponding to the supplied model thin section

\sphinxstylestrong{{[}BOUNDARY{]} (INTEGER):} Number of boundary layers to add to the
top and bottom of the simulation domain. This parameter helps avoid
local compression effects due to dead end pores connected to the top
and bottom of the simulation domain. Default is 10 boundary layers.

\sphinxstylestrong{{[}PLOT{]} (BOOLEAN):} Keyword parameter to display a plot of the
binarized model domain before simulating fluid flow. This option is
recommended during initial parameterization runs to ensure that
fluid and solid phases are being properly represented within the
domain. Default is False.
\end{quote}

\sphinxstylestrong{END IMAGE PARAMETERS (STRING):} Required block keyword to end the
parameterization of the image parameters data block.

\sphinxstylestrong{START PERMEABILITY PARAMETERS (STRING):} Required keyword to begin
the lattice Boltzmann permeability model parameterization block.
\begin{quote}

\sphinxstylestrong{{[}NITERS{]} (INTEGER):} Number of simulation time steps (number of
iterations) applied to the lattice Boltzmann permeability
simulation. Defaults to 1. It is highly recommended to change this
value as one time step will not reach equilibrium conditions.

\sphinxstylestrong{{[}RHO{]} (FLOAT):} Initial fluid density for lattice Boltzmann
simulation. Defaults to 1.

\sphinxstylestrong{{[}TAU{]} (FLOAT):} Lattice Boltzmann BGK relaxation time parameter.
Defaults to 1. The acceptable range of TAU is from 0.5 to 1.5.

\sphinxstylestrong{{[}GRAVITY{]} (FLOAT):} Gravity force applied to fluid as a body
force. Gravity drives fluid flow in the simulation. Default is 1e-3.
\end{quote}

\sphinxstylestrong{END PERMEABILITY PARAMETERS (STRING):} Required keyword to end the
lattice Boltzmann permeability model parameterization block

\sphinxstylestrong{{[}START OUTPUT CONTROL{]} (STRING):} Keyword to begin the lattice
Boltzmann output control parameterization block. This block contains
optional parameters that write updates to the terminal and save imagery
of lattice Boltzmann simulation progression.
\begin{quote}

\sphinxstylestrong{{[}VERBOSE{]} (INTEGER):} Integer flag that indicates the number of
time steps between lattice Boltzmann print statements to the
terminal. This option updates user of the model’s progression.
Default is 100.

\sphinxstylestrong{{[}IMAGE\_SAVE\_INTERVAL{]} (INTEGER):} Integer flag that indicates
the time step interval to save model fluid velocity to a matplotlib
image. Default is None and no images are saved to disk.

\sphinxstylestrong{{[}IMAGE\_SAVE\_NAME{]} (STRING):} Base name of images to save as
output. This parameter is ignored unless the IMAGE\_SAVE\_INTERVAL
parameter is used. Default IMAGE\_SAVE\_NAME is “LB”.

\sphinxstylestrong{{[}IMAGE\_SAVE\_FOLDER{]} (STRING):} Directory location to save
simulation velocity images. This parameter is ignored unless the
IMAGE\_SAVE\_INTERVAL parameter is used. Default IMAGE\_SAVE\_FOLDER
is “\textasciitilde{}/Desktop/LBimages”.

\sphinxstylestrong{{[}VMIN{]} (FLOAT):} VMIN controls the minimum boundary of velocity
to plot on a lattice Boltzmann output image. This parameter is used
to adjust the color scale for plotting only. VMIN can only be used
if IMAGE\_SAVE\_INTERVAL is used. Default value is -0.010.

\sphinxstylestrong{{[}VMAX{]} (FLOAT):} VMAX controls the maximum boundary of velocity
to plot on a lattice Boltzmann output image. This parameter is used
adjust the color scale for plotting only. VMAX can only be used if
IMAGE\_SAVE\_INTERVAL is used. Default value is 0.0.
\end{quote}

\sphinxstylestrong{{[}END OUTPUT CONTROL{]} (STRING):} Keyword to end the lattice Boltzmann
output control parameterization block.


\subsection{Example Lattice Boltzmann configuration file:}
\label{\detokenize{index:example-lattice-boltzmann-configuration-file}}
\sphinxincludegraphics[width=6.50000in,height=5.20000in]{{LBCF}.png}


\section{Colloids Simulation Control File}
\label{\detokenize{index:colloids-simulation-control-file}}
The colloids simulation control file uses a block input structure to
parameterize colloid simulations within the \sphinxstyleemphasis{LB-Colloids} system. Four
model parameter blocks are available to the user to parameterize colloid
models. The MODEL PARAMETERS block must be supplied in the colloid
simulation control file to run a model. The PHYSICAL PARAMETERS and
CHEMICAL PARAMETERS configuration blocks are optional, however it will
be necessary to specify some parameters within each of these blocks to
properly simulate experimental conditions for individual models.
Defaults within these blocks pertain to glass bead media and kaolinite
colloids. The OUTPUT PARAMETERS configuration block is optional and
contains useful parameters for later data analysis. Configuration blocks
can be added to the colloids simulation control file in any order as
long as the proper formatting requirements are fulfilled. Parameters
within a configuration block may be listed in any order as long as
parameter requirements are fulfilled. Only one parameter may be listed
per line within the configuration file.


\subsection{Input Structure:}
\label{\detokenize{index:id2}}
\sphinxstylestrong{START MODEL PARAMETERS (STRING):} Keyword to begin the model
parameters input block. This input block contains necessary
parameterization information to run a basic colloid simulation.
\begin{quote}

\sphinxstylestrong{LBMODEL (STRING):} HDF5 file name containing results from a
steady state lattice Boltzmann simulation

\sphinxstylestrong{LBRES (FLOAT):} Lattice Boltzmann grid resolution.

\sphinxstylestrong{GRIDREF (FLOAT):} Grid refinement option which creates a bilinear
interpolation of the lattice Boltzmann model domain. Colloid grid
resolution is defined by
\begin{quote}
\begin{equation*}
\begin{split}COLRES = \frac{\text{LBRES}}{\text{GRIDREF}}\end{split}
\end{equation*}\end{quote}

\sphinxstylestrong{ITERS (INTEGER):} Number of time steps the colloid simulation
will run for

\sphinxstylestrong{TIMESTEP (FLOAT):} Time step length. Reduction of time step
length creates increased stability and greater accuracy, but longer
model run times. A recommended starting point is 1e-06 s.

\sphinxstylestrong{NCOLS (INTEGER):} Number of colloids to introduce into the system
in the initial time step.

\sphinxstylestrong{{[}CONTINUOUS{]} (INTEGER):} Continuous is a flag that indicates
multiple releases of colloids throughout the simulation. The value
of continuous is the interval when additional colloids are released
into the colloid simulation. Default is 0 (a single pulse of
colloids released at the beginning of time step 1).

\sphinxstylestrong{{[}AC{]} (FLOAT):} The colloid radius parameter defaults to 1e-6 m.

\sphinxstylestrong{{[}RHO\_COLLOID{]} (FLOAT):} Colloid density parameter is optional
and adjustable based on type of colloidal particle being simulated.
Default value is 2650. Kg/m$^{\text{3}}$

\sphinxstylestrong{{[}TEMPERATURE{]} (FLOAT):} Fluid temperature parameter defaults to
298.15 K.
\end{quote}

\sphinxstylestrong{END MODEL PARAMETERS (STRING):} Keyword to end the model parameters
input block.

\sphinxstylestrong{{[}START PHYSICAL PARAMETERS{]} (STRING):} Keyword to that indicates the
beginning of the physical parameters input block.
\begin{quote}

\sphinxstylestrong{{[}RHO\_WATER{]} (FLOAT):} Physical density of water. Default is 997.
Kg/m$^{\text{3}}$

\sphinxstylestrong{{[}RHO\_COLLOID{]} (FLOAT):} Colloid density parameter. It is
unnecessary to parameterize if RHO\_COLLOID has been added in the
model parameters input block. Defaults to 2650 Kg/m$^{\text{3}}$

\sphinxstylestrong{{[}VISCOSITY{]} (FLOAT):} Dynamic viscosity of the fluid phase within
a colloid simulation. Default is 8.9e-4 Pa S.

\sphinxstylestrong{{[}SCALE\_LB{]} (FLOAT):} Lattice Boltzmann velocity scaling factor.
Exercise extreme caution in using this option.
\end{quote}

\sphinxstylestrong{{[}END PHYSICAL PARAMETERS{]} (STRING):} Keyword to end the physical
parameters input block. This is a required parameter if START PHYSICAL
PARAMETERS keyword is used.

\sphinxstylestrong{{[}START CHEMICAL PARAMETERS{]} (STRING):} Keyword to that indicates the
beginning of the chemical parameters input block.
\begin{quote}

\sphinxstylestrong{{[}I{]} (FLOAT):} The ionic strength of the fluid phase is calculated
by the equation
\begin{quote}
\begin{equation*}
\begin{split}I =\sum_{i = 1}^{n}{Z_{i}^{2}M_{i}}\end{split}
\end{equation*}\end{quote}

where Z$_{\text{i}}$ is the cation charge and M$_{\text{i}}$ is the
molarity of each solution component i. The default ionic strength is
set to 1e-3 M.

\sphinxstylestrong{{[}ZETA\_SOLID{]} (FLOAT):} Bulk zeta potential of the solid phase
within a colloid simulation. Default value is for glass bead media
-60.9e-3 mV.

\sphinxstylestrong{{[}ZETA\_COLLOID{]} (FLOAT):} Bulk zeta potential of colloids
introduced into a colloid simulation. Default value is for kaolinite
colloids -40.5e-3 mV.

\sphinxstylestrong{{[}CONCENTRATION{]} (DICTIONARY {[}STRING, FLOAT{]}):} Optional parameter
that allows the user to specify cation molarity pairs to
parameterize the model instead of using ionic strength. This
parameter must be supplied if VALENCE is used.

\sphinxstylestrong{{[}VALENCE{]} (DICTIONARY {[}STRING, INTEGER{]}):} Optional parameter
that allows the user to specify cation valence pairs to parameterize
the model instead of using ionic strength. This parameter must be
supplied if CONCENTRATION is used.

\sphinxstylestrong{{[}LVDWST\_WATER{]} (FLOAT):} The van der Waals surface tension of
the simulation fluid which is used to parameterize van der Waals
interactions within the colloids simulation. Default value is
21.8e-3 J/m$^{\text{2}}$ which corresponds to water.

\sphinxstylestrong{{[}LVDWST\_COLLOID{]} (FLOAT):} The van der Waals surface tension of
the simulated colloidal material which is used to parameterize van
der Waals interactions within the colloids simulation. Default value
is 39.9e-3 J/m$^{\text{2}}$ which corresponds to a kaolinite colloid.

\sphinxstylestrong{{[}LVDWST\_SOLID{]} (FLOAT):} The van der Waals surface tension of
the simulated solid phase which is used to parameterize van der
Waals interactions within the colloids simulation. Default value is
33.7e-3 J/m$^{\text{2}}$ for glass bead porous media.

\sphinxstylestrong{{[}PSI+\_WATER{]} (FLOAT):} The electron-acceptor parameter of Lewis
Acid Base surface tension for the simulation fluid. Default is
25.5e-3 J/m$^{\text{2}}$ for water

\sphinxstylestrong{{[}PSI+\_COLLOID{]} (FLOAT):} The electron-acceptor parameter of
Lewis Acid Base surface tension for the colloid material. Default is
0.4e-3 J/m$^{\text{2}}$ for a kaolinite colloid.

\sphinxstylestrong{{[}PSI+\_SOLID{]} (FLOAT):} The electron-acceptor parameter of Lewis
Acid Base surface tension for the solid phase. Default is 1.3e-3
J/m$^{\text{2}}$ for a glass bead porous media.

\sphinxstylestrong{{[}PSI-\_WATER{]} (FLOAT):} The electron-donor parameter of Lewis
Acid Base surface tension for the simulation fluid. Default is
25.5e-3 J/m$^{\text{2}}$ for water.

\sphinxstylestrong{{[}PSI-\_COLLOID{]} (FLOAT):} The electron-donor parameter of Lewis
Acid Base surface tension for the colloid material. Default is
34.3e-3 J/m$^{\text{2}}$ for a kaolinite colloid.

\sphinxstylestrong{{[}PSI-\_SOLID{]} (FLOAT):} The electron-donor parameter of Lewis
Acid Base surface tension for the solid phase. Default is 62.2e-3
J/m$^{\text{2}}$ for glass bead porous media.

\sphinxstylestrong{{[}SHEER\_PLANE{]} (FLOAT):} Equivalent to the thickness of one layer
of water molecules. Also referred to as thickness of the stern
layer. Default is 3e-10 m.

\sphinxstylestrong{{[}EPSILON\_R{]} (FLOAT):} The dielectric constant of water at the
simulation temperature. The default value is 78.3 which corresponds
to 25$^{\text{o}}$ C.
\end{quote}

\sphinxstylestrong{{[}END CHEMICAL PARAMETERS{]}:} Keyword to end the chemical parameters
input block. This is a required parameter if START CHEMICAL PARAMETERS
keyword is used.

\sphinxstylestrong{{[}START OUTPUT CONTROL{]} (STRING):} Keyword to that indicates the
beginning of the output control input block.
\begin{quote}

\sphinxstylestrong{{[}PRINT\_TIME{]} (INTEGER):} Integer flag that indicates the number
of time steps between colloid simulations print statements to the
terminal. Updates user of the model’s progression. Default is equal
to the parameter ITERS.

\sphinxstylestrong{{[}STORE\_TIME{]} (INTEGER):} Integer flag that indicates the number
of time steps between internal storage functions within the colloid
model. Increasing the value of this flag reduces memory consumption.
This flag is also used to specify the interval for saving to a
TIMESERIES file and output plotting.

\sphinxstylestrong{{[}ENDPOINT{]} (STRING):} String flag that indicates an endpoint file
should be saved. This option is highly recommended for use. The
endpoint variable should correspond to the name of the endpoint
file the user wishes to save.

\sphinxstylestrong{{[}TIMESERIES{]} (STRING):} String flag that indicates a timeseries
file should be saved. The save interval is indicated by the
STORE\_TIME variable. The timeseries variable corresponds to the
name of the timeseries file the user wishes to save.

\sphinxstylestrong{{[}PATHLINE{]} (STRING):} String flag that indicates a pathline file
should be saved. Colloid position is written to output at every time
step using this option. The pathline variable corresponds to the
name of the pathline file the user wishes to save.

\sphinxstylestrong{{[}PLOT{]} (BOOLEAN):} Boolean flag that indicates a plot of colloid
positions within the model domain be produced upon successful model
completion. The plotting interval of colloid positions is set with
the STORE\_TIME flag. The plotted image will be saved as a \textless{}.png\textgreater{}
file with the same base name as was provided for ENDPOINT. If no
endpoint file was provided the figure will be displayed for the
user to save manually. Default is False.

\sphinxstylestrong{{[}SHOWFIG{]} (BOOLEAN):} Boolean flag to indicate that the user
wants to examine the figure before manually saving to file. SHOWFIG
will only work when PLOT is True. Default is False.

\sphinxstylestrong{{[}OVERWRITE{]} (BOOLEAN):} Boolean flag to overwrite data on
existing HDF5 file. This flag is useful if the user does not wish to
re-run lattice Boltzmann simulations while optimizing a colloid
simulation.
\end{quote}

\sphinxstylestrong{{[}END OUTPUT CONTROL{]} (STRING):} Keyword to end the output control
input block. This is a required parameter if START OUTPUT CONTROL
keyword is used.


\subsection{Example colloid model configuration file:}
\label{\detokenize{index:example-colloid-model-configuration-file}}
\sphinxincludegraphics[width=6.50000in,height=5.17708in]{{CMCF}.png}


\chapter{Lattice Boltzmann}
\label{\detokenize{index:id3}}

\section{Lattice Boltzmann boundary conditions}
\label{\detokenize{index:lattice-boltzmann-boundary-conditions}}
The lattice Boltzmann boundary condition modules provide classes and methods to prepare model domains, supplied to LB-Colloids as .png images, for use in a lattice Boltzmann simulation. Before a lattice Boltzmann simulation can take place, it is necessary to convert the model domain imagery into a boolean representation of that media. The boolean conversion allows for easy recognition of pore-solid interface nodes and the representation of geometrically complex domains. Boundary layers can also be added to the upper and lower boundaries of the lattice Boltzmann model. Boundary layers are applied to reduce the potential development of local pressure maxima and compresssion effects due to periodic boundary conditions at the inlet and outlet of the system. Bounceback boundary conditions are applied to all other surfaces within the system.


\subsection{API documentation}
\label{\detokenize{index:api-documentation}}\label{\detokenize{index:module-lb_colloids}}\index{lb\_colloids (module)}\index{LBImage (in module lb\_colloids)}

\begin{fulllineitems}
\phantomsection\label{\detokenize{index:lb_colloids.LBImage}}\pysigline{\sphinxcode{lb\_colloids.}\sphinxbfcode{LBImage}}
alias of {\hyperref[\detokenize{index:module-lb_colloids.LB.LB_2Dimage}]{\sphinxcrossref{\sphinxcode{lb\_colloids.LB.LB\_2Dimage}}}}

\end{fulllineitems}

\phantomsection\label{\detokenize{index:module-lb_colloids.LB.LB_2Dimage}}\index{lb\_colloids.LB.LB\_2Dimage (module)}
Lattice boltzmann image preparation and domain setup module

This module contains an image reading utility and a binarization
utility. The LB\_2Dimage module is aliased in LB-Colloids as LBImage.

Example usage of how to create a binary image with boundary conditions applied
and save the base model for usage in with LB\_permeability

\begin{sphinxVerbatim}[commandchars=\\\{\}]
\PYG{g+gp}{\PYGZgt{}\PYGZgt{}\PYGZgt{} }\PYG{k+kn}{from} \PYG{n+nn}{lb\PYGZus{}colloids} \PYG{k}{import} \PYG{n}{LBImage}
\PYG{g+gp}{\PYGZgt{}\PYGZgt{}\PYGZgt{} }\PYG{n}{image} \PYG{o}{=} \PYG{n}{LBImage}\PYG{o}{.}\PYG{n}{Images}\PYG{p}{(}\PYG{l+s+s2}{\PYGZdq{}}\PYG{l+s+s2}{my\PYGZus{}thin\PYGZus{}section.png}\PYG{l+s+s2}{\PYGZdq{}}\PYG{p}{)}
\PYG{g+gp}{\PYGZgt{}\PYGZgt{}\PYGZgt{} }\PYG{n}{binary} \PYG{o}{=} \PYG{n}{LBImage}\PYG{o}{.}\PYG{n}{BoundaryCondition}\PYG{p}{(}\PYG{n}{image}\PYG{p}{,} \PYG{n}{fluidvx}\PYG{o}{=}\PYG{p}{[}\PYG{l+m+mi}{0}\PYG{p}{]}\PYG{p}{,} \PYG{n}{solidvx}\PYG{o}{=}\PYG{p}{[}\PYG{l+m+mi}{233}\PYG{p}{,} \PYG{l+m+mi}{255}\PYG{p}{]}\PYG{p}{,} \PYG{n}{nlayers}\PYG{o}{=}\PYG{l+m+mi}{3}\PYG{p}{)}
\PYG{g+gp}{\PYGZgt{}\PYGZgt{}\PYGZgt{} }\PYG{n}{binary}\PYG{o}{.}\PYG{n}{binarized}
\PYG{g+gp}{\PYGZgt{}\PYGZgt{}\PYGZgt{} }\PYG{n}{HDF5\PYGZus{}write}\PYG{p}{(}\PYG{n}{binary}\PYG{o}{.}\PYG{n}{binarized}\PYG{p}{,} \PYG{n}{binary}\PYG{o}{.}\PYG{n}{porosity}\PYG{p}{,} \PYG{n}{binary}\PYG{o}{.}\PYG{n}{boundary}\PYG{p}{,} \PYG{l+s+s1}{\PYGZsq{}}\PYG{l+s+s1}{LBModel.hdf5}\PYG{l+s+s1}{\PYGZsq{}}\PYG{p}{)}
\end{sphinxVerbatim}
\index{BoundaryCondition (class in lb\_colloids.LB.LB\_2Dimage)}

\begin{fulllineitems}
\phantomsection\label{\detokenize{index:lb_colloids.LB.LB_2Dimage.BoundaryCondition}}\pysiglinewithargsret{\sphinxbfcode{class }\sphinxcode{lb\_colloids.LB.LB\_2Dimage.}\sphinxbfcode{BoundaryCondition}}{\emph{data}, \emph{fluidvx}, \emph{solidvx}, \emph{nlayers}, \emph{bottom=False}}{}
Class to instatiate open boundary layers at the top and bottom of the
lattice boltzmann image, closed boundary layers at either side of the
image, and binarize the array into a boolen type.
\begin{quote}\begin{description}
\item[{Parameters}] \leavevmode\begin{itemize}
\item {} 
\sphinxstyleliteralstrong{data} (\sphinxstyleliteralemphasis{np.ndarray}) \textendash{} NxN array of image data relating to the input file

\item {} 
\sphinxstyleliteralstrong{fluidvx} (\sphinxstyleliteralemphasis{list}) \textendash{} list of fluid voxel grey values

\item {} 
\sphinxstyleliteralstrong{solidvx} (\sphinxstyleliteralemphasis{list}) \textendash{} list of solid voxel grey values

\item {} 
\sphinxstyleliteralstrong{nlayers} (\sphinxstyleliteralemphasis{int}) \textendash{} number of open boundary layers to add to the top and
bottom of the image array.

\end{itemize}

\end{description}\end{quote}
\paragraph{Attributes}
\index{binarized (lb\_colloids.LB.LB\_2Dimage.BoundaryCondition attribute)}

\begin{fulllineitems}
\phantomsection\label{\detokenize{index:lb_colloids.LB.LB_2Dimage.BoundaryCondition.binarized}}\pysigline{\sphinxbfcode{binarized}}
\sphinxstyleemphasis{return} \textendash{} Binary image with boundary conditions applied

\end{fulllineitems}

\index{fluid\_voxels (lb\_colloids.LB.LB\_2Dimage.BoundaryCondition attribute)}

\begin{fulllineitems}
\phantomsection\label{\detokenize{index:lb_colloids.LB.LB_2Dimage.BoundaryCondition.fluid_voxels}}\pysigline{\sphinxbfcode{fluid\_voxels}}
\sphinxstyleemphasis{return} \textendash{} user defined fluid volxels

\end{fulllineitems}

\index{grey\_values (lb\_colloids.LB.LB\_2Dimage.BoundaryCondition attribute)}

\begin{fulllineitems}
\phantomsection\label{\detokenize{index:lb_colloids.LB.LB_2Dimage.BoundaryCondition.grey_values}}\pysigline{\sphinxbfcode{grey\_values}}
\sphinxstyleemphasis{return} \textendash{} unique image grayscale values

\end{fulllineitems}

\index{nlayers (lb\_colloids.LB.LB\_2Dimage.BoundaryCondition attribute)}

\begin{fulllineitems}
\phantomsection\label{\detokenize{index:lb_colloids.LB.LB_2Dimage.BoundaryCondition.nlayers}}\pysigline{\sphinxbfcode{nlayers}}
\sphinxstyleemphasis{return} \textendash{} Number of boundary condition layers

\end{fulllineitems}

\index{porosity (lb\_colloids.LB.LB\_2Dimage.BoundaryCondition attribute)}

\begin{fulllineitems}
\phantomsection\label{\detokenize{index:lb_colloids.LB.LB_2Dimage.BoundaryCondition.porosity}}\pysigline{\sphinxbfcode{porosity}}
\sphinxstyleemphasis{return} \textendash{} image porosity

\end{fulllineitems}

\index{solid\_voxels (lb\_colloids.LB.LB\_2Dimage.BoundaryCondition attribute)}

\begin{fulllineitems}
\phantomsection\label{\detokenize{index:lb_colloids.LB.LB_2Dimage.BoundaryCondition.solid_voxels}}\pysigline{\sphinxbfcode{solid\_voxels}}
\sphinxstyleemphasis{return} \textendash{} user defined solid voxels

\end{fulllineitems}


\end{fulllineitems}

\index{HDF5\_write (class in lb\_colloids.LB.LB\_2Dimage)}

\begin{fulllineitems}
\phantomsection\label{\detokenize{index:lb_colloids.LB.LB_2Dimage.HDF5_write}}\pysiglinewithargsret{\sphinxbfcode{class }\sphinxcode{lb\_colloids.LB.LB\_2Dimage.}\sphinxbfcode{HDF5\_write}}{\emph{arr}, \emph{porosity}, \emph{boundary}, \emph{output}}{}
Write class for LB2d\_image. Writes a HDF5 file that includes
the binary image, porosit \&, number of boundary layers,
\begin{quote}\begin{description}
\item[{Parameters}] \leavevmode\begin{itemize}
\item {} 
\sphinxstyleliteralstrong{arr} (\sphinxstyleliteralemphasis{np.ndarray}) \textendash{} binarized image data

\item {} 
\sphinxstyleliteralstrong{porosity} (\sphinxstyleliteralemphasis{float}) \textendash{} porosity of the image

\item {} 
\sphinxstyleliteralstrong{boundary} (\sphinxstyleliteralemphasis{int}) \textendash{} number of boundary layers

\item {} 
\sphinxstyleliteralstrong{output} (\sphinxstyleliteralemphasis{str}) \textendash{} output hdf5 file name

\end{itemize}

\end{description}\end{quote}

\end{fulllineitems}

\index{Images (class in lb\_colloids.LB.LB\_2Dimage)}

\begin{fulllineitems}
\phantomsection\label{\detokenize{index:lb_colloids.LB.LB_2Dimage.Images}}\pysiglinewithargsret{\sphinxbfcode{class }\sphinxcode{lb\_colloids.LB.LB\_2Dimage.}\sphinxbfcode{Images}}{\emph{infile}}{}
Convience class to open and adjust RGB images to BW if
necessary.
\begin{quote}\begin{description}
\item[{Parameters}] \leavevmode
\sphinxstyleliteralstrong{infile} (\sphinxstyleliteralemphasis{string}) \textendash{} binary input file name (image file domain)

\end{description}\end{quote}

\end{fulllineitems}

\index{run() (in module lb\_colloids.LB.LB\_2Dimage)}

\begin{fulllineitems}
\phantomsection\label{\detokenize{index:lb_colloids.LB.LB_2Dimage.run}}\pysiglinewithargsret{\sphinxcode{lb\_colloids.LB.LB\_2Dimage.}\sphinxbfcode{run}}{\emph{image}, \emph{solid}, \emph{fluid}, \emph{output}, \emph{boundary=5}}{}
Funcitonal approach to run the LB-Colloids script
\begin{quote}\begin{description}
\item[{Parameters}] \leavevmode\begin{itemize}
\item {} 
\sphinxstyleliteralstrong{image} (\sphinxstyleliteralemphasis{str}) \textendash{} image file name

\item {} 
\sphinxstyleliteralstrong{solid} (\sphinxstyleliteralemphasis{list}) \textendash{} list of interger grey scale values corresponding
to the solid phase

\item {} 
\sphinxstyleliteralstrong{fluid} (\sphinxstyleliteralemphasis{list}) \textendash{} list of interger grey scale values corresponding
to the fluid phase

\item {} 
\sphinxstyleliteralstrong{output} (\sphinxstyleliteralemphasis{str}) \textendash{} output hdf5 file name

\item {} 
\sphinxstyleliteralstrong{boundary} (\sphinxstyleliteralemphasis{int}) \textendash{} number of boundary layers for the model

\end{itemize}

\end{description}\end{quote}

\begin{sphinxVerbatim}[commandchars=\\\{\}]
\PYG{g+gp}{\PYGZgt{}\PYGZgt{}\PYGZgt{} }\PYG{n}{LBImage}\PYG{o}{.}\PYG{n}{run}\PYG{p}{(}\PYG{l+s+s1}{\PYGZsq{}}\PYG{l+s+s1}{my\PYGZus{}thin\PYGZus{}section.png}\PYG{l+s+s1}{\PYGZsq{}}\PYG{p}{,} \PYG{n}{fluid}\PYG{o}{=}\PYG{p}{[}\PYG{l+m+mi}{0}\PYG{p}{]}\PYG{p}{,} \PYG{n}{solid}\PYG{o}{=}\PYG{p}{[}\PYG{l+m+mi}{233}\PYG{p}{,} \PYG{l+m+mi}{255}\PYG{p}{]}\PYG{p}{,}
\PYG{g+gp}{\PYGZgt{}\PYGZgt{}\PYGZgt{} }            \PYG{n}{output}\PYG{o}{=}\PYG{l+s+s2}{\PYGZdq{}}\PYG{l+s+s2}{LBModel.hdf5}\PYG{l+s+s2}{\PYGZdq{}}\PYG{p}{,} \PYG{n}{nlayers}\PYG{o}{=}\PYG{l+m+mi}{3}\PYG{p}{)}
\PYG{g+go}{\PYGZgt{}\PYGZgt{}\PYGZgt{}}
\end{sphinxVerbatim}

\end{fulllineitems}



\section{Lattice Boltzmann Permeability}
\label{\detokenize{index:lattice-boltzmann-permeability}}
The lattice Boltzmann permeability modules contain the principle methods and subroutines for simulating fluid flow within a model domain. Two possible programatic styles can be used for simulating lattice Boltzmann fluid domain models. The python “kernel” is primarily used for delevopment and debugging purposes. The default FORTRAN “kernel” provides a compiled version of this code, which is executes at speeds of approximately 100x faster than the python version. The FORTRAN code is dynamically compiled to the local machine upon installation of LB-Colloids. If FORTRAN compilation fails, it is reccomended that the user use the python kernel until the compiling issue is resolved. Documentation of the lattice Boltzmann permeability modules does not include documentation of the FORTRAN subroutines or the python mathematical code. For an understanding of the mathematics behind the lattice Boltzmann method an examination of source code and additional reading into lattice Boltzmann literature is required.


\subsection{API documentation}
\label{\detokenize{index:id4}}\phantomsection\label{\detokenize{index:module-lb_colloids.LB.LB_2Dpermeability}}\index{lb\_colloids.LB.LB\_2Dpermeability (module)}
D2Q9 lattice Boltzmann simulations are performed using classes and methods
contained in this module. The LB2DModel class is the main class the user
will interact with within this module. LB2DModel calls a series of subroutines
to run and save the lattice Boltzmann simulation

Basic mathematical relationships have been implemented from relevant academic literature. For
a complete handling of the mathematics please Chen and Doolen 1996 is a great starting point. A listing
of mathematical relationships are provided here for completeness.
\begin{equation*}
\begin{split}\rho = \sum_{i=1}^{n} f_{i}\end{split}
\end{equation*}\begin{equation*}
\begin{split}\rho \pmb{u} = \sum_{i=1}^{n} f_{i} \pmb{e_{i}}\end{split}
\end{equation*}\begin{equation*}
\begin{split}v = \frac{1}{6} (\frac{2}{\tau}- 1)\end{split}
\end{equation*}\begin{equation*}
\begin{split}f_{i}(x + e_{i}, t + \Delta t) = f_{i}(x, t) - \frac{f_{i} - f_{i}^{eq}}{\tau}\end{split}
\end{equation*}\begin{equation*}
\begin{split}f_{i}^{eq} = \rho w_{i}[1 + 3\pmb{e_{i} \cdot u} +
\frac{9}{2}(\pmb{e_{i} \cdot u})^{2} - \frac{3}{2} u^{2}\end{split}
\end{equation*}
Although lattice Boltzmann mathematics are included within the python
methods contained within this module, it is highly reccomended that the user
use the default Fortran Kernal option to run LB models. The python kernal is approximately
100x slower than Fortran.

Example showing the build and run of a D2Q9 LB2DModel

\begin{sphinxVerbatim}[commandchars=\\\{\}]
\PYG{g+gp}{\PYGZgt{}\PYGZgt{}\PYGZgt{} }\PYG{k+kn}{from} \PYG{n+nn}{lb\PYGZus{}colloids} \PYG{k}{import} \PYG{n}{LB2DModel}
\PYG{g+gp}{\PYGZgt{}\PYGZgt{}\PYGZgt{} }\PYG{k+kn}{from} \PYG{n+nn}{lb\PYGZus{}colloids} \PYG{k}{import} \PYG{n}{LBImage}
\PYG{g+go}{\PYGZgt{}\PYGZgt{}\PYGZgt{}}
\PYG{g+gp}{\PYGZgt{}\PYGZgt{}\PYGZgt{} }\PYG{n}{image} \PYG{o}{=} \PYG{n}{LBImage}\PYG{o}{.}\PYG{n}{Images}\PYG{p}{(}\PYG{l+s+s2}{\PYGZdq{}}\PYG{l+s+s2}{my\PYGZus{}thin\PYGZus{}section.png}\PYG{l+s+s2}{\PYGZdq{}}\PYG{p}{)}
\PYG{g+gp}{\PYGZgt{}\PYGZgt{}\PYGZgt{} }\PYG{n}{binary} \PYG{o}{=} \PYG{n}{LBImage}\PYG{o}{.}\PYG{n}{BoundaryCondition}\PYG{p}{(}\PYG{n}{image}\PYG{p}{,} \PYG{n}{fluidvx}\PYG{o}{=}\PYG{p}{[}\PYG{l+m+mi}{0}\PYG{p}{]}\PYG{p}{,} \PYG{n}{solidvx}\PYG{o}{=}\PYG{p}{[}\PYG{l+m+mi}{233}\PYG{p}{,} \PYG{l+m+mi}{255}\PYG{p}{]}\PYG{p}{,} \PYG{n}{nlayers}\PYG{o}{=}\PYG{l+m+mi}{3}\PYG{p}{)}
\PYG{g+gp}{\PYGZgt{}\PYGZgt{}\PYGZgt{} }\PYG{n}{model} \PYG{o}{=} \PYG{n}{LB2DModel}\PYG{p}{(}\PYG{n}{img}\PYG{o}{=}\PYG{n}{binary}\PYG{o}{.}\PYG{n}{binarized}\PYG{p}{)}
\PYG{g+gp}{\PYGZgt{}\PYGZgt{}\PYGZgt{} }\PYG{n}{model}\PYG{o}{.}\PYG{n}{niters} \PYG{o}{=} \PYG{l+m+mi}{2000}
\PYG{g+gp}{\PYGZgt{}\PYGZgt{}\PYGZgt{} }\PYG{n}{model}\PYG{o}{.}\PYG{n}{rho} \PYG{o}{=} \PYG{l+m+mf}{1.0}
\PYG{g+gp}{\PYGZgt{}\PYGZgt{}\PYGZgt{} }\PYG{n}{model}\PYG{o}{.}\PYG{n}{tau} \PYG{o}{=} \PYG{l+m+mf}{0.8}
\PYG{g+gp}{\PYGZgt{}\PYGZgt{}\PYGZgt{} }\PYG{n}{result} \PYG{o}{=} \PYG{n}{model}\PYG{o}{.}\PYG{n}{run}\PYG{p}{(}\PYG{n}{output}\PYG{o}{=}\PYG{l+s+s2}{\PYGZdq{}}\PYG{l+s+s2}{LBModel.hdf5}\PYG{l+s+s2}{\PYGZdq{}}\PYG{p}{)}
\end{sphinxVerbatim}
\index{HDF5\_write (class in lb\_colloids.LB.LB\_2Dpermeability)}

\begin{fulllineitems}
\phantomsection\label{\detokenize{index:lb_colloids.LB.LB_2Dpermeability.HDF5_write}}\pysiglinewithargsret{\sphinxbfcode{class }\sphinxcode{lb\_colloids.LB.LB\_2Dpermeability.}\sphinxbfcode{HDF5\_write}}{\emph{mrho}, \emph{tau}, \emph{u}, \emph{f}, \emph{rho}, \emph{output}, \emph{mean\_uy}, \emph{mean\_ux}, \emph{pore\_diameter}, \emph{reynolds\_number}, \emph{velocity\_factor}, \emph{img=None}, \emph{porosity=None}, \emph{boundary=None}}{}
Hdf5 model write class to save simulation results from a LB Permeability model run
\begin{quote}\begin{description}
\item[{Parameters}] \leavevmode\begin{itemize}
\item {} 
\sphinxstyleliteralstrong{mrho} (\sphinxstyleliteralemphasis{float}) \textendash{} mean fluid density

\item {} 
\sphinxstyleliteralstrong{tau} (\sphinxstyleliteralemphasis{float}) \textendash{} lb relaxation time

\item {} 
\sphinxstyleliteralstrong{u} (\sphinxstyleliteralemphasis{np.ndarray}) \textendash{} fluid velocity array

\item {} 
\sphinxstyleliteralstrong{f} (\sphinxstyleliteralemphasis{np.ndarray}) \textendash{} distribution function

\item {} 
\sphinxstyleliteralstrong{rho} (\sphinxstyleliteralemphasis{np.ndarray}) \textendash{} density array

\item {} 
\sphinxstyleliteralstrong{output} (\sphinxstyleliteralemphasis{str}) \textendash{} hdf file name

\item {} 
\sphinxstyleliteralstrong{mean\_uy} (\sphinxstyleliteralemphasis{float}) \textendash{} mean fluid velocity y direction

\item {} 
\sphinxstyleliteralstrong{mean\_ux} (\sphinxstyleliteralemphasis{float}) \textendash{} mean fluid velocity x direction

\item {} 
\sphinxstyleliteralstrong{pore\_diameter} (\sphinxstyleliteralemphasis{float}) \textendash{} mean pore diameter

\item {} 
\sphinxstyleliteralstrong{reynolds\_number} (\sphinxstyleliteralemphasis{float}) \textendash{} calculated reynolds number

\item {} 
\sphinxstyleliteralstrong{velocity\_factor} (\sphinxstyleliteralemphasis{float}) \textendash{} non-dimensional to dimensional velocity conversion factor

\item {} 
\sphinxstyleliteralstrong{img} (\sphinxstyleliteralemphasis{float}) \textendash{} binary image array

\item {} 
\sphinxstyleliteralstrong{porosity} (\sphinxstyleliteralemphasis{float}) \textendash{} porosity

\item {} 
\sphinxstyleliteralstrong{boundary} (\sphinxstyleliteralemphasis{float}) \textendash{} nlayers boundaty condition

\end{itemize}

\end{description}\end{quote}

\end{fulllineitems}

\index{LB2DModel (class in lb\_colloids.LB.LB\_2Dpermeability)}

\begin{fulllineitems}
\phantomsection\label{\detokenize{index:lb_colloids.LB.LB_2Dpermeability.LB2DModel}}\pysiglinewithargsret{\sphinxbfcode{class }\sphinxcode{lb\_colloids.LB.LB\_2Dpermeability.}\sphinxbfcode{LB2DModel}}{\emph{img}, \emph{kernel='fortran'}}{}
object oriented method to instantiate and run a Two-Dimensional
lattice boltzmann model. Calls upon either fortran or python
kernels to run a model.

Please reference documentation for
\sphinxstyleemphasis{lb\_colloids.LB.LB\_2Dpermeability.}\sphinxstylestrong{LB2DModel}
for a full listing of attributes and methods.

Uses protective programming to ensure user supplied data fits within normal model
parameters.

All attributes can be reset before a model run by passing a valid value to them
\begin{quote}\begin{description}
\item[{Parameters}] \leavevmode\begin{itemize}
\item {} 
\sphinxstyleliteralstrong{img} (\sphinxstyleliteralemphasis{np.ndarray}) \textendash{} binarized image array from LBImage

\item {} 
\sphinxstyleliteralstrong{kernel} (\sphinxstyleliteralemphasis{str}) \textendash{} the simulation kernel. Default is fortran

\end{itemize}

\end{description}\end{quote}
\begin{quote}

run:  method to run the lb model and return a distribution function
\end{quote}
\paragraph{Attributes}
\paragraph{Methods}
\index{cs (lb\_colloids.LB.LB\_2Dpermeability.LB2DModel attribute)}

\begin{fulllineitems}
\phantomsection\label{\detokenize{index:lb_colloids.LB.LB_2Dpermeability.LB2DModel.cs}}\pysigline{\sphinxbfcode{cs}}
\sphinxstyleemphasis{return} \textendash{} Lattice speed of sound

\end{fulllineitems}

\index{cs2 (lb\_colloids.LB.LB\_2Dpermeability.LB2DModel attribute)}

\begin{fulllineitems}
\phantomsection\label{\detokenize{index:lb_colloids.LB.LB_2Dpermeability.LB2DModel.cs2}}\pysigline{\sphinxbfcode{cs2}}
\sphinxstyleemphasis{return} \textendash{} Lattice speed of sound ** 2

\end{fulllineitems}

\index{get\_mean\_pore\_size() (lb\_colloids.LB.LB\_2Dpermeability.LB2DModel method)}

\begin{fulllineitems}
\phantomsection\label{\detokenize{index:lb_colloids.LB.LB_2Dpermeability.LB2DModel.get_mean_pore_size}}\pysiglinewithargsret{\sphinxbfcode{get\_mean\_pore\_size}}{}{}~\begin{quote}\begin{description}
\item[{Returns}] \leavevmode
the mean pore diameter of the domain

\end{description}\end{quote}

\end{fulllineitems}

\index{get\_reynolds\_number() (lb\_colloids.LB.LB\_2Dpermeability.LB2DModel method)}

\begin{fulllineitems}
\phantomsection\label{\detokenize{index:lb_colloids.LB.LB_2Dpermeability.LB2DModel.get_reynolds_number}}\pysiglinewithargsret{\sphinxbfcode{get\_reynolds\_number}}{}{}~\begin{quote}\begin{description}
\item[{Returns}] \leavevmode
the model’s reynolds number after simulation

\end{description}\end{quote}

\end{fulllineitems}

\index{get\_velocity\_conversion() (lb\_colloids.LB.LB\_2Dpermeability.LB2DModel method)}

\begin{fulllineitems}
\phantomsection\label{\detokenize{index:lb_colloids.LB.LB_2Dpermeability.LB2DModel.get_velocity_conversion}}\pysiglinewithargsret{\sphinxbfcode{get\_velocity\_conversion}}{}{}~\begin{quote}\begin{description}
\item[{Returns}] \leavevmode
the conversion factor from LB velocity to Phys.

\end{description}\end{quote}

\end{fulllineitems}

\index{gravity (lb\_colloids.LB.LB\_2Dpermeability.LB2DModel attribute)}

\begin{fulllineitems}
\phantomsection\label{\detokenize{index:lb_colloids.LB.LB_2Dpermeability.LB2DModel.gravity}}\pysigline{\sphinxbfcode{gravity}}
\sphinxstyleemphasis{return} \textendash{} Body force applied to simulation

\end{fulllineitems}

\index{img (lb\_colloids.LB.LB\_2Dpermeability.LB2DModel attribute)}

\begin{fulllineitems}
\phantomsection\label{\detokenize{index:lb_colloids.LB.LB_2Dpermeability.LB2DModel.img}}\pysigline{\sphinxbfcode{img}}
\sphinxstyleemphasis{return} \textendash{} Binarized lattice boltzmann fluid domain

\end{fulllineitems}

\index{kernel (lb\_colloids.LB.LB\_2Dpermeability.LB2DModel attribute)}

\begin{fulllineitems}
\phantomsection\label{\detokenize{index:lb_colloids.LB.LB_2Dpermeability.LB2DModel.kernel}}\pysigline{\sphinxbfcode{kernel}}
\sphinxstyleemphasis{return} \textendash{} Kernel type (Python or Fortran)

\end{fulllineitems}

\index{niters (lb\_colloids.LB.LB\_2Dpermeability.LB2DModel attribute)}

\begin{fulllineitems}
\phantomsection\label{\detokenize{index:lb_colloids.LB.LB_2Dpermeability.LB2DModel.niters}}\pysigline{\sphinxbfcode{niters}}
\sphinxstyleemphasis{return} \textendash{} Number of lattice Boltzmann time steps to be simulated

\end{fulllineitems}

\index{nlayers (lb\_colloids.LB.LB\_2Dpermeability.LB2DModel attribute)}

\begin{fulllineitems}
\phantomsection\label{\detokenize{index:lb_colloids.LB.LB_2Dpermeability.LB2DModel.nlayers}}\pysigline{\sphinxbfcode{nlayers}}
\sphinxstyleemphasis{return} \textendash{} Calculated number of boundary layers applied to the simulation domain

\end{fulllineitems}

\index{nx (lb\_colloids.LB.LB\_2Dpermeability.LB2DModel attribute)}

\begin{fulllineitems}
\phantomsection\label{\detokenize{index:lb_colloids.LB.LB_2Dpermeability.LB2DModel.nx}}\pysigline{\sphinxbfcode{nx}}
\sphinxstyleemphasis{return} \textendash{} Model size in the x-direction

\end{fulllineitems}

\index{ny (lb\_colloids.LB.LB\_2Dpermeability.LB2DModel attribute)}

\begin{fulllineitems}
\phantomsection\label{\detokenize{index:lb_colloids.LB.LB_2Dpermeability.LB2DModel.ny}}\pysigline{\sphinxbfcode{ny}}
\sphinxstyleemphasis{return} \textendash{} Model size in the y-direction

\end{fulllineitems}

\index{physical\_rho (lb\_colloids.LB.LB\_2Dpermeability.LB2DModel attribute)}

\begin{fulllineitems}
\phantomsection\label{\detokenize{index:lb_colloids.LB.LB_2Dpermeability.LB2DModel.physical_rho}}\pysigline{\sphinxbfcode{physical\_rho}}
\sphinxstyleemphasis{return} \textendash{} Physical density of the simulation fluid

\end{fulllineitems}

\index{physical\_viscosity (lb\_colloids.LB.LB\_2Dpermeability.LB2DModel attribute)}

\begin{fulllineitems}
\phantomsection\label{\detokenize{index:lb_colloids.LB.LB_2Dpermeability.LB2DModel.physical_viscosity}}\pysigline{\sphinxbfcode{physical\_viscosity}}
\sphinxstyleemphasis{return} \textendash{} Physical viscosity of the simulation fluid

\end{fulllineitems}

\index{porosity (lb\_colloids.LB.LB\_2Dpermeability.LB2DModel attribute)}

\begin{fulllineitems}
\phantomsection\label{\detokenize{index:lb_colloids.LB.LB_2Dpermeability.LB2DModel.porosity}}\pysigline{\sphinxbfcode{porosity}}
\sphinxstyleemphasis{return} \textendash{} calculated porosity of the simulation domain

\end{fulllineitems}

\index{q (lb\_colloids.LB.LB\_2Dpermeability.LB2DModel attribute)}

\begin{fulllineitems}
\phantomsection\label{\detokenize{index:lb_colloids.LB.LB_2Dpermeability.LB2DModel.q}}\pysigline{\sphinxbfcode{q}}
\sphinxstyleemphasis{return} \textendash{} number of simulation fluid nodes

\end{fulllineitems}

\index{resolution (lb\_colloids.LB.LB\_2Dpermeability.LB2DModel attribute)}

\begin{fulllineitems}
\phantomsection\label{\detokenize{index:lb_colloids.LB.LB_2Dpermeability.LB2DModel.resolution}}\pysigline{\sphinxbfcode{resolution}}
\sphinxstyleemphasis{return} \textendash{} Model resolution in physical units (meters)

\end{fulllineitems}

\index{rho (lb\_colloids.LB.LB\_2Dpermeability.LB2DModel attribute)}

\begin{fulllineitems}
\phantomsection\label{\detokenize{index:lb_colloids.LB.LB_2Dpermeability.LB2DModel.rho}}\pysigline{\sphinxbfcode{rho}}
\sphinxstyleemphasis{return} \textendash{} Non-dimensional fluid density

\end{fulllineitems}

\index{run() (lb\_colloids.LB.LB\_2Dpermeability.LB2DModel method)}

\begin{fulllineitems}
\phantomsection\label{\detokenize{index:lb_colloids.LB.LB_2Dpermeability.LB2DModel.run}}\pysiglinewithargsret{\sphinxbfcode{run}}{\emph{output='LBModel.hdf5'}, \emph{image\_int=None}, \emph{image\_folder=None}, \emph{image\_name='LB\_'}, \emph{vmax=0}, \emph{vmin=-0.01}, \emph{verbose=None}}{}
user method to run the lattice Boltzmann model and return the resulting
distribution function.

\end{fulllineitems}

\index{tau (lb\_colloids.LB.LB\_2Dpermeability.LB2DModel attribute)}

\begin{fulllineitems}
\phantomsection\label{\detokenize{index:lb_colloids.LB.LB_2Dpermeability.LB2DModel.tau}}\pysigline{\sphinxbfcode{tau}}
\sphinxstyleemphasis{return} \textendash{} Simulation relaxation time

\end{fulllineitems}

\index{viscosity (lb\_colloids.LB.LB\_2Dpermeability.LB2DModel attribute)}

\begin{fulllineitems}
\phantomsection\label{\detokenize{index:lb_colloids.LB.LB_2Dpermeability.LB2DModel.viscosity}}\pysigline{\sphinxbfcode{viscosity}}
\sphinxstyleemphasis{return} \textendash{} Calculated lattice Boltzmann viscosity for the simulation

\end{fulllineitems}


\end{fulllineitems}

\index{darcy\_velocity() (in module lb\_colloids.LB.LB\_2Dpermeability)}

\begin{fulllineitems}
\phantomsection\label{\detokenize{index:lb_colloids.LB.LB_2Dpermeability.darcy_velocity}}\pysiglinewithargsret{\sphinxcode{lb\_colloids.LB.LB\_2Dpermeability.}\sphinxbfcode{darcy\_velocity}}{\emph{x}, \emph{img}, \emph{nbound}}{}
Method to get the darcy velocity of the steady state lb model
based on outflow velocity
:param np.ndarray x: macroscopic velocity in the y-direction
:param np.ndarray img: image array corresponding to model domain
:param int nbound: number of boundary layers applied to model domain
\begin{quote}\begin{description}
\item[{Returns}] \leavevmode
darcy velocity

\end{description}\end{quote}

\end{fulllineitems}

\index{get\_mean\_pore\_size() (in module lb\_colloids.LB.LB\_2Dpermeability)}

\begin{fulllineitems}
\phantomsection\label{\detokenize{index:lb_colloids.LB.LB_2Dpermeability.get_mean_pore_size}}\pysiglinewithargsret{\sphinxcode{lb\_colloids.LB.LB\_2Dpermeability.}\sphinxbfcode{get\_mean\_pore\_size}}{\emph{img}, \emph{nx}}{}
Finds the mean pore diameter of the domain
\begin{quote}\begin{description}
\item[{Parameters}] \leavevmode\begin{itemize}
\item {} 
\sphinxstyleliteralstrong{img} (\sphinxstyleliteralemphasis{np.ndarray}) \textendash{} binary image array of domain

\item {} 
\sphinxstyleliteralstrong{nx} (\sphinxstyleliteralemphasis{int}) \textendash{} number of pixels in x direction of fluid domain

\end{itemize}

\item[{Returns}] \leavevmode\begin{quote}\begin{description}
\item[{return}] \leavevmode
Model domain mean pore size

\end{description}\end{quote}

\end{description}\end{quote}

\end{fulllineitems}

\index{get\_reynolds\_number() (in module lb\_colloids.LB.LB\_2Dpermeability)}

\begin{fulllineitems}
\phantomsection\label{\detokenize{index:lb_colloids.LB.LB_2Dpermeability.get_reynolds_number}}\pysiglinewithargsret{\sphinxcode{lb\_colloids.LB.LB\_2Dpermeability.}\sphinxbfcode{get\_reynolds\_number}}{\emph{pore\_diameter}, \emph{uy}, \emph{porosity}, \emph{rho}, \emph{viscosity}}{}
Calculate the model’s reynolds number after simulation
based off of mean velocity and mean fluid density
\begin{quote}\begin{description}
\item[{Parameters}] \leavevmode\begin{itemize}
\item {} 
\sphinxstyleliteralstrong{pore\_diameter} (\sphinxstyleliteralemphasis{float}) \textendash{} calculated lb pore diameter

\item {} 
\sphinxstyleliteralstrong{uy} (\sphinxstyleliteralemphasis{float}) \textendash{} lb mean fluid velocity in y direction

\item {} 
\sphinxstyleliteralstrong{porosity} (\sphinxstyleliteralemphasis{float}) \textendash{} porosity of the medium

\item {} 
\sphinxstyleliteralstrong{rho} (\sphinxstyleliteralemphasis{float}) \textendash{} lb mean fluid density

\item {} 
\sphinxstyleliteralstrong{viscosity} (\sphinxstyleliteralemphasis{float}) \textendash{} lb fluid viscosity

\end{itemize}

\end{description}\end{quote}
\begin{quote}\begin{description}
\item[{Returns}] \leavevmode
Simulation Reynolds number

\end{description}\end{quote}

\end{fulllineitems}

\index{get\_velocity\_conversion() (in module lb\_colloids.LB.LB\_2Dpermeability)}

\begin{fulllineitems}
\phantomsection\label{\detokenize{index:lb_colloids.LB.LB_2Dpermeability.get_velocity_conversion}}\pysiglinewithargsret{\sphinxcode{lb\_colloids.LB.LB\_2Dpermeability.}\sphinxbfcode{get\_velocity\_conversion}}{\emph{reynolds\_number}, \emph{uy}, \emph{rho}, \emph{pore\_diameter}, \emph{viscosity}}{}
Calculates the physical velocity conversion factor from
LB reynolds number and user supplied physical parameters.
\begin{quote}\begin{description}
\item[{Parameters}] \leavevmode\begin{itemize}
\item {} 
\sphinxstyleliteralstrong{reynolds\_number} (\sphinxstyleliteralemphasis{float}) \textendash{} Fluid simulations reynolds number

\item {} 
\sphinxstyleliteralstrong{uy} (\sphinxstyleliteralemphasis{float}) \textendash{} mean y velocity from LB model

\item {} 
\sphinxstyleliteralstrong{rho} (\sphinxstyleliteralemphasis{float}) \textendash{} physical density of the fluid

\item {} 
\sphinxstyleliteralstrong{pore\_diameter} (\sphinxstyleliteralemphasis{float}) \textendash{} physical pore diameter

\item {} 
\sphinxstyleliteralstrong{viscosity} (\sphinxstyleliteralemphasis{float}) \textendash{} physical fluid viscosity

\end{itemize}

\end{description}\end{quote}
\begin{quote}\begin{description}
\item[{Returns}] \leavevmode
Non-dimensional fluid velocity to physical fluid velocity conversion factor

\end{description}\end{quote}

\end{fulllineitems}

\phantomsection\label{\detokenize{index:module-lb_colloids}}\index{lb\_colloids (module)}\index{LB2DModel (class in lb\_colloids)}

\begin{fulllineitems}
\phantomsection\label{\detokenize{index:lb_colloids.LB2DModel}}\pysiglinewithargsret{\sphinxbfcode{class }\sphinxcode{lb\_colloids.}\sphinxbfcode{LB2DModel}}{\emph{img}, \emph{kernel='fortran'}}{}
object oriented method to instantiate and run a Two-Dimensional
lattice boltzmann model. Calls upon either fortran or python
kernels to run a model.

Please reference documentation for
\sphinxstyleemphasis{lb\_colloids.LB.LB\_2Dpermeability.}\sphinxstylestrong{LB2DModel}
for a full listing of attributes and methods.

Uses protective programming to ensure user supplied data fits within normal model
parameters.

All attributes can be reset before a model run by passing a valid value to them
\begin{quote}\begin{description}
\item[{Parameters}] \leavevmode\begin{itemize}
\item {} 
\sphinxstyleliteralstrong{img} (\sphinxstyleliteralemphasis{np.ndarray}) \textendash{} binarized image array from LBImage

\item {} 
\sphinxstyleliteralstrong{kernel} (\sphinxstyleliteralemphasis{str}) \textendash{} the simulation kernel. Default is fortran

\end{itemize}

\end{description}\end{quote}
\begin{quote}

run:  method to run the lb model and return a distribution function
\end{quote}
\paragraph{Attributes}
\paragraph{Methods}

\end{fulllineitems}



\section{Lattice Boltzmann Input Output}
\label{\detokenize{index:lattice-boltzmann-input-output}}
Lattice Boltzmann input/output (\sphinxstyleemphasis{IO}) modules have been developed to facilitate end user functionality, such as reading configuration files and creating lattice Boltzmann fluid velocity images. The lattice Boltzmann IO module structure is a late addition to the project and is still in it’s infancy, although the packaged code is completely stable. All lattice Boltzmann input and output read and write objects will  eventually be migrated to this set of modules.


\subsection{API documentation}
\label{\detokenize{index:id5}}\phantomsection\label{\detokenize{index:module-lb_colloids}}\index{lb\_colloids (module)}\index{lbIO (in module lb\_colloids)}

\begin{fulllineitems}
\phantomsection\label{\detokenize{index:lb_colloids.lbIO}}\pysigline{\sphinxcode{lb\_colloids.}\sphinxbfcode{lbIO}}
alias of {\hyperref[\detokenize{index:module-lb_colloids.LB.LBIO}]{\sphinxcrossref{\sphinxcode{lb\_colloids.LB.LBIO}}}}

\end{fulllineitems}

\phantomsection\label{\detokenize{index:module-lb_colloids.LB.LBIO}}\index{lb\_colloids.LB.LBIO (module)}
lbIO is the lattice Boltzmann input control module. This module contains a Config class that
reads in the lattice Boltzmann configuration file and parses the parameters
contained within that file. HDF5 write options will eventually be moved from other
portions of the lattice Boltzmann modeling system to LBIO.py

Example import of a lattice Boltzmann configuration file is shown below

\begin{sphinxVerbatim}[commandchars=\\\{\}]
\PYG{g+gp}{\PYGZgt{}\PYGZgt{}\PYGZgt{} }\PYG{k+kn}{from} \PYG{n+nn}{lb\PYGZus{}colloids} \PYG{k}{import} \PYG{n}{lbIO}
\PYG{g+gp}{\PYGZgt{}\PYGZgt{}\PYGZgt{} }\PYG{n}{config\PYGZus{}file} \PYG{o}{=} \PYG{l+s+s2}{\PYGZdq{}}\PYG{l+s+s2}{LB\PYGZus{}model.config}\PYG{l+s+s2}{\PYGZdq{}}
\PYG{g+gp}{\PYGZgt{}\PYGZgt{}\PYGZgt{} }\PYG{n}{config} \PYG{o}{=} \PYG{n}{lbIO}\PYG{o}{.}\PYG{n}{Config}\PYG{p}{(}\PYG{n}{config\PYGZus{}file}\PYG{p}{)}
\PYG{g+gp}{\PYGZgt{}\PYGZgt{}\PYGZgt{} }\PYG{n}{model\PYGZus{}dict} \PYG{o}{=} \PYG{n}{config}\PYG{o}{.}\PYG{n}{model\PYGZus{}parameters}\PYG{p}{(}\PYG{p}{)}
\PYG{g+gp}{\PYGZgt{}\PYGZgt{}\PYGZgt{} }\PYG{n}{image\PYGZus{}dict} \PYG{o}{=} \PYG{n}{config}\PYG{o}{.}\PYG{n}{image\PYGZus{}parameters}\PYG{p}{(}\PYG{p}{)}
\PYG{g+gp}{\PYGZgt{}\PYGZgt{}\PYGZgt{} }\PYG{n}{permeability\PYGZus{}dict} \PYG{o}{=} \PYG{n}{config}\PYG{o}{.}\PYG{n}{permeability\PYGZus{}parameters}\PYG{p}{(}\PYG{p}{)}
\PYG{g+gp}{\PYGZgt{}\PYGZgt{}\PYGZgt{} }\PYG{n}{output\PYGZus{}dict} \PYG{o}{=} \PYG{n}{config}\PYG{o}{.}\PYG{n}{output\PYGZus{}parameters}\PYG{p}{(}\PYG{p}{)}
\end{sphinxVerbatim}
\index{Config (class in lb\_colloids.LB.LBIO)}

\begin{fulllineitems}
\phantomsection\label{\detokenize{index:lb_colloids.LB.LBIO.Config}}\pysiglinewithargsret{\sphinxbfcode{class }\sphinxcode{lb\_colloids.LB.LBIO.}\sphinxbfcode{Config}}{\emph{fname}}{}
Class to handle the input output control of the lattice botzmann modeling
system. Can be added to, and uses consistancy checks to ensure correct
types and required values are present for LBModel to run.
\begin{quote}\begin{description}
\item[{Parameters}] \leavevmode
\sphinxstyleliteralstrong{fname} (\sphinxstyleliteralemphasis{str}) \textendash{} configuration file name

\end{description}\end{quote}
\paragraph{Methods}
\index{check\_if\_valid() (lb\_colloids.LB.LBIO.Config method)}

\begin{fulllineitems}
\phantomsection\label{\detokenize{index:lb_colloids.LB.LBIO.Config.check_if_valid}}\pysiglinewithargsret{\sphinxbfcode{check\_if\_valid}}{\emph{blockname}, \emph{pname}, \emph{validparams}}{}
Method checks if a specific parameter is valid for the block it was supplied
in the configuration file.
\begin{quote}\begin{description}
\item[{Parameters}] \leavevmode\begin{itemize}
\item {} 
\sphinxstyleliteralstrong{blockname} (\sphinxstyleliteralemphasis{str}) \textendash{} the name of the configuration block

\item {} 
\sphinxstyleliteralstrong{pname} (\sphinxstyleliteralemphasis{str}) \textendash{} parameter name

\item {} 
\sphinxstyleliteralstrong{validparams} (\sphinxstyleliteralemphasis{tupele}) \textendash{} tuple of valid parameters for the specific configuration block

\end{itemize}

\end{description}\end{quote}

\end{fulllineitems}

\index{check\_model\_parameters() (lb\_colloids.LB.LBIO.Config method)}

\begin{fulllineitems}
\phantomsection\label{\detokenize{index:lb_colloids.LB.LBIO.Config.check_model_parameters}}\pysiglinewithargsret{\sphinxbfcode{check\_model\_parameters}}{\emph{ModelDict}}{}
Check for required parameters. If not present inform the user which parameter(s) are needed
\begin{quote}\begin{description}
\item[{Parameters}] \leavevmode
\sphinxstyleliteralstrong{ModelDict} (\sphinxstyleliteralemphasis{dict}) \textendash{} Model Input block dictionary

\end{description}\end{quote}

\end{fulllineitems}

\index{get\_block() (lb\_colloids.LB.LBIO.Config method)}

\begin{fulllineitems}
\phantomsection\label{\detokenize{index:lb_colloids.LB.LBIO.Config.get_block}}\pysiglinewithargsret{\sphinxbfcode{get\_block}}{\emph{blockname}}{}
Method to locate the beginning and end of each input block
\begin{quote}\begin{description}
\item[{Parameters}] \leavevmode
\sphinxstyleliteralstrong{blockname} (\sphinxstyleliteralemphasis{str}) \textendash{} blockname of a parameters block in the config file.

\end{description}\end{quote}
\begin{quote}\begin{description}
\item[{Returns}] \leavevmode
(list) returns a list of parameters contained within the block

\end{description}\end{quote}

\end{fulllineitems}

\index{image\_parameters() (lb\_colloids.LB.LBIO.Config method)}

\begin{fulllineitems}
\phantomsection\label{\detokenize{index:lb_colloids.LB.LBIO.Config.image_parameters}}\pysiglinewithargsret{\sphinxbfcode{image\_parameters}}{}{}
reads the IMAGE PARAMETERS block of the configuration file and creates
the ModelDict which contains the required parameters to run LB-Colloid
\begin{quote}\begin{description}
\item[{Returns}] \leavevmode
(dict) Dictionary containing image paramaters for simulation

\end{description}\end{quote}

\end{fulllineitems}

\index{model\_parameters() (lb\_colloids.LB.LBIO.Config method)}

\begin{fulllineitems}
\phantomsection\label{\detokenize{index:lb_colloids.LB.LBIO.Config.model_parameters}}\pysiglinewithargsret{\sphinxbfcode{model\_parameters}}{}{}
reads the MODEL PARAMETERS block of the configuration file and creates
the ModelDict which contains the required parameters to run LB-Colloid
\begin{quote}\begin{description}
\item[{Returns}] \leavevmode
(dict) Dictionary containing model parameter block information

\end{description}\end{quote}

\end{fulllineitems}

\index{output\_parameters() (lb\_colloids.LB.LBIO.Config method)}

\begin{fulllineitems}
\phantomsection\label{\detokenize{index:lb_colloids.LB.LBIO.Config.output_parameters}}\pysiglinewithargsret{\sphinxbfcode{output\_parameters}}{}{}
reads the OUTPUT CONTROL block of the configuration file and creates
the ModelDict which contains the required parameters to run LB-Colloid
\begin{quote}\begin{description}
\item[{Returns}] \leavevmode
(dict) Dictionary containing output control parameters

\end{description}\end{quote}

\end{fulllineitems}

\index{parametertype() (lb\_colloids.LB.LBIO.Config method)}

\begin{fulllineitems}
\phantomsection\label{\detokenize{index:lb_colloids.LB.LBIO.Config.parametertype}}\pysiglinewithargsret{\sphinxbfcode{parametertype}}{\emph{parameter}}{}
Method takes a parameter string, splits it, and sets the parameter type
\begin{quote}\begin{description}
\item[{Parameters}] \leavevmode
\sphinxstyleliteralstrong{parameter} (\sphinxstyleliteralemphasis{str}) \textendash{} string of “\textless{}pname\textgreater{}: \textless{}param\textgreater{}”

\end{description}\end{quote}
\begin{quote}\begin{description}
\item[{Returns}] \leavevmode
(tuple) parameter name, parameter associated with pname

\end{description}\end{quote}

\end{fulllineitems}

\index{permeability\_parameters() (lb\_colloids.LB.LBIO.Config method)}

\begin{fulllineitems}
\phantomsection\label{\detokenize{index:lb_colloids.LB.LBIO.Config.permeability_parameters}}\pysiglinewithargsret{\sphinxbfcode{permeability\_parameters}}{}{}
reads the PERMEABILITY PARAMETERS block of the configuration file and creates
the ModelDict which contains the required parameters to run LB-Colloid
\begin{quote}\begin{description}
\item[{Returns}] \leavevmode
(dict) Dictionary containing parameters from the Permeability input block

\end{description}\end{quote}

\end{fulllineitems}


\end{fulllineitems}

\index{LB\_pretty (in module lb\_colloids.LB)}

\begin{fulllineitems}
\phantomsection\label{\detokenize{index:lb_colloids.LB.LB_pretty}}\pysigline{\sphinxcode{lb\_colloids.LB.}\sphinxbfcode{LB\_pretty}}
alias of {\hyperref[\detokenize{index:module-lb_colloids.LB.LB_pretty}]{\sphinxcrossref{\sphinxcode{lb\_colloids.LB.LB\_pretty}}}}

\end{fulllineitems}

\phantomsection\label{\detokenize{index:module-lb_colloids.LB.LB_pretty}}\index{lb\_colloids.LB.LB\_pretty (module)}
LB\_pretty contains methods to format and generate matplotlib images of
lattice Boltzmann velocity results. These methods are called internally
in LB2DModel if the user specifies plotting within the output dictionary.

example code to save an image of fluid magnitude within a domain from hdf5 output

\begin{sphinxVerbatim}[commandchars=\\\{\}]
\PYG{g+gp}{\PYGZgt{}\PYGZgt{}\PYGZgt{} }\PYG{k+kn}{from} \PYG{n+nn}{lb\PYGZus{}colloids}\PYG{n+nn}{.}\PYG{n+nn}{LB}\PYG{n+nn}{.}\PYG{n+nn}{LB\PYGZus{}pretty} \PYG{k}{import} \PYG{n}{velocity\PYGZus{}image}
\PYG{g+gp}{\PYGZgt{}\PYGZgt{}\PYGZgt{} }\PYG{k+kn}{from} \PYG{n+nn}{lb\PYGZus{}colloids} \PYG{k}{import} \PYG{n}{ColloidOutput}
\PYG{g+go}{\PYGZgt{}\PYGZgt{}\PYGZgt{}}
\PYG{g+gp}{\PYGZgt{}\PYGZgt{}\PYGZgt{} }\PYG{n}{model} \PYG{o}{=} \PYG{n}{ColloidOutput}\PYG{o}{.}\PYG{n}{Hdf5Reader}\PYG{p}{(}\PYG{l+s+s2}{\PYGZdq{}}\PYG{l+s+s2}{LB\PYGZus{}model.hdf5}\PYG{l+s+s2}{\PYGZdq{}}\PYG{p}{)}
\PYG{g+gp}{\PYGZgt{}\PYGZgt{}\PYGZgt{} }\PYG{n}{u\PYGZus{}y} \PYG{o}{=} \PYG{n}{model}\PYG{o}{.}\PYG{n}{get\PYGZus{}data}\PYG{p}{(}\PYG{l+s+s2}{\PYGZdq{}}\PYG{l+s+s2}{lb\PYGZus{}velocity\PYGZus{}y}\PYG{l+s+s2}{\PYGZdq{}}\PYG{p}{)}
\PYG{g+gp}{\PYGZgt{}\PYGZgt{}\PYGZgt{} }\PYG{n}{u\PYGZus{}x} \PYG{o}{=} \PYG{n}{model}\PYG{o}{.}\PYG{n}{get\PYGZus{}data}\PYG{p}{(}\PYG{l+s+s2}{\PYGZdq{}}\PYG{l+s+s2}{lb\PYGZus{}velocity\PYGZus{}x}\PYG{l+s+s2}{\PYGZdq{}}\PYG{p}{)}
\PYG{g+gp}{\PYGZgt{}\PYGZgt{}\PYGZgt{} }\PYG{n}{image} \PYG{o}{=} \PYG{n}{model}\PYG{o}{.}\PYG{n}{get\PYGZus{}data}\PYG{p}{(}\PYG{l+s+s2}{\PYGZdq{}}\PYG{l+s+s2}{image}\PYG{l+s+s2}{\PYGZdq{}}\PYG{p}{)}
\PYG{g+gp}{\PYGZgt{}\PYGZgt{}\PYGZgt{} }\PYG{n}{u} \PYG{o}{=} \PYG{n}{np}\PYG{o}{.}\PYG{n}{array}\PYG{p}{(}\PYG{p}{[}\PYG{n}{u\PYGZus{}y}\PYG{p}{,} \PYG{n}{u\PYGZus{}x}\PYG{p}{]}\PYG{p}{)}
\PYG{g+gp}{\PYGZgt{}\PYGZgt{}\PYGZgt{} }\PYG{n}{velocity\PYGZus{}image}\PYG{p}{(}\PYG{n}{u}\PYG{p}{,} \PYG{n}{image}\PYG{p}{,} \PYG{l+s+s2}{\PYGZdq{}}\PYG{l+s+s2}{LB}\PYG{l+s+s2}{\PYGZdq{}}\PYG{p}{,} \PYG{l+m+mi}{1000}\PYG{p}{,} \PYG{n}{vel}\PYG{o}{=}\PYG{k+kc}{False}\PYG{p}{,} \PYG{n}{vmin}\PYG{o}{=}\PYG{l+m+mf}{0.0001}\PYG{p}{,} \PYG{n}{vmax}\PYG{o}{=}\PYG{l+m+mf}{0.1}\PYG{p}{)}
\end{sphinxVerbatim}
\index{set\_colormap() (in module lb\_colloids.LB.LB\_pretty)}

\begin{fulllineitems}
\phantomsection\label{\detokenize{index:lb_colloids.LB.LB_pretty.set_colormap}}\pysiglinewithargsret{\sphinxcode{lb\_colloids.LB.LB\_pretty.}\sphinxbfcode{set\_colormap}}{\emph{vel}}{}
Method to set the matplotlib colormap option based
on python version and datatype
\begin{quote}\begin{description}
\item[{Parameters}] \leavevmode
\sphinxstyleliteralstrong{vel} (\sphinxstyleliteralemphasis{bool}) \textendash{} Are we plotting a yvelocity image

\end{description}\end{quote}
\begin{quote}\begin{description}
\item[{Returns}] \leavevmode
(object) Matplotlib colormap object

\end{description}\end{quote}

\end{fulllineitems}

\index{velocity\_image() (in module lb\_colloids.LB.LB\_pretty)}

\begin{fulllineitems}
\phantomsection\label{\detokenize{index:lb_colloids.LB.LB_pretty.velocity_image}}\pysiglinewithargsret{\sphinxcode{lb\_colloids.LB.LB\_pretty.}\sphinxbfcode{velocity\_image}}{\emph{u}, \emph{img}, \emph{name}, \emph{numit}, \emph{vel}, \emph{vmin}, \emph{vmax}}{}
Lattice Boltmann fluid velocity image generation routine. Can return
plot velocity or fluid magnitude.
\begin{quote}\begin{description}
\item[{Parameters}] \leavevmode\begin{itemize}
\item {} 
\sphinxstyleliteralstrong{u} (\sphinxstyleliteralemphasis{np.ndarray}) \textendash{} Lattice Boltzmann macroscopic velocity array

\item {} 
\sphinxstyleliteralstrong{img} (\sphinxstyleliteralemphasis{np.ndarray}) \textendash{} LBImage binarized image array

\item {} 
\sphinxstyleliteralstrong{name} (\sphinxstyleliteralemphasis{str}) \textendash{} base name to save figure to

\item {} 
\sphinxstyleliteralstrong{numit} (\sphinxstyleliteralemphasis{int}) \textendash{} time step that image is produced at

\item {} 
\sphinxstyleliteralstrong{vel} (\sphinxstyleliteralemphasis{bool}) \textendash{} velocity flag, if false magnitude is plotted

\item {} 
\sphinxstyleliteralstrong{vmin} (\sphinxstyleliteralemphasis{float}) \textendash{} Matplotlib vmin

\item {} 
\sphinxstyleliteralstrong{vmax} (\sphinxstyleliteralemphasis{float}) \textendash{} Matplotlib vmax

\end{itemize}

\end{description}\end{quote}

\end{fulllineitems}



\chapter{Colloid Simulation}
\label{\detokenize{index:id6}}

\section{LB Colloid simulations}
\label{\detokenize{index:lb-colloid-simulations}}
The lattice Boltzmann colloids simulation module, LB\_Colloid, contains the base simulation classes for colloid models. The LB\_Colloid module is general and can be adapted for use with other computational fluid dynamic methods that produce velocity arrays within a discernable porous media. Colloid streaming, updating, and internal data storage are managed within this module. Simulation time and physical time are also tracked within the colloids simulation module.


\subsection{API documentation}
\label{\detokenize{index:id7}}\phantomsection\label{\detokenize{index:module-lb_colloids}}\index{lb\_colloids (module)}\index{ColloidModel (in module lb\_colloids)}

\begin{fulllineitems}
\phantomsection\label{\detokenize{index:lb_colloids.ColloidModel}}\pysigline{\sphinxcode{lb\_colloids.}\sphinxbfcode{ColloidModel}}
alias of {\hyperref[\detokenize{index:module-lb_colloids.Colloids.LB_Colloid}]{\sphinxcrossref{\sphinxcode{lb\_colloids.Colloids.LB\_Colloid}}}}

\end{fulllineitems}

\phantomsection\label{\detokenize{index:module-lb_colloids.Colloids.LB_Colloid}}\index{lb\_colloids.Colloids.LB\_Colloid (module)}
The LB\_Colloid module contains base classes to simulate colloid transport for
colloid simulation. This module acts as the control center. The class Colloid
is the base representation of a colloid and contains
streaming and updating rules. The class TrackTime is the simulation timer, which tracks
both number of time steps and the time step length. Also of importance is the run()
method. This method initiates a colloid simulation from a IO.Config object.

A user can initiate a model run with a Colloid\_IO.Config() object.
Please see the Input Output section for details on building the Colloid\_IO.Config() object

\begin{sphinxVerbatim}[commandchars=\\\{\}]
\PYG{g+gp}{\PYGZgt{}\PYGZgt{}\PYGZgt{} }\PYG{k+kn}{from} \PYG{n+nn}{lb\PYGZus{}colloids} \PYG{k}{import} \PYG{n}{ColloidModel}
\PYG{g+go}{\PYGZgt{}\PYGZgt{}\PYGZgt{}}
\PYG{g+gp}{\PYGZgt{}\PYGZgt{}\PYGZgt{} }\PYG{n}{config} \PYG{o}{=} \PYG{n}{IO}\PYG{o}{.}\PYG{n}{Config}\PYG{p}{(}\PYG{p}{)}  \PYG{c+c1}{\PYGZsh{} We assume that the Colloid\PYGZus{}IO.Config() object is already built. See the Colloid\PYGZus{}IO section for details}
\PYG{g+gp}{\PYGZgt{}\PYGZgt{}\PYGZgt{} }\PYG{n}{ColloidModel}\PYG{o}{.}\PYG{n}{run}\PYG{p}{(}\PYG{n}{config}\PYG{p}{)}
\end{sphinxVerbatim}
\index{Colloid (class in lb\_colloids.Colloids.LB\_Colloid)}

\begin{fulllineitems}
\phantomsection\label{\detokenize{index:lb_colloids.Colloids.LB_Colloid.Colloid}}\pysiglinewithargsret{\sphinxbfcode{class }\sphinxcode{lb\_colloids.Colloids.LB\_Colloid.}\sphinxbfcode{Colloid}}{\emph{xlen}, \emph{ylen}, \emph{resolution}, \emph{tag=0}}{}
Primary colloid class to instantiate and track colloids through a colloid simulation
\begin{quote}\begin{description}
\item[{Parameters}] \leavevmode\begin{itemize}
\item {} 
\sphinxstyleliteralstrong{xlen} (\sphinxstyleliteralemphasis{int}) \textendash{} grid length after interpolation in the x-direction

\item {} 
\sphinxstyleliteralstrong{ylen} (\sphinxstyleliteralemphasis{int}) \textendash{} grid length after interpolation in the y-direction

\item {} 
\sphinxstyleliteralstrong{resolution} (\sphinxstyleliteralemphasis{float}) \textendash{} grid resolution after interpolation

\item {} 
\sphinxstyleliteralstrong{tag} (\sphinxstyleliteralemphasis{int}) \textendash{} colloid number for output writing

\end{itemize}

\end{description}\end{quote}
\paragraph{Methods}
\index{reset\_master\_positions() (lb\_colloids.Colloids.LB\_Colloid.Colloid method)}

\begin{fulllineitems}
\phantomsection\label{\detokenize{index:lb_colloids.Colloids.LB_Colloid.Colloid.reset_master_positions}}\pysiglinewithargsret{\sphinxbfcode{reset\_master\_positions}}{}{}
Resets the master position storage mechanism for all colloids. Master position
storage is used to later generate colloid-colloid DLVO fields.

\end{fulllineitems}

\index{store\_position() (lb\_colloids.Colloids.LB\_Colloid.Colloid method)}

\begin{fulllineitems}
\phantomsection\label{\detokenize{index:lb_colloids.Colloids.LB_Colloid.Colloid.store_position}}\pysiglinewithargsret{\sphinxbfcode{store\_position}}{\emph{timer}}{}
Method to store colloid position and update the the time of storage
\begin{quote}\begin{description}
\item[{Parameters}] \leavevmode
\sphinxstyleliteralstrong{timer} (\sphinxstyleliteralemphasis{float}) \textendash{} current model time

\end{description}\end{quote}

\end{fulllineitems}

\index{strip\_positions() (lb\_colloids.Colloids.LB\_Colloid.Colloid method)}

\begin{fulllineitems}
\phantomsection\label{\detokenize{index:lb_colloids.Colloids.LB_Colloid.Colloid.strip_positions}}\pysiglinewithargsret{\sphinxbfcode{strip\_positions}}{}{}
Memory saving function to strip unused, save colloid position information

\end{fulllineitems}

\index{update\_position() (lb\_colloids.Colloids.LB\_Colloid.Colloid method)}

\begin{fulllineitems}
\phantomsection\label{\detokenize{index:lb_colloids.Colloids.LB_Colloid.Colloid.update_position}}\pysiglinewithargsret{\sphinxbfcode{update\_position}}{\emph{xvelocity}, \emph{yvelocity}, \emph{ts}}{}
grid index method to update continuous colloid system using the discete grid forces.
idxry must be inverted because we assume (0,0) at top left corner and down is negitive.
\begin{quote}\begin{description}
\item[{Parameters}] \leavevmode\begin{itemize}
\item {} 
\sphinxstyleliteralstrong{xvelocity} (\sphinxstyleliteralemphasis{np.ndarray}) \textendash{} colloid simulation velocity array in the x domain

\item {} 
\sphinxstyleliteralstrong{yvelocity} (\sphinxstyleliteralemphasis{np.ndarray}) \textendash{} colloid simulation velocity array in the y domain

\item {} 
\sphinxstyleliteralstrong{float} (\sphinxstyleliteralemphasis{ts}) \textendash{} physical time step in seconds

\end{itemize}

\end{description}\end{quote}

\end{fulllineitems}

\index{update\_special() (lb\_colloids.Colloids.LB\_Colloid.Colloid method)}

\begin{fulllineitems}
\phantomsection\label{\detokenize{index:lb_colloids.Colloids.LB_Colloid.Colloid.update_special}}\pysiglinewithargsret{\sphinxbfcode{update\_special}}{\emph{irx}, \emph{iry}, \emph{flag}}{}
Special updater class for colloids that have exited the model
domain or experienced an internal error
\begin{quote}\begin{description}
\item[{Parameters}] \leavevmode\begin{itemize}
\item {} 
\sphinxstyleliteralstrong{irx} (\sphinxstyleliteralemphasis{int}) \textendash{} grid index in the x domain

\item {} 
\sphinxstyleliteralstrong{iry} (\sphinxstyleliteralemphasis{int}) \textendash{} grid index in the y domain

\item {} 
\sphinxstyleliteralstrong{flag} (\sphinxstyleliteralemphasis{int}) \textendash{} number indicator of special condition. 3 = normal breakthrough condition

\end{itemize}

\end{description}\end{quote}

\end{fulllineitems}


\end{fulllineitems}

\index{Singleton (class in lb\_colloids.Colloids.LB\_Colloid)}

\begin{fulllineitems}
\phantomsection\label{\detokenize{index:lb_colloids.Colloids.LB_Colloid.Singleton}}\pysigline{\sphinxbfcode{class }\sphinxcode{lb\_colloids.Colloids.LB\_Colloid.}\sphinxbfcode{Singleton}}
Singleton object to hold Colloid positions. Potential object to hold
calculated arrays to avoid passing and reduce memory requirements

\end{fulllineitems}

\index{TrackTime (class in lb\_colloids.Colloids.LB\_Colloid)}

\begin{fulllineitems}
\phantomsection\label{\detokenize{index:lb_colloids.Colloids.LB_Colloid.TrackTime}}\pysiglinewithargsret{\sphinxbfcode{class }\sphinxcode{lb\_colloids.Colloids.LB\_Colloid.}\sphinxbfcode{TrackTime}}{\emph{ts}}{}
TrackTime class is the model timer. This class enables stripping stored time steps which
is useful to free memory after writing the data to an external file. Is necessary for
output class functionality!
\begin{quote}\begin{description}
\item[{Parameters}] \leavevmode
\sphinxstyleliteralstrong{ts} (\sphinxstyleliteralemphasis{int}) \textendash{} model time step set by user (physical time)

\end{description}\end{quote}
\paragraph{Methods}
\index{print\_time() (lb\_colloids.Colloids.LB\_Colloid.TrackTime method)}

\begin{fulllineitems}
\phantomsection\label{\detokenize{index:lb_colloids.Colloids.LB_Colloid.TrackTime.print_time}}\pysiglinewithargsret{\sphinxbfcode{print\_time}}{}{}
Method prints time in a standard format to the terminal

\end{fulllineitems}

\index{strip\_time() (lb\_colloids.Colloids.LB\_Colloid.TrackTime method)}

\begin{fulllineitems}
\phantomsection\label{\detokenize{index:lb_colloids.Colloids.LB_Colloid.TrackTime.strip_time}}\pysiglinewithargsret{\sphinxbfcode{strip\_time}}{}{}
Memory saving method that removes unused information from timer arrays

\end{fulllineitems}

\index{update\_time() (lb\_colloids.Colloids.LB\_Colloid.TrackTime method)}

\begin{fulllineitems}
\phantomsection\label{\detokenize{index:lb_colloids.Colloids.LB_Colloid.TrackTime.update_time}}\pysiglinewithargsret{\sphinxbfcode{update\_time}}{}{}
Method to update the current model time and store it

\end{fulllineitems}


\end{fulllineitems}

\index{run() (in module lb\_colloids.Colloids.LB\_Colloid)}

\begin{fulllineitems}
\phantomsection\label{\detokenize{index:lb_colloids.Colloids.LB_Colloid.run}}\pysiglinewithargsret{\sphinxcode{lb\_colloids.Colloids.LB\_Colloid.}\sphinxbfcode{run}}{\emph{config}}{}
Model definition to setup and run the LB\_Colloids from config file.
This is the also the preferred user interaction method to run simulations
when working with python.
\begin{quote}\begin{description}
\item[{Parameters}] \leavevmode
\sphinxstyleliteralstrong{config} (\sphinxstyleliteralemphasis{colloid\_IO.config}) \textendash{} object or list of colloid\_IO.Config objects

\end{description}\end{quote}

\end{fulllineitems}



\section{LB Colloid mathematics}
\label{\detokenize{index:lb-colloid-mathematics}}
The colloid mathematics module is responsible for calculating forces that define colloid-surface and colloid-colloid interactions. Objects exist within this module for each physical and physio-chemical interaction force that is simulated within a colloid particle tracking simulation. Forces are converted to a ‘velocity-like’ representation to account for the change in position due to applied forces. Calculations are performed for Brownian motion, drag forces, gravity, bouyancy, electrostatic repulsion, and attrative chemical interaction energies. A detailed review of the mathematical theory involved will not be described in this document. The base equations for each interaction are listed within the user documentation. The python version of colloid interactions will be migrated to FORTAN for computational efficiency. User functionality and object calls are not expected to change after migration.


\subsection{API documentation}
\label{\detokenize{index:id8}}\phantomsection\label{\detokenize{index:module-lb_colloids}}\index{lb\_colloids (module)}\index{ColloidMath (in module lb\_colloids)}

\begin{fulllineitems}
\phantomsection\label{\detokenize{index:lb_colloids.ColloidMath}}\pysigline{\sphinxcode{lb\_colloids.}\sphinxbfcode{ColloidMath}}
alias of {\hyperref[\detokenize{index:module-lb_colloids.Colloids.Colloid_Math}]{\sphinxcrossref{\sphinxcode{lb\_colloids.Colloids.Colloid\_Math}}}}

\end{fulllineitems}

\phantomsection\label{\detokenize{index:module-lb_colloids.Colloids.Colloid_Math}}\index{lb\_colloids.Colloids.Colloid\_Math (module)}
ColloidMath is the primary mathematics module for Colloid Simulations.
This module contains both Physical and Chemical formulations of colloid forces within a
porous media. The DLVO and ColloidColloid classes contain complex formulations
of chemical interaction forces. Other classes contain physical force calculations or
provide mathematical conversion from Force to a Velocity like unit that can be used
to recover the change in colloid position. Users should not have to call these classes
directly when running a model.

Basic examples of how these modules are called assume that a user has already provided input to
the lbIO.Config() module and the appropriate dictionaries have been built. For more information
on required parameters and keywords please inspect API documentation for each respective class.
Docstrings also provide basic mathematical relationships for each class.

\begin{sphinxVerbatim}[commandchars=\\\{\}]
\PYG{g+gp}{\PYGZgt{}\PYGZgt{}\PYGZgt{} }\PYG{k+kn}{from} \PYG{n+nn}{lb\PYGZus{}colloids} \PYG{k}{import} \PYG{n}{ColloidMath} \PYG{k}{as} \PYG{n}{cm}
\PYG{g+go}{\PYGZgt{}\PYGZgt{}\PYGZgt{}}
\PYG{g+gp}{\PYGZgt{}\PYGZgt{}\PYGZgt{} }\PYG{n}{grav} \PYG{o}{=} \PYG{n}{cm}\PYG{o}{.}\PYG{n}{Gravity}\PYG{p}{(}\PYG{o}{*}\PYG{o}{*}\PYG{n}{PhysicalDict}\PYG{p}{)}
\PYG{g+gp}{\PYGZgt{}\PYGZgt{}\PYGZgt{} }\PYG{n}{grav}\PYG{o}{.}\PYG{n}{gravity}  \PYG{c+c1}{\PYGZsh{} returns the gravity force on a colloid}
\PYG{g+gp}{\PYGZgt{}\PYGZgt{}\PYGZgt{} }\PYG{n}{bouy} \PYG{o}{=} \PYG{n}{cm}\PYG{o}{.}\PYG{n}{Bouyancy}\PYG{p}{(}\PYG{o}{*}\PYG{o}{*}\PYG{n}{PhysicalDict}\PYG{p}{)}
\PYG{g+gp}{\PYGZgt{}\PYGZgt{}\PYGZgt{} }\PYG{n}{bouy}\PYG{o}{.}\PYG{n}{bouyancy}  \PYG{c+c1}{\PYGZsh{} returns the bouyancy force on a colloid}
\PYG{g+gp}{\PYGZgt{}\PYGZgt{}\PYGZgt{} }\PYG{n}{gap} \PYG{o}{=} \PYG{n}{cm}\PYG{o}{.}\PYG{n}{Gap}\PYG{p}{(}\PYG{n}{xarr}\PYG{p}{,} \PYG{n}{yarr}\PYG{p}{,} \PYG{o}{*}\PYG{o}{*}\PYG{n}{PhysicalDict}\PYG{p}{)}
\PYG{g+gp}{\PYGZgt{}\PYGZgt{}\PYGZgt{} }\PYG{n}{brownian} \PYG{o}{=} \PYG{n}{cm}\PYG{o}{.}\PYG{n}{Brownian}\PYG{p}{(}\PYG{n}{gap}\PYG{o}{.}\PYG{n}{f1}\PYG{p}{,} \PYG{n}{gap}\PYG{o}{.}\PYG{n}{f2}\PYG{p}{,} \PYG{o}{*}\PYG{o}{*}\PYG{n}{PhysicalDict}\PYG{p}{)}
\PYG{g+gp}{\PYGZgt{}\PYGZgt{}\PYGZgt{} }\PYG{n}{brownian}\PYG{o}{.}\PYG{n}{brownian\PYGZus{}x}  \PYG{c+c1}{\PYGZsh{} returns brownian force in the x\PYGZhy{}direction on a colloid}
\PYG{g+gp}{\PYGZgt{}\PYGZgt{}\PYGZgt{} }\PYG{n}{brownian}\PYG{o}{.}\PYG{n}{brownian\PYGZus{}y}
\PYG{g+gp}{\PYGZgt{}\PYGZgt{}\PYGZgt{} }\PYG{n}{drag} \PYG{o}{=} \PYG{n}{cm}\PYG{o}{.}\PYG{n}{Drag}\PYG{p}{(}\PYG{n}{ux}\PYG{p}{,} \PYG{n}{uy}\PYG{p}{,} \PYG{n}{gap}\PYG{o}{.}\PYG{n}{f1}\PYG{p}{,} \PYG{n}{gap}\PYG{o}{.}\PYG{n}{f2}\PYG{p}{,} \PYG{n}{gap}\PYG{o}{.}\PYG{n}{f3}\PYG{p}{,} \PYG{n}{gap}\PYG{o}{.}\PYG{n}{f4}\PYG{p}{,} \PYG{o}{*}\PYG{o}{*}\PYG{n}{PhysicalDict}\PYG{p}{)}
\PYG{g+gp}{\PYGZgt{}\PYGZgt{}\PYGZgt{} }\PYG{n}{drag}\PYG{o}{.}\PYG{n}{drag\PYGZus{}x}  \PYG{c+c1}{\PYGZsh{} returns an array of drag forces in the x\PYGZhy{}direction}
\PYG{g+gp}{\PYGZgt{}\PYGZgt{}\PYGZgt{} }\PYG{n}{dlvo} \PYG{o}{=} \PYG{n}{cm}\PYG{o}{.}\PYG{n}{DLVO}\PYG{p}{(}\PYG{n}{xarr}\PYG{p}{,} \PYG{n}{yarr}\PYG{p}{,} \PYG{o}{*}\PYG{o}{*}\PYG{n}{ChemicalDict}\PYG{p}{)}
\PYG{g+gp}{\PYGZgt{}\PYGZgt{}\PYGZgt{} }\PYG{n}{dlvo}\PYG{o}{.}\PYG{n}{EDLx}  \PYG{c+c1}{\PYGZsh{} returns an array of electric double layer forces in the x\PYGZhy{}direction}
\PYG{g+gp}{\PYGZgt{}\PYGZgt{}\PYGZgt{} }\PYG{n}{dlvo}\PYG{o}{.}\PYG{n}{LewisABy}  \PYG{c+c1}{\PYGZsh{} returns an array of lewis acid base forces in the y\PYGZhy{}direction}
\PYG{g+gp}{\PYGZgt{}\PYGZgt{}\PYGZgt{} }\PYG{n}{dlvo}\PYG{o}{.}\PYG{n}{LVDWx}  \PYG{c+c1}{\PYGZsh{} returns an array of lifshitz\PYGZhy{}van der waals forces in the x\PYGZhy{}direction}
\PYG{g+gp}{\PYGZgt{}\PYGZgt{}\PYGZgt{} }\PYG{n}{colcol} \PYG{o}{=} \PYG{n}{cm}\PYG{o}{.}\PYG{n}{ColloidColloid}\PYG{p}{(}\PYG{n}{xarr}\PYG{p}{,} \PYG{o}{*}\PYG{o}{*}\PYG{n}{ChemicalDict}\PYG{p}{)}
\PYG{g+gp}{\PYGZgt{}\PYGZgt{}\PYGZgt{} }\PYG{n}{colcol}\PYG{o}{.}\PYG{n}{x\PYGZus{}array}  \PYG{c+c1}{\PYGZsh{} returns an array of dlvo forces for colloid\PYGZhy{}colloid interactions}
\PYG{g+gp}{\PYGZgt{}\PYGZgt{}\PYGZgt{} }\PYG{n}{colcol}\PYG{o}{.}\PYG{n}{update}\PYG{p}{(}\PYG{n}{Singleton}\PYG{o}{.}\PYG{n}{positions}\PYG{p}{)}  \PYG{c+c1}{\PYGZsh{} updates the class to generate new colloid\PYGZhy{}colloid interaction arrays}
\end{sphinxVerbatim}
\index{Bouyancy (class in lb\_colloids.Colloids.Colloid\_Math)}

\begin{fulllineitems}
\phantomsection\label{\detokenize{index:lb_colloids.Colloids.Colloid_Math.Bouyancy}}\pysiglinewithargsret{\sphinxbfcode{class }\sphinxcode{lb\_colloids.Colloids.Colloid\_Math.}\sphinxbfcode{Bouyancy}}{\emph{**kwargs}}{}
Class to estimate the gravitational force experienced by a colloid. Gravity
is applied as a positive value to maintain vector direction.
\begin{equation*}
\begin{split}F^{b} = \frac{4 \pi a_{c}^{3} \rho_{w} g}{3}\end{split}
\end{equation*}\begin{quote}\begin{description}
\item[{Keyword Arguments}] \leavevmode\begin{itemize}
\item {} 
\sphinxstyleliteralstrong{rho\_water} (\sphinxstyleliteralemphasis{flaot}) \textendash{} density of water \(kg/m^3\). Default is 997.

\item {} 
\sphinxstyleliteralstrong{rho\_colloid} (\sphinxstyleliteralemphasis{float}) \textendash{} particle density of a colloid in \(kg/m^3\). Default is 2650.

\item {} 
\sphinxstyleliteralstrong{ac} (\sphinxstyleliteralemphasis{float}) \textendash{} colloid radius in m. Default is 1e-6.

\end{itemize}

\end{description}\end{quote}
\begin{quote}\begin{description}
\item[{Returns}] \leavevmode
bouyancy (float) Bouyancy force that a colloid experiences

\end{description}\end{quote}

\end{fulllineitems}

\index{Brownian (class in lb\_colloids.Colloids.Colloid\_Math)}

\begin{fulllineitems}
\phantomsection\label{\detokenize{index:lb_colloids.Colloids.Colloid_Math.Brownian}}\pysiglinewithargsret{\sphinxbfcode{class }\sphinxcode{lb\_colloids.Colloids.Colloid\_Math.}\sphinxbfcode{Brownian}}{\emph{f1}, \emph{f4}, \emph{**kwargs}}{}
Class to estimate brownian forces on colloids. Uses the relationships outlined in Qui et. al. 2010
where
\begin{align*}\!\begin{aligned}
F_{x}^{B} = \xi \sqrt{\frac{2D_{0}}{f_{1}dt}}G(0,1)\\
F_{y}^{B} = \xi \sqrt{\frac{2D_{0}}{f_{4}dt}}G(0,1)\\
\end{aligned}\end{align*}\begin{quote}\begin{description}
\item[{Parameters}] \leavevmode\begin{itemize}
\item {} 
\sphinxstyleliteralstrong{f1} (\sphinxstyleliteralemphasis{np.ndarray}) \textendash{} Drag force correction term {[}Gao et. al. 2010. Computers and Math with App{]}

\item {} 
\sphinxstyleliteralstrong{f4} (\sphinxstyleliteralemphasis{np.ndarray}) \textendash{} Drag force correction term {[}Gao et. al. 2010{]}

\end{itemize}

\item[{Keyword Arguments}] \leavevmode\begin{itemize}
\item {} 
\sphinxstyleliteralstrong{ac} (\sphinxstyleliteralemphasis{float}) \textendash{} Colloid radius. Default 1e-6

\item {} 
\sphinxstyleliteralstrong{viscosity} (\sphinxstyleliteralemphasis{float}) \textendash{} Dynamic viscosity of water. Default 8.9e-4 Pa S.

\item {} 
\sphinxstyleliteralstrong{T} (\sphinxstyleliteralemphasis{float}) \textendash{} Absolute temperature in K. Default is 298.15

\end{itemize}

\end{description}\end{quote}
\begin{quote}\begin{description}
\item[{Returns}] \leavevmode
brownian\_x: (np.ndarray) array of browian (random)
forces in the x direction {[}Qiu et. al 2011.{]}

\item[{Returns}] \leavevmode
brownian\_y: (np.ndarray) array of browian (random)
forces in the y direction {[}Qiu et. al 2011.{]}

\end{description}\end{quote}
\paragraph{Attributes}

\end{fulllineitems}

\index{ColloidColloid (class in lb\_colloids.Colloids.Colloid\_Math)}

\begin{fulllineitems}
\phantomsection\label{\detokenize{index:lb_colloids.Colloids.Colloid_Math.ColloidColloid}}\pysiglinewithargsret{\sphinxbfcode{class }\sphinxcode{lb\_colloids.Colloids.Colloid\_Math.}\sphinxbfcode{ColloidColloid}}{\emph{arr}, \emph{**kwargs}}{}
The ColloidColloid class is used to calculate colloid-colloid interaction forces
using the formulations presented in Liang 2008, Qui 2012, and Israelichevi 1996.
Attractive forces are based on the Liang \& Israelichevi formulation. Electric
doulbe layer forces are calculated using Qui et. al. 2012.

The ColloidColloid object also provides methods to update ColloidColloid force
array fields during model streaming.

Colloid colloid interaction energies are calculated via:
\begin{equation*}
\begin{split}\Phi^{EDL} = 32 \pi \epsilon_{0} \epsilon_{r} a_{c}
(\frac{kT}{Ze})^{2} * [tanh(\frac{Ze\psi_c}{4kT})]^{2}
* exp(-\kappa h)\end{split}
\end{equation*}\begin{equation*}
\begin{split}A_{H} = 384 \pi \frac{\psi_{c}^{2} h k T I^{*}}{\kappa^{2}} exp(- \kappa h)\end{split}
\end{equation*}\begin{equation*}
\begin{split}\Phi^{A} = - \frac{A_{H}}{6}[\frac{2a_{c}^{2}}{h^{2} + 4a_{c}h} +
\frac{2a_{c}^{2}}{(h + 2a_{c})^{2}} + ln(1 - \frac{4a_{c}^{2}}{(h + 2a_{c})^{2}})]\end{split}
\end{equation*}\begin{quote}\begin{description}
\item[{Parameters}] \leavevmode\begin{itemize}
\item {} 
\sphinxstyleliteralstrong{arr} (\sphinxstyleliteralemphasis{np.ndarray}) \textendash{} A np.ndarray that represents the shape of the colloid
domain

\item {} 
\sphinxstyleliteralstrong{resolution} (\sphinxstyleliteralemphasis{float}) \textendash{} Colloid model resolution

\end{itemize}

\item[{Keyword Arguments}] \leavevmode\begin{itemize}
\item {} 
\sphinxstyleliteralstrong{valence} (\sphinxstyleliteralemphasis{dict}) \textendash{} Valences of all species in solution. (Optional)

\item {} 
\sphinxstyleliteralstrong{concentration} (\sphinxstyleliteralemphasis{dict}) \textendash{} Concentration of all species in solution (Optional)

\item {} 
\sphinxstyleliteralstrong{zeta\_colloid} (\sphinxstyleliteralemphasis{float}) \textendash{} Measured\_zeta potential of colloid (Reccomended).
Default -40.5e-3 Na-Kaolinite Colloid {[}Chorom 1995. Eur. Jour. of Soil Science{]}

\item {} 
\sphinxstyleliteralstrong{zeta\_surface} (\sphinxstyleliteralemphasis{float}) \textendash{} Bulk\_zeta potential of porous media (Reccomended).
Default -60.9e-3 Glass bead media {[}Ducker 1992, Langmuir V8{]}

\item {} 
\sphinxstyleliteralstrong{I} (\sphinxstyleliteralemphasis{float}) \textendash{} Ionic strength of simulated solution (Reccomended). Default 1e-3 M

\item {} 
\sphinxstyleliteralstrong{ac} (\sphinxstyleliteralemphasis{float}) \textendash{} Colloid radius in meters. Default 1e-6 m.

\item {} 
\sphinxstyleliteralstrong{epsilon\_r} (\sphinxstyleliteralemphasis{float}) \textendash{} Relative dielectric permativity of water. (Optional)
Default 78.304 @ 298 K {[}Malmberg and Maryott 1956. Jour. Res. Nat. Beau. Std. V56(1)

\item {} 
\sphinxstyleliteralstrong{sheer\_plane} (\sphinxstyleliteralemphasis{float}) \textendash{} Equivelent to the thickness of one layer of water molecules. (Optional)
Default 3e-10 m {[}Interface Science and Technology, 2008. Volume 16 Chapter 3{]}

\item {} 
\sphinxstyleliteralstrong{T} (\sphinxstyleliteralemphasis{float}) \textendash{} Temperature of simulation fluid. Default 298.15 k

\end{itemize}

\end{description}\end{quote}
\paragraph{Attributes}
\paragraph{Methods}
\index{colloid\_potential (lb\_colloids.Colloids.Colloid\_Math.ColloidColloid attribute)}

\begin{fulllineitems}
\phantomsection\label{\detokenize{index:lb_colloids.Colloids.Colloid_Math.ColloidColloid.colloid_potential}}\pysigline{\sphinxbfcode{colloid\_potential}}
Property method that generates colloid potential

\end{fulllineitems}

\index{debye (lb\_colloids.Colloids.Colloid\_Math.ColloidColloid attribute)}

\begin{fulllineitems}
\phantomsection\label{\detokenize{index:lb_colloids.Colloids.Colloid_Math.ColloidColloid.debye}}\pysigline{\sphinxbfcode{debye}}
Property method to calculate the debye length on the fly

\end{fulllineitems}

\index{ionic\_strength (lb\_colloids.Colloids.Colloid\_Math.ColloidColloid attribute)}

\begin{fulllineitems}
\phantomsection\label{\detokenize{index:lb_colloids.Colloids.Colloid_Math.ColloidColloid.ionic_strength}}\pysigline{\sphinxbfcode{ionic\_strength}}
Property method to calculate ionic\_strength on the fly

\end{fulllineitems}

\index{positions (lb\_colloids.Colloids.Colloid\_Math.ColloidColloid attribute)}

\begin{fulllineitems}
\phantomsection\label{\detokenize{index:lb_colloids.Colloids.Colloid_Math.ColloidColloid.positions}}\pysigline{\sphinxbfcode{positions}}
Property method to generate colloid positions if they are not stored yet

\end{fulllineitems}

\index{update() (lb\_colloids.Colloids.Colloid\_Math.ColloidColloid method)}

\begin{fulllineitems}
\phantomsection\label{\detokenize{index:lb_colloids.Colloids.Colloid_Math.ColloidColloid.update}}\pysiglinewithargsret{\sphinxbfcode{update}}{\emph{colloids}}{}
Updates the colloidal positions and force arrays for the system
\begin{quote}\begin{description}
\item[{Parameters}] \leavevmode
\sphinxstyleliteralstrong{colloids} (\sphinxstyleliteralemphasis{list}) \textendash{} (list, \textless{}class: Colloids.LB\_Colloid.Colloid)

\end{description}\end{quote}

\end{fulllineitems}

\index{x (lb\_colloids.Colloids.Colloid\_Math.ColloidColloid attribute)}

\begin{fulllineitems}
\phantomsection\label{\detokenize{index:lb_colloids.Colloids.Colloid_Math.ColloidColloid.x}}\pysigline{\sphinxbfcode{x}}
Property method to generate the x force array for colloid-colloid interaction

\end{fulllineitems}

\index{x\_array (lb\_colloids.Colloids.Colloid\_Math.ColloidColloid attribute)}

\begin{fulllineitems}
\phantomsection\label{\detokenize{index:lb_colloids.Colloids.Colloid_Math.ColloidColloid.x_array}}\pysigline{\sphinxbfcode{x\_array}}
Property method to generate the full x force array for colloid-colloid interaction

\end{fulllineitems}

\index{x\_distance\_array (lb\_colloids.Colloids.Colloid\_Math.ColloidColloid attribute)}

\begin{fulllineitems}
\phantomsection\label{\detokenize{index:lb_colloids.Colloids.Colloid_Math.ColloidColloid.x_distance_array}}\pysigline{\sphinxbfcode{x\_distance\_array}}
Generates an angular distance array in the x direction.

\end{fulllineitems}

\index{y (lb\_colloids.Colloids.Colloid\_Math.ColloidColloid attribute)}

\begin{fulllineitems}
\phantomsection\label{\detokenize{index:lb_colloids.Colloids.Colloid_Math.ColloidColloid.y}}\pysigline{\sphinxbfcode{y}}
Property method to generate or return the y force array for colloid-colloid interaction

\end{fulllineitems}

\index{y\_array (lb\_colloids.Colloids.Colloid\_Math.ColloidColloid attribute)}

\begin{fulllineitems}
\phantomsection\label{\detokenize{index:lb_colloids.Colloids.Colloid_Math.ColloidColloid.y_array}}\pysigline{\sphinxbfcode{y\_array}}
Property method to generate the full y force array for colloid-colloid interaction

\end{fulllineitems}

\index{y\_distance\_array (lb\_colloids.Colloids.Colloid\_Math.ColloidColloid attribute)}

\begin{fulllineitems}
\phantomsection\label{\detokenize{index:lb_colloids.Colloids.Colloid_Math.ColloidColloid.y_distance_array}}\pysigline{\sphinxbfcode{y\_distance\_array}}
Generates an angular distance array in the y direction

\end{fulllineitems}


\end{fulllineitems}

\index{DLVO (class in lb\_colloids.Colloids.Colloid\_Math)}

\begin{fulllineitems}
\phantomsection\label{\detokenize{index:lb_colloids.Colloids.Colloid_Math.DLVO}}\pysiglinewithargsret{\sphinxbfcode{class }\sphinxcode{lb\_colloids.Colloids.Colloid\_Math.}\sphinxbfcode{DLVO}}{\emph{xarr}, \emph{yarr}, \emph{**kwargs}}{}
Class method to calculate vectorized DLVO force arrays for colloid surface interaction using
methods outlined in Qui et. al. 2011 and Liang et. al. 2008? \sphinxstyleemphasis{Check this later}

Parameterization of this class is handled primary through the ChemistryDict by {\color{red}\bfseries{}**}kwargs

Mathematics used in calcuation of DLVO interaction energies are:
\begin{equation*}
\begin{split}\frac{1}{\kappa} = (\frac{\epsilon_{r} \epsilon_{0} k T}{e^{2} N_{A} I^{*}})^{\frac{1}{2}}\end{split}
\end{equation*}\begin{equation*}
\begin{split}\Phi^{EDL} = \pi \epsilon_{0} \epsilon_{r} a_{c}
(2 \psi_{s} \psi_{c}
ln(\frac{1 + exp(-\kappa h)}{1 - exp(-\kappa h)})
+ (\psi_{s}^{2} + \psi_{c}^{2})
ln(1 - exp(-2 \kappa h)))\end{split}
\end{equation*}\begin{quote}\begin{description}
\item[{Parameters}] \leavevmode\begin{itemize}
\item {} 
\sphinxstyleliteralstrong{xarr} (\sphinxstyleliteralemphasis{np.ndarray}) \textendash{} Physical distance from solid boundaries in the x direction

\item {} 
\sphinxstyleliteralstrong{yarr} (\sphinxstyleliteralemphasis{np.ndarray}) \textendash{} Physical distance from solid boundaries in the y direction

\end{itemize}

\item[{Keyword Arguments}] \leavevmode\begin{itemize}
\item {} 
\sphinxstyleliteralstrong{valence} (\sphinxstyleliteralemphasis{dict}) \textendash{} Valences of all species in solution. (Optional)

\item {} 
\sphinxstyleliteralstrong{concentration} (\sphinxstyleliteralemphasis{dict}) \textendash{} Concentration of all species in solution (Optional)

\item {} 
\sphinxstyleliteralstrong{zeta\_colloid} (\sphinxstyleliteralemphasis{float}) \textendash{} Measured\_zeta potential of colloid (Reccomended).
Default -40.5e-3 Na-Kaolinite Colloid {[}Chorom 1995. Eur. Jour. of Soil Science{]}

\item {} 
\sphinxstyleliteralstrong{zeta\_surface} (\sphinxstyleliteralemphasis{float}) \textendash{} Bulk\_zeta potential of porous media (Reccomended).
Default -60.9e-3 Glass bead media {[}Ducker 1992, Langmuir V8{]}

\item {} 
\sphinxstyleliteralstrong{I} (\sphinxstyleliteralemphasis{float}) \textendash{} Ionic strength of simulated solution (Reccomended). Default 1e-3 M

\item {} 
\sphinxstyleliteralstrong{ac} (\sphinxstyleliteralemphasis{float}) \textendash{} Colloid radius in meters. Default 1e-6 m.

\item {} 
\sphinxstyleliteralstrong{epsilon\_r} (\sphinxstyleliteralemphasis{float}) \textendash{} Relative dielectric permativity of water. (Optional)
Default 78.304 @ 298 K {[}Malmberg and Maryott 1956. Jour. Res. Nat. Beau. Std. V56(1)

\item {} 
\sphinxstyleliteralstrong{sheer\_plane} (\sphinxstyleliteralemphasis{float}) \textendash{} Equivelent to the thickness of one layer of water molecules. (Optional)
Default 3e-10 m {[}Interface Science and Technology, 2008. Volume 16 Chapter 3{]}

\item {} 
\sphinxstyleliteralstrong{T} (\sphinxstyleliteralemphasis{float}) \textendash{} Temperature of simulation fluid. Default 298.15 k

\item {} 
\sphinxstyleliteralstrong{lvdwst\_colloid} (\sphinxstyleliteralemphasis{float}) \textendash{} Lifshits-van der Waals surface tension component from colloid. (Reccomended)
Default is 39.9e-3 J/m**2 {[}Giese et. al. 1996, Jour. Disp. Sci. \& Tech. 17(5){]}

\item {} 
\sphinxstyleliteralstrong{lvdwst\_solid} (\sphinxstyleliteralemphasis{float}) \textendash{} Lifshits-van der Waals surface tension component from solid. (Reccomended)
Default is 33.7e-3 J/m**2 {[}Giese et. al. 1996{]}

\item {} 
\sphinxstyleliteralstrong{lvdwst\_water} (\sphinxstyleliteralemphasis{float}) \textendash{} Lifshits-van der Waals surface tension component from water. (Reccomended)
Default is 21.8e-3 J/m**2 {[}Interface Science and Technology, 2008. V16(2){]}

\item {} 
\sphinxstyleliteralstrong{psi+\_colloid} (\sphinxstyleliteralemphasis{float}) \textendash{} Lewis acid base electron acceptor parameter. (Reccomended)
Default is 0.4e-3 J/m**2 {[}Giese et. al. 1996{]}

\item {} 
\sphinxstyleliteralstrong{psi+\_solid} (\sphinxstyleliteralemphasis{float}) \textendash{} Lewis acid base electron acceptor parameter. (Reccomended)
Default is 1.3e-3 J/m**2 {[}Giese et. al. 1996{]}

\item {} 
\sphinxstyleliteralstrong{psi+\_water} (\sphinxstyleliteralemphasis{float}) \textendash{} Lewis acid base electron acceptor parameter. (Reccomended)
Default is 25.5e-3 J/m**2 {[}Interface Science and Technology, 2008. V16(2){]}

\item {} 
\sphinxstyleliteralstrong{psi-\_colloid} (\sphinxstyleliteralemphasis{float}) \textendash{} Lewis acid base electron donor parameter. (Reccomended)
Default is 34.3e-3 J/m**2 {[}Giese et. al. 1996{]}

\item {} 
\sphinxstyleliteralstrong{psi-\_solid} (\sphinxstyleliteralemphasis{float}) \textendash{} Lewis acid base electron donor parameter. (Reccomended)
Default is 62.2e-3 J/m**2 {[}Giese et. al. 1996{]}

\item {} 
\sphinxstyleliteralstrong{psi-\_water} (\sphinxstyleliteralemphasis{float}) \textendash{} Lewis acid base electron donor parameter. (Reccomended)
Default is 25.5e-3 J/m**2 {[}Interface Science and Technology, 2008. V16(2){]}

\item {} 
\sphinxstyleliteralstrong{xvArr} (\sphinxstyleliteralemphasis{np.ndarray}) \textendash{} Array of vector directions.This array is applied to properly represent attractive
and repulsive forces

\item {} 
\sphinxstyleliteralstrong{yvArr} (\sphinxstyleliteralemphasis{np.ndarray}) \textendash{} Array of vector directions.This array is applied to properly represent attractive
and repulsive forces

\end{itemize}

\end{description}\end{quote}
\begin{quote}\begin{description}
\item[{Returns}] \leavevmode
EDLx (np.ndarray) vectorized np.array of electric-double-layer force values in the x-direction

\item[{Returns}] \leavevmode
EDLy (np.ndarray) vectorized np.array of electric-double-layer force values in the y-direction

\item[{Returns}] \leavevmode
LVDWx (np.ndarray) vectorized np.array of lifshitz-van-der-walls force values in the x-direction

\item[{Returns}] \leavevmode
LVDWy (np.ndarray) vectorized np.array of lifshitz-van-der-walls force values in the y-direction

\item[{Returns}] \leavevmode
LewisABx (np.ndarray) vectorized np.array of lewis acid base force values in the x-direction

\item[{Returns}] \leavevmode
LewisABy (np.ndarray) vectorized np.array of lewis acid base force values in the y-direction

\end{description}\end{quote}
\paragraph{Attributes}
\paragraph{Methods}
\index{attractive\_x (lb\_colloids.Colloids.Colloid\_Math.DLVO attribute)}

\begin{fulllineitems}
\phantomsection\label{\detokenize{index:lb_colloids.Colloids.Colloid_Math.DLVO.attractive_x}}\pysigline{\sphinxbfcode{attractive\_x}}
Calculates the combined attractive force between colloid surface
based upon Liang et. al. 2008
\begin{quote}\begin{description}
\item[{Returns}] \leavevmode
np.ndarray

\end{description}\end{quote}

\end{fulllineitems}

\index{attractive\_y (lb\_colloids.Colloids.Colloid\_Math.DLVO attribute)}

\begin{fulllineitems}
\phantomsection\label{\detokenize{index:lb_colloids.Colloids.Colloid_Math.DLVO.attractive_y}}\pysigline{\sphinxbfcode{attractive\_y}}
Calculates the combined attractive force between colloid surface
based upon Liang et. al. 2008
\begin{quote}\begin{description}
\item[{Returns}] \leavevmode
np.ndarray

\end{description}\end{quote}

\end{fulllineitems}

\index{ionic() (lb\_colloids.Colloids.Colloid\_Math.DLVO method)}

\begin{fulllineitems}
\phantomsection\label{\detokenize{index:lb_colloids.Colloids.Colloid_Math.DLVO.ionic}}\pysiglinewithargsret{\sphinxbfcode{ionic}}{\emph{valence}, \emph{concentration}}{}
Calculates the 2*I from user supplied valence and concentraitons
\begin{equation*}
\begin{split}I^{*} = \sum_{i} Z_{i}^{2} M_{i}\end{split}
\end{equation*}\begin{quote}\begin{description}
\item[{Parameters}] \leavevmode\begin{itemize}
\item {} 
\sphinxstyleliteralstrong{valence} (\sphinxstyleliteralemphasis{dict}) \textendash{} Dictionary of chemical species, valence

\item {} 
\sphinxstyleliteralstrong{concentration} (\sphinxstyleliteralemphasis{dict}) \textendash{} Dictionary of chemical species, concentration

\end{itemize}

\end{description}\end{quote}
\begin{quote}\begin{description}
\item[{Returns}] \leavevmode
I (float) 2*ionic stength

\end{description}\end{quote}

\end{fulllineitems}

\index{k\_debye (lb\_colloids.Colloids.Colloid\_Math.DLVO attribute)}

\begin{fulllineitems}
\phantomsection\label{\detokenize{index:lb_colloids.Colloids.Colloid_Math.DLVO.k_debye}}\pysigline{\sphinxbfcode{k\_debye}}
Method to calculate Debye length
\begin{quote}\begin{description}
\item[{Returns}] \leavevmode
Debye length (float)

\end{description}\end{quote}

\end{fulllineitems}


\end{fulllineitems}

\index{Drag (class in lb\_colloids.Colloids.Colloid\_Math)}

\begin{fulllineitems}
\phantomsection\label{\detokenize{index:lb_colloids.Colloids.Colloid_Math.Drag}}\pysiglinewithargsret{\sphinxbfcode{class }\sphinxcode{lb\_colloids.Colloids.Colloid\_Math.}\sphinxbfcode{Drag}}{\emph{ux}, \emph{uy}, \emph{f1}, \emph{f2}, \emph{f3}, \emph{f4}, \emph{**kwargs}}{}
Class to calculate colloidal drag forces from fluid velocity arrays. Based from calculations
outlined in Gao et, al 2010 and Qui et. al. 2011.
\begin{align*}\!\begin{aligned}
F_{x}^{D} = \frac{\xi}{f_{4}} (f_{3}u_{x} - V_{x})\\
F_{y}^{D} = \xi (f_{2} u_{y} - \frac{V_{y}}{f_{1}})\\
\end{aligned}\end{align*}\begin{quote}\begin{description}
\item[{Parameters}] \leavevmode\begin{itemize}
\item {} 
\sphinxstyleliteralstrong{ux} (\sphinxstyleliteralemphasis{np.ndarray}) \textendash{} fluid velocity in the x-direction

\item {} 
\sphinxstyleliteralstrong{uy} (\sphinxstyleliteralemphasis{np.ndarray}) \textendash{} fluid velocity in the y-direction

\item {} 
\sphinxstyleliteralstrong{Vx} (\sphinxstyleliteralemphasis{np.ndarray}) \textendash{} colloid velocity in the x-direction

\item {} 
\sphinxstyleliteralstrong{Vy} (\sphinxstyleliteralemphasis{np.ndarray}) \textendash{} colloid velocity in the y-direction

\item {} 
\sphinxstyleliteralstrong{f1} (\sphinxstyleliteralemphasis{np.ndarray}) \textendash{} Hydrodynamic force correction term {[}Gao et. al. 2010.{]}

\item {} 
\sphinxstyleliteralstrong{f2} (\sphinxstyleliteralemphasis{np.ndarray}) \textendash{} Hydrodynamic force correction term {[}Gao et. al. 2010.{]}

\item {} 
\sphinxstyleliteralstrong{f3} (\sphinxstyleliteralemphasis{np.ndarray}) \textendash{} Hydrodynamic force correction term {[}Gao et. al. 2010.{]}

\item {} 
\sphinxstyleliteralstrong{f4} (\sphinxstyleliteralemphasis{np.ndarray}) \textendash{} Hydrodynamic force correction term {[}Gao et. al. 2010.{]}

\end{itemize}

\item[{Keyword Arguments}] \leavevmode\begin{itemize}
\item {} 
\sphinxstyleliteralstrong{ac} (\sphinxstyleliteralemphasis{float}) \textendash{} Colloid radius. Default is 1e-6 m

\item {} 
\sphinxstyleliteralstrong{viscosity} (\sphinxstyleliteralemphasis{float}) \textendash{} Dynamic fluid viscosity of water. Default 8.9e-4 Pa S

\item {} 
\sphinxstyleliteralstrong{rho\_colloid} (\sphinxstyleliteralemphasis{float}) \textendash{} Colloid particle density. Default \(2650 kg/m^3\)

\item {} 
\sphinxstyleliteralstrong{rho\_water} (\sphinxstyleliteralemphasis{float}) \textendash{} Water density. Default \(997 kg/m^3\)

\end{itemize}

\end{description}\end{quote}
\begin{quote}\begin{description}
\item[{Returns}] \leavevmode
drag\_x (np.ndarray) non-vectorized drag forces in the x-direction

\item[{Returns}] \leavevmode
drag\_y: (np.ndarray) non-vectorized drag forces in the y-direction

\end{description}\end{quote}
\paragraph{Attributes}
\paragraph{Methods}
\index{drag\_x (lb\_colloids.Colloids.Colloid\_Math.Drag attribute)}

\begin{fulllineitems}
\phantomsection\label{\detokenize{index:lb_colloids.Colloids.Colloid_Math.Drag.drag_x}}\pysigline{\sphinxbfcode{drag\_x}}
\sphinxstyleemphasis{return} \textendash{} drag force array in the x direction

\end{fulllineitems}

\index{update() (lb\_colloids.Colloids.Colloid\_Math.Drag method)}

\begin{fulllineitems}
\phantomsection\label{\detokenize{index:lb_colloids.Colloids.Colloid_Math.Drag.update}}\pysiglinewithargsret{\sphinxbfcode{update}}{\emph{vx}, \emph{vy}}{}
Updates the colloid velocity array for producing drag forces
:param vx:
:param vy:

\end{fulllineitems}


\end{fulllineitems}

\index{ForceToVelocity (class in lb\_colloids.Colloids.Colloid\_Math)}

\begin{fulllineitems}
\phantomsection\label{\detokenize{index:lb_colloids.Colloids.Colloid_Math.ForceToVelocity}}\pysiglinewithargsret{\sphinxbfcode{class }\sphinxcode{lb\_colloids.Colloids.Colloid\_Math.}\sphinxbfcode{ForceToVelocity}}{\emph{forces}, \emph{**kwargs}}{}
Class that calculates a “velocity-like” value from force arrays
\begin{quote}\begin{description}
\item[{Parameters}] \leavevmode
\sphinxstyleliteralstrong{forces} (\sphinxstyleliteralemphasis{np.ndarray}) \textendash{} Array of forces felt by a colloid

\item[{Keyword Arguments}] \leavevmode\begin{itemize}
\item {} 
\sphinxstyleliteralstrong{ts} (\sphinxstyleliteralemphasis{float}) \textendash{} Physical time step value

\item {} 
\sphinxstyleliteralstrong{rho\_colloid} (\sphinxstyleliteralemphasis{float}) \textendash{} Colloid particle density, default \(2650 kg/m^3\)

\item {} 
\sphinxstyleliteralstrong{ac} (\sphinxstyleliteralemphasis{float}) \textendash{} colloid radius, default 1e-6 m

\end{itemize}

\end{description}\end{quote}
\begin{quote}\begin{description}
\item[{Returns}] \leavevmode
velocity (np.array, np.float) Array of “velocities” calculated from forces

\end{description}\end{quote}

\end{fulllineitems}

\index{Gap (class in lb\_colloids.Colloids.Colloid\_Math)}

\begin{fulllineitems}
\phantomsection\label{\detokenize{index:lb_colloids.Colloids.Colloid_Math.Gap}}\pysiglinewithargsret{\sphinxbfcode{class }\sphinxcode{lb\_colloids.Colloids.Colloid\_Math.}\sphinxbfcode{Gap}}{\emph{xarr}, \emph{yarr}, \emph{**kwargs}}{}
Class that calculates the non-dimensional gap distance between colloid and surface.

This class also calculates hydrodynamic force correction terms outlined in Gao et. al. 2010.
Note: Passing a np.nan value into here can return an overflow warning!
\begin{equation*}
\begin{split}f_{1}(\bar{h}) = 1.0 - 0.443 exp(-1.299\bar{h}) - 0.5568 exp(-0.32\bar{h}^{0.75})\end{split}
\end{equation*}\begin{equation*}
\begin{split}f_{2}(\bar{h}) = 1.0 + 1.455 exp(-1.2596\bar{h}) - 0.7951 exp(-0.56\bar{h}^{0.50})\end{split}
\end{equation*}\begin{equation*}
\begin{split}f_{3}(\bar{h}) = 1.0 - 0.487 exp(-5.423\bar{h}) - 0.5905 exp(-37.83\bar{h}^{0.50})\end{split}
\end{equation*}\begin{equation*}
\begin{split}f_{4}(\bar{h}) = 1.0 - 0.35 exp(-0.25\bar{h}) - 0.40 exp(-10\bar{h})\end{split}
\end{equation*}\begin{quote}\begin{description}
\item[{Parameters}] \leavevmode\begin{itemize}
\item {} 
\sphinxstyleliteralstrong{xarr} (\sphinxstyleliteralemphasis{np.ndarray}) \textendash{} Array of x-distances to nearest solid surface

\item {} 
\sphinxstyleliteralstrong{yarr} (\sphinxstyleliteralemphasis{np.ndarray}) \textendash{} Array of y-distances to nearest solid surface

\end{itemize}

\item[{Keyword Arguments}] \leavevmode
\sphinxstyleliteralstrong{ac} (\sphinxstyleliteralemphasis{float}) \textendash{} Radius of a colloid. Default is 1e-6

\end{description}\end{quote}
\begin{quote}\begin{description}
\item[{Returns}] \leavevmode
f1 (np.ndarray) Drag force correction term {[}Gao et al 2010{]}

\item[{Returns}] \leavevmode
f2 (np.ndarray) Drag force correction term {[}Gao et al 2010{]}

\item[{Returns}] \leavevmode
f3 (np.ndarray) Drag force correction term {[}Gao et al 2010{]}

\item[{Returns}] \leavevmode
f4 (np.ndarray) Drag force correction term {[}Gao et al 2010{]}

\end{description}\end{quote}
\paragraph{Methods}

\end{fulllineitems}

\index{Gravity (class in lb\_colloids.Colloids.Colloid\_Math)}

\begin{fulllineitems}
\phantomsection\label{\detokenize{index:lb_colloids.Colloids.Colloid_Math.Gravity}}\pysiglinewithargsret{\sphinxbfcode{class }\sphinxcode{lb\_colloids.Colloids.Colloid\_Math.}\sphinxbfcode{Gravity}}{\emph{**kwargs}}{}
Class to generate the estimated gravitational force experienced by a colloid
\begin{equation*}
\begin{split}F^{G} = \frac{-4 \pi a_{c}^{3} \rho_{c} g}{3}\end{split}
\end{equation*}\begin{quote}\begin{description}
\item[{Keyword Arguments}] \leavevmode\begin{itemize}
\item {} 
\sphinxstyleliteralstrong{rho\_colloid} (\sphinxstyleliteralemphasis{float}) \textendash{} Particle density of a colloid in \(kg/m^3\). Default is 2650.

\item {} 
\sphinxstyleliteralstrong{ac} (\sphinxstyleliteralemphasis{float}) \textendash{} colloid radius in m. Default is 1e-6

\end{itemize}

\end{description}\end{quote}
\begin{quote}\begin{description}
\item[{Returns}] \leavevmode
gravity (float) Gravitational force that a colloid experiences

\end{description}\end{quote}

\end{fulllineitems}

\index{Velocity (class in lb\_colloids.Colloids.Colloid\_Math)}

\begin{fulllineitems}
\phantomsection\label{\detokenize{index:lb_colloids.Colloids.Colloid_Math.Velocity}}\pysiglinewithargsret{\sphinxbfcode{class }\sphinxcode{lb\_colloids.Colloids.Colloid\_Math.}\sphinxbfcode{Velocity}}{\emph{LBx}, \emph{LBy}, \emph{velocity\_factor}, \emph{**kwargs}}{}
Class that dimensionalizes LB velocity from non-dimensional lattice Boltzmann units
\begin{quote}\begin{description}
\item[{Parameters}] \leavevmode\begin{itemize}
\item {} 
\sphinxstyleliteralstrong{LBx} (\sphinxstyleliteralemphasis{np.ndarray}) \textendash{} Array of Lattice Boltzmann velocities in the x-direction

\item {} 
\sphinxstyleliteralstrong{LBy} (\sphinxstyleliteralemphasis{np.ndarray}) \textendash{} Array of Lattice Boltzmann velocities in the y-direction

\item {} 
\sphinxstyleliteralstrong{velocity\_factor} (\sphinxstyleliteralemphasis{float}) \textendash{} LB to physical velocity conversion factor. Default is 1

\end{itemize}

\item[{Keyword Arguments}] \leavevmode\begin{itemize}
\item {} 
\sphinxstyleliteralstrong{ts} (\sphinxstyleliteralemphasis{float}) \textendash{} Time step value, default is 1.

\item {} 
\sphinxstyleliteralstrong{scale\_lb} (\sphinxstyleliteralemphasis{float}) \textendash{} Scale the dimensionalized velocity from lattice Boltzmann. Use with caution. Default is 1

\end{itemize}

\end{description}\end{quote}
\begin{quote}\begin{description}
\item[{Returns}] \leavevmode
xvelocity (np.array, np.float) array of dimensionalized velocities in the x-direction

\item[{Returns}] \leavevmode
yvelocity (np.array, np.float) array of dimensionalized velocities in the y-direction

\end{description}\end{quote}

\end{fulllineitems}



\section{LB Colloid Input Output}
\label{\detokenize{index:lb-colloid-input-output}}
Colloid simulation input/output (\sphinxstyleemphasis{IO}) modules have been developed to facilitate end user functionality, such as reading configuration files, building configuration files with python, data processing, and visualization. Output modules have been developed for the user familiar with python. These modules include methods for reading HDF5 output, ascii output, performing statistical analysis on data, recovering macroscopic ADE parameters, and performing data visualization using the matplotlib library.

{\color{red}\bfseries{}\textbar{}}todo: find a place to include run\_model.py


\subsection{API documentation}
\label{\detokenize{index:id13}}\phantomsection\label{\detokenize{index:module-lb_colloids}}\index{lb\_colloids (module)}\index{cIO (in module lb\_colloids)}

\begin{fulllineitems}
\phantomsection\label{\detokenize{index:lb_colloids.cIO}}\pysigline{\sphinxcode{lb\_colloids.}\sphinxbfcode{cIO}}
alias of {\hyperref[\detokenize{index:module-lb_colloids.Colloids.Colloid_IO}]{\sphinxcrossref{\sphinxcode{lb\_colloids.Colloids.Colloid\_IO}}}}

\end{fulllineitems}

\phantomsection\label{\detokenize{index:module-lb_colloids.Colloids.Colloid_IO}}\index{lb\_colloids.Colloids.Colloid\_IO (module)}
Basic Input and Output control for Colloid Simulation models are hosted within
Colloid\_IO.py. Config and ColloidConfig are classes to set up
model dictionaries can be passed along to the main simulation routines.
ColloidConfig is a backend method for the super user. This provides a simple overridden
dictionary class that is able to build configuration files and a list object that
can be passed directly to the Config class.

Importing classes from this module follows the notation:

\begin{sphinxVerbatim}[commandchars=\\\{\}]
\PYG{g+gp}{\PYGZgt{}\PYGZgt{}\PYGZgt{} }\PYG{k+kn}{from} \PYG{n+nn}{lb\PYGZus{}colloids} \PYG{k}{import} \PYG{n}{cIO}
\PYG{g+go}{\PYGZgt{}\PYGZgt{}\PYGZgt{}}
\PYG{g+gp}{\PYGZgt{}\PYGZgt{}\PYGZgt{} }\PYG{n}{config} \PYG{o}{=} \PYG{n}{cIO}\PYG{o}{.}\PYG{n}{Config}\PYG{p}{(}\PYG{l+s+s2}{\PYGZdq{}}\PYG{l+s+s2}{Colloids.config}\PYG{l+s+s2}{\PYGZdq{}}\PYG{p}{)}
\PYG{g+gp}{\PYGZgt{}\PYGZgt{}\PYGZgt{} }\PYG{n}{cc} \PYG{o}{=} \PYG{n}{ColloidsConfig}\PYG{p}{(}\PYG{p}{)}
\PYG{g+gp}{\PYGZgt{}\PYGZgt{}\PYGZgt{} }\PYG{n}{cc}\PYG{p}{[}\PYG{l+s+s1}{\PYGZsq{}}\PYG{l+s+s1}{I}\PYG{l+s+s1}{\PYGZsq{}}\PYG{p}{]} \PYG{o}{=} \PYG{l+m+mf}{0.01}
\PYG{g+gp}{\PYGZgt{}\PYGZgt{}\PYGZgt{} }\PYG{c+c1}{\PYGZsh{} user must supply all required parameters to the ColloidsConfig dictionary, or an AssertionError will be raised}
\PYG{g+gp}{\PYGZgt{}\PYGZgt{}\PYGZgt{} }\PYG{n}{cc\PYGZus{}config} \PYG{o}{=} \PYG{n}{cc}\PYG{o}{.}\PYG{n}{config}
\PYG{g+gp}{\PYGZgt{}\PYGZgt{}\PYGZgt{} }\PYG{n}{config} \PYG{o}{=} \PYG{n}{cIO}\PYG{o}{.}\PYG{n}{Config}\PYG{p}{(}\PYG{n}{cc\PYGZus{}config}\PYG{p}{)}
\end{sphinxVerbatim}
\index{ColloidsConfig (class in lb\_colloids.Colloids.Colloid\_IO)}

\begin{fulllineitems}
\phantomsection\label{\detokenize{index:lb_colloids.Colloids.Colloid_IO.ColloidsConfig}}\pysigline{\sphinxbfcode{class }\sphinxcode{lb\_colloids.Colloids.Colloid\_IO.}\sphinxbfcode{ColloidsConfig}}
OO class to build config files, or build a list that the Config class will recognize and parse
Recomended setup method for the super user who is looping many models.
Facilitates easy sensitivity analysis, etc….

Class uses a dictionary override to set parameters to the class, and writes them out as a list
or as a configuration file

example class usage:

\begin{sphinxVerbatim}[commandchars=\\\{\}]
\PYG{g+gp}{\PYGZgt{}\PYGZgt{}\PYGZgt{} }\PYG{k+kn}{from} \PYG{n+nn}{lb\PYGZus{}colloids} \PYG{k}{import} \PYG{n}{cIO}
\PYG{g+gp}{\PYGZgt{}\PYGZgt{}\PYGZgt{} }\PYG{n}{cconfig} \PYG{o}{=} \PYG{n}{cIO}\PYG{o}{.}\PYG{n}{ColloidsConfig}\PYG{p}{(}\PYG{p}{)}
\PYG{g+gp}{\PYGZgt{}\PYGZgt{}\PYGZgt{} }\PYG{n}{cconfig}\PYG{p}{[}\PYG{l+s+s1}{\PYGZsq{}}\PYG{l+s+s1}{I}\PYG{l+s+s1}{\PYGZsq{}}\PYG{p}{]} \PYG{o}{=} \PYG{l+m+mf}{0.1}
\PYG{g+gp}{\PYGZgt{}\PYGZgt{}\PYGZgt{} }\PYG{n}{cconfig}\PYG{p}{[}\PYG{l+s+s1}{\PYGZsq{}}\PYG{l+s+s1}{ncols}\PYG{l+s+s1}{\PYGZsq{}}\PYG{p}{]} \PYG{o}{=} \PYG{l+m+mi}{500}
\PYG{g+gp}{\PYGZgt{}\PYGZgt{}\PYGZgt{} }\PYG{n}{x} \PYG{o}{=} \PYG{n}{cconfig}\PYG{o}{.}\PYG{n}{config}  \PYG{c+c1}{\PYGZsh{} returns a formatted list that imitates a colloids configuration file}
\PYG{g+gp}{\PYGZgt{}\PYGZgt{}\PYGZgt{} }\PYG{n}{config} \PYG{o}{=} \PYG{n}{cIO}\PYG{o}{.}\PYG{n}{Config}\PYG{p}{(}\PYG{n}{x}\PYG{p}{)}
\end{sphinxVerbatim}
\paragraph{Attributes}
\paragraph{Methods}
\index{chemical\_parameters (lb\_colloids.Colloids.Colloid\_IO.ColloidsConfig attribute)}

\begin{fulllineitems}
\phantomsection\label{\detokenize{index:lb_colloids.Colloids.Colloid_IO.ColloidsConfig.chemical_parameters}}\pysigline{\sphinxbfcode{chemical\_parameters}}
Current user supplied chemical parameters

\end{fulllineitems}

\index{config (lb\_colloids.Colloids.Colloid\_IO.ColloidsConfig attribute)}

\begin{fulllineitems}
\phantomsection\label{\detokenize{index:lb_colloids.Colloids.Colloid_IO.ColloidsConfig.config}}\pysigline{\sphinxbfcode{config}}
Property method that creates the config list on the fly for the user from the overridden dictionary
\begin{quote}\begin{description}
\item[{Returns}] \leavevmode
self.\_\_config (list) formatted configuration file list

\end{description}\end{quote}

\end{fulllineitems}

\index{model\_parameters (lb\_colloids.Colloids.Colloid\_IO.ColloidsConfig attribute)}

\begin{fulllineitems}
\phantomsection\label{\detokenize{index:lb_colloids.Colloids.Colloid_IO.ColloidsConfig.model_parameters}}\pysigline{\sphinxbfcode{model\_parameters}}
Current user supplied model parameters

\end{fulllineitems}

\index{output\_control\_parameters (lb\_colloids.Colloids.Colloid\_IO.ColloidsConfig attribute)}

\begin{fulllineitems}
\phantomsection\label{\detokenize{index:lb_colloids.Colloids.Colloid_IO.ColloidsConfig.output_control_parameters}}\pysigline{\sphinxbfcode{output\_control\_parameters}}
Current user supplied output control parameters

\end{fulllineitems}

\index{physical\_parameters (lb\_colloids.Colloids.Colloid\_IO.ColloidsConfig attribute)}

\begin{fulllineitems}
\phantomsection\label{\detokenize{index:lb_colloids.Colloids.Colloid_IO.ColloidsConfig.physical_parameters}}\pysigline{\sphinxbfcode{physical\_parameters}}
Current user supplied physical parameters

\end{fulllineitems}

\index{valid\_chemical\_parameters (lb\_colloids.Colloids.Colloid\_IO.ColloidsConfig attribute)}

\begin{fulllineitems}
\phantomsection\label{\detokenize{index:lb_colloids.Colloids.Colloid_IO.ColloidsConfig.valid_chemical_parameters}}\pysigline{\sphinxbfcode{valid\_chemical\_parameters}}
List of valid chemical parameters

\end{fulllineitems}

\index{valid\_model\_parameters (lb\_colloids.Colloids.Colloid\_IO.ColloidsConfig attribute)}

\begin{fulllineitems}
\phantomsection\label{\detokenize{index:lb_colloids.Colloids.Colloid_IO.ColloidsConfig.valid_model_parameters}}\pysigline{\sphinxbfcode{valid\_model\_parameters}}
List of valid model parameters

\end{fulllineitems}

\index{valid\_output\_control\_parameters (lb\_colloids.Colloids.Colloid\_IO.ColloidsConfig attribute)}

\begin{fulllineitems}
\phantomsection\label{\detokenize{index:lb_colloids.Colloids.Colloid_IO.ColloidsConfig.valid_output_control_parameters}}\pysigline{\sphinxbfcode{valid\_output\_control\_parameters}}
List of valid output control parameters

\end{fulllineitems}

\index{valid\_physical\_parameters (lb\_colloids.Colloids.Colloid\_IO.ColloidsConfig attribute)}

\begin{fulllineitems}
\phantomsection\label{\detokenize{index:lb_colloids.Colloids.Colloid_IO.ColloidsConfig.valid_physical_parameters}}\pysigline{\sphinxbfcode{valid\_physical\_parameters}}
List of valid physical parameters

\end{fulllineitems}

\index{write() (lb\_colloids.Colloids.Colloid\_IO.ColloidsConfig method)}

\begin{fulllineitems}
\phantomsection\label{\detokenize{index:lb_colloids.Colloids.Colloid_IO.ColloidsConfig.write}}\pysiglinewithargsret{\sphinxbfcode{write}}{\emph{fname}}{}
Writes a configuration file with user supplied parameters to file.
\begin{quote}\begin{description}
\item[{Parameters}] \leavevmode
\sphinxstyleliteralstrong{fname} (\sphinxstyleliteralemphasis{str}) \textendash{} Configuration file name to write

\end{description}\end{quote}

\end{fulllineitems}


\end{fulllineitems}

\index{Config (class in lb\_colloids.Colloids.Colloid\_IO)}

\begin{fulllineitems}
\phantomsection\label{\detokenize{index:lb_colloids.Colloids.Colloid_IO.Config}}\pysiglinewithargsret{\sphinxbfcode{class }\sphinxcode{lb\_colloids.Colloids.Colloid\_IO.}\sphinxbfcode{Config}}{\emph{fname}}{}
Class to open and parse configuration files for LB-Colloids. Many data checks
have been implemented to look for consistancy in data type and
configuration variable for each input block.
\begin{quote}\begin{description}
\item[{Parameters}] \leavevmode
\sphinxstyleliteralstrong{fname} (\sphinxstyleliteralemphasis{str}) \textendash{} Configuration file name ex. Model.config. This class can also accepts

\end{description}\end{quote}

a list of configuration variables that have been set up by cIO.ColloidsConfig()
\begin{quote}\begin{description}
\item[{Returns}] \leavevmode
model\_parameters (dict) Dictionary of necessary model parameters

\item[{Returns}] \leavevmode
physical\_parameters (dict) Dictionary of optional physical parameters

\item[{Returns}] \leavevmode
chemical\_parmaeters (dict) Dictionary of chemical parameter options

\item[{Returns}] \leavevmode
output\_control (dict) Dictionary of output control options

\end{description}\end{quote}
\paragraph{Methods}
\index{add\_universal\_parameters() (lb\_colloids.Colloids.Colloid\_IO.Config method)}

\begin{fulllineitems}
\phantomsection\label{\detokenize{index:lb_colloids.Colloids.Colloid_IO.Config.add_universal_parameters}}\pysiglinewithargsret{\sphinxbfcode{add\_universal\_parameters}}{\emph{Dict}}{}
Add common model parameters to other dictionaries if present. Necessary for parameterization by
kwargs of physics and chemistry.
\begin{quote}\begin{description}
\item[{Parameters}] \leavevmode
\sphinxstyleliteralstrong{Dict} (\sphinxstyleliteralemphasis{dict}) \textendash{} dictionary to add set of universal paramameters from the required parameters

\item[{Returns}] \leavevmode
Dict (dict)

\end{description}\end{quote}

\end{fulllineitems}

\index{adjust\_pname() (lb\_colloids.Colloids.Colloid\_IO.Config method)}

\begin{fulllineitems}
\phantomsection\label{\detokenize{index:lb_colloids.Colloids.Colloid_IO.Config.adjust_pname}}\pysiglinewithargsret{\sphinxbfcode{adjust\_pname}}{\emph{pname}}{}
Adjusts parameter name from configuration file name to LB-Colloids name for a limited number of
parameters. Sets all other parameters as lowercase to follow PEP-8
\begin{quote}\begin{description}
\item[{Parameters}] \leavevmode
\sphinxstyleliteralstrong{pname} (\sphinxstyleliteralemphasis{str}) \textendash{} parameter name

\end{description}\end{quote}

\end{fulllineitems}

\index{check\_if\_valid() (lb\_colloids.Colloids.Colloid\_IO.Config method)}

\begin{fulllineitems}
\phantomsection\label{\detokenize{index:lb_colloids.Colloids.Colloid_IO.Config.check_if_valid}}\pysiglinewithargsret{\sphinxbfcode{check\_if\_valid}}{\emph{blockname}, \emph{pname}, \emph{validparams}}{}
Method checks if a specific parameter is valid for the block it was supplied to
in the configuration file.
\begin{quote}\begin{description}
\item[{Parameters}] \leavevmode\begin{itemize}
\item {} 
\sphinxstyleliteralstrong{blockname} (\sphinxstyleliteralemphasis{str}) \textendash{} Name of the input block

\item {} 
\sphinxstyleliteralstrong{pname} (\sphinxstyleliteralemphasis{str}) \textendash{} 

\item {} 
\sphinxstyleliteralstrong{validparams} (\sphinxstyleliteralemphasis{tuple}) \textendash{} tuple of valid parameter names for the specific input block

\end{itemize}

\end{description}\end{quote}

\end{fulllineitems}

\index{check\_model\_parameters() (lb\_colloids.Colloids.Colloid\_IO.Config method)}

\begin{fulllineitems}
\phantomsection\label{\detokenize{index:lb_colloids.Colloids.Colloid_IO.Config.check_model_parameters}}\pysiglinewithargsret{\sphinxbfcode{check\_model\_parameters}}{\emph{ModelDict}}{}
Check for required parameters. If not present inform the user which parameter(s) are needed
\begin{quote}\begin{description}
\item[{Parameters}] \leavevmode
\sphinxstyleliteralstrong{ModelDict} (\sphinxstyleliteralemphasis{dict}) \textendash{} Checks the model\_dict for all required parameters.

\item[{Raises}] \leavevmode
\sphinxstyleliteralstrong{AssertionError} \textendash{} If all required parameters are not supplied

\end{description}\end{quote}

\end{fulllineitems}

\index{chemical\_parameters() (lb\_colloids.Colloids.Colloid\_IO.Config method)}

\begin{fulllineitems}
\phantomsection\label{\detokenize{index:lb_colloids.Colloids.Colloid_IO.Config.chemical_parameters}}\pysiglinewithargsret{\sphinxbfcode{chemical\_parameters}}{}{}
Reads the CHEMICAL PARAMETERS block of the configuration file and creates
the chemical\_dict that passes optional parameters to the chemical force calculations
in the Colloid\_Math.py module

\end{fulllineitems}

\index{get\_block() (lb\_colloids.Colloids.Colloid\_IO.Config method)}

\begin{fulllineitems}
\phantomsection\label{\detokenize{index:lb_colloids.Colloids.Colloid_IO.Config.get_block}}\pysiglinewithargsret{\sphinxbfcode{get\_block}}{\emph{blockname}}{}
Method to isolate an input block from a configuration file for parsing
\begin{quote}\begin{description}
\item[{Parameters}] \leavevmode
\sphinxstyleliteralstrong{blockname} (\sphinxstyleliteralemphasis{str}) \textendash{} Blockname of an input block in the configuration file.

\end{description}\end{quote}
\begin{quote}\begin{description}
\item[{Returns}] \leavevmode
(list) returns a list of parameters contained within the block

\end{description}\end{quote}

\end{fulllineitems}

\index{model\_parameters() (lb\_colloids.Colloids.Colloid\_IO.Config method)}

\begin{fulllineitems}
\phantomsection\label{\detokenize{index:lb_colloids.Colloids.Colloid_IO.Config.model_parameters}}\pysiglinewithargsret{\sphinxbfcode{model\_parameters}}{}{}
Reads the MODEL PARAMETERS block of the configuration file and creates
the model\_dict which contains the required parameters to run LB-Colloid

\end{fulllineitems}

\index{output\_control() (lb\_colloids.Colloids.Colloid\_IO.Config method)}

\begin{fulllineitems}
\phantomsection\label{\detokenize{index:lb_colloids.Colloids.Colloid_IO.Config.output_control}}\pysiglinewithargsret{\sphinxbfcode{output\_control}}{}{}
Reads the OUTPUT CONTROL block of the configuration file and creates
the output\_dict that passes optional parameters to control model output

\end{fulllineitems}

\index{parametertype() (lb\_colloids.Colloids.Colloid\_IO.Config method)}

\begin{fulllineitems}
\phantomsection\label{\detokenize{index:lb_colloids.Colloids.Colloid_IO.Config.parametertype}}\pysiglinewithargsret{\sphinxbfcode{parametertype}}{\emph{parameter}}{}
Method takes a parameter string, splits it, and sets the parameter type
\begin{quote}\begin{description}
\item[{Parameters}] \leavevmode
\sphinxstyleliteralstrong{parameter} (\sphinxstyleliteralemphasis{str}) \textendash{} String of “\textless{}pname\textgreater{}: \textless{}param\textgreater{}”

\end{description}\end{quote}
\begin{quote}\begin{description}
\item[{Returns}] \leavevmode
(pname, param) (tuple) parameter name, value

\end{description}\end{quote}

\end{fulllineitems}

\index{physical\_parameters() (lb\_colloids.Colloids.Colloid\_IO.Config method)}

\begin{fulllineitems}
\phantomsection\label{\detokenize{index:lb_colloids.Colloids.Colloid_IO.Config.physical_parameters}}\pysiglinewithargsret{\sphinxbfcode{physical\_parameters}}{}{}
Reads the PHYSICAL PARAMETERS block of the configuration file and creates
the physics\_dict that passes optional parameters to the physical force calculations
in the Colloid\_Math.py module

\end{fulllineitems}


\end{fulllineitems}

\index{HDF5WriteArray (class in lb\_colloids.Colloids.Colloid\_IO)}

\begin{fulllineitems}
\phantomsection\label{\detokenize{index:lb_colloids.Colloids.Colloid_IO.HDF5WriteArray}}\pysiglinewithargsret{\sphinxbfcode{class }\sphinxcode{lb\_colloids.Colloids.Colloid\_IO.}\sphinxbfcode{HDF5WriteArray}}{\emph{ux}, \emph{uy}, \emph{colloidcolloid}, \emph{model\_dict}, \emph{chemical\_dict}, \emph{physical\_dict}}{}
Class to write chemical and physical force arrays to the Model HDF5 object
for later use in data processing and analysis.
\begin{quote}\begin{description}
\item[{Parameters}] \leavevmode\begin{itemize}
\item {} 
\sphinxstyleliteralstrong{ux} (\sphinxstyleliteralemphasis{np.ndarray}) \textendash{} Dimensionalized fluid velocity in the x-direction

\item {} 
\sphinxstyleliteralstrong{uy} (\sphinxstyleliteralemphasis{np.ndarray}) \textendash{} Dimensionalized fluid velocity in the y-direction

\item {} 
\sphinxstyleliteralstrong{colloidcolloid} ({\hyperref[\detokenize{index:lb_colloids.Colloids.Colloid_Math.ColloidColloid}]{\sphinxcrossref{\sphinxstyleliteralemphasis{Colloid\_Math.ColloidColloid}}}}) \textendash{} ColloidColloid object

\item {} 
\sphinxstyleliteralstrong{model\_dict} (\sphinxstyleliteralemphasis{dict}) \textendash{} The supplied model dict from parameterization

\item {} 
\sphinxstyleliteralstrong{chemical\_dict} (\sphinxstyleliteralemphasis{dict}) \textendash{} The supplied chemical dict used for parameterization

\item {} 
\sphinxstyleliteralstrong{physical\_dict} (\sphinxstyleliteralemphasis{dict}) \textendash{} The supplied physical dict used for parameterization

\end{itemize}

\end{description}\end{quote}

\end{fulllineitems}

\index{Output (class in lb\_colloids.Colloids.Colloid\_IO)}

\begin{fulllineitems}
\phantomsection\label{\detokenize{index:lb_colloids.Colloids.Colloid_IO.Output}}\pysiglinewithargsret{\sphinxbfcode{class }\sphinxcode{lb\_colloids.Colloids.Colloid\_IO.}\sphinxbfcode{Output}}{\emph{fi}, \emph{**kwargs}}{}
Output class for writing formatted ASCII files. This class generates
\textless{}.endpoint\textgreater{}, \textless{}.timeseries\textgreater{} and \textless{}.pathline\textgreater{} files. All keywords are required.
\begin{quote}\begin{description}
\item[{Parameters}] \leavevmode
\sphinxstyleliteralstrong{fi} (\sphinxstyleliteralemphasis{str}) \textendash{} filename of the output.

\item[{Keyword Arguments}] \leavevmode\begin{itemize}
\item {} 
\sphinxstyleliteralstrong{overwrite} (\sphinxstyleliteralemphasis{bool}) \textendash{} Determines if file is overwritten, or appended to.
useful for generating new pathline and timeseries files

\item {} 
\sphinxstyleliteralstrong{ts} (\sphinxstyleliteralemphasis{float}) \textendash{} Physical time step

\item {} 
\sphinxstyleliteralstrong{lbres} (\sphinxstyleliteralemphasis{float}) \textendash{} Lattice Boltzmann resolution

\item {} 
\sphinxstyleliteralstrong{gridref} (\sphinxstyleliteralemphasis{float}) \textendash{} Grid refinement factor

\item {} 
\sphinxstyleliteralstrong{ncols} (\sphinxstyleliteralemphasis{int}) \textendash{} number of colloids simulated

\item {} 
\sphinxstyleliteralstrong{xlen} (\sphinxstyleliteralemphasis{int}) \textendash{} length of the xdomain in pixels

\item {} 
\sphinxstyleliteralstrong{ylen} (\sphinxstyleliteralemphasis{int}) \textendash{} length of the ydomain in pixels

\item {} 
\sphinxstyleliteralstrong{mean\_ux} (\sphinxstyleliteralemphasis{float}) \textendash{} mean fluid velocity in x-direction

\item {} 
\sphinxstyleliteralstrong{mean\_uy} (\sphinxstyleliteralemphasis{float}) \textendash{} mean fluid velocity in y-direction

\item {} 
\sphinxstyleliteralstrong{continuous} (\sphinxstyleliteralemphasis{int}) \textendash{} flag for continuous release of colloids

\end{itemize}

\end{description}\end{quote}
\paragraph{Methods}
\index{write\_output() (lb\_colloids.Colloids.Colloid\_IO.Output method)}

\begin{fulllineitems}
\phantomsection\label{\detokenize{index:lb_colloids.Colloids.Colloid_IO.Output.write_output}}\pysiglinewithargsret{\sphinxbfcode{write\_output}}{\emph{timer}, \emph{colloids}, \emph{pathline=True}}{}
Set up and write colloid streaming output to an ASCII file
\begin{quote}\begin{description}
\item[{Parameters}] \leavevmode\begin{itemize}
\item {} 
\sphinxstyleliteralstrong{timer} ({\hyperref[\detokenize{index:lb_colloids.Colloids.LB_Colloid.TrackTime}]{\sphinxcrossref{\sphinxstyleliteralemphasis{TrackTime}}}}) \textendash{} Model TrackTime instance

\item {} 
\sphinxstyleliteralstrong{colloids} (\sphinxstyleliteralemphasis{list}) \textendash{} Colloid simulation list containing LB\_Colloid.Colloid objects

\item {} 
\sphinxstyleliteralstrong{pathline} (\sphinxstyleliteralemphasis{bool}) \textendash{} Flag to indicate if an endpoint or pathline/timeseries is being written.

\end{itemize}

\end{description}\end{quote}

\end{fulllineitems}

\index{write\_single\_colloid() (lb\_colloids.Colloids.Colloid\_IO.Output method)}

\begin{fulllineitems}
\phantomsection\label{\detokenize{index:lb_colloids.Colloids.Colloid_IO.Output.write_single_colloid}}\pysiglinewithargsret{\sphinxbfcode{write\_single\_colloid}}{\emph{timer}, \emph{colloid}}{}
Method to write a single colloid to an endpoint file upon breakthrough
:param TrackTime timer: Model TrackTime instance
:param LB\_Colloid.Colloid colloid:

\end{fulllineitems}


\end{fulllineitems}

\phantomsection\label{\detokenize{index:module-lb_colloids}}\index{lb\_colloids (module)}\index{ColloidOutput (in module lb\_colloids)}

\begin{fulllineitems}
\phantomsection\label{\detokenize{index:lb_colloids.ColloidOutput}}\pysigline{\sphinxcode{lb\_colloids.}\sphinxbfcode{ColloidOutput}}
alias of {\hyperref[\detokenize{index:module-lb_colloids.Colloids.Colloid_output}]{\sphinxcrossref{\sphinxcode{lb\_colloids.Colloids.Colloid\_output}}}}

\end{fulllineitems}

\phantomsection\label{\detokenize{index:module-lb_colloids.Colloids.Colloid_output}}\index{lb\_colloids.Colloids.Colloid\_output (module)}
The Colloid\_output module contains classes to read LB Colloids simulation
outputs and perform post processing. Many classes are available to
provide plotting functionality. ModelPlot and CCModelPlot are useful for
visualizing colloid-surface forces and colloid-colloid forces respectively.

example import of the Colloid\_output.py module is as follows

\begin{sphinxVerbatim}[commandchars=\\\{\}]
\PYG{g+gp}{\PYGZgt{}\PYGZgt{}\PYGZgt{} }\PYG{k+kn}{from} \PYG{n+nn}{lb\PYGZus{}colloids} \PYG{k}{import} \PYG{n}{ColloidOutput}
\PYG{g+gp}{\PYGZgt{}\PYGZgt{}\PYGZgt{} }\PYG{k+kn}{import} \PYG{n+nn}{matplotlib}\PYG{n+nn}{.}\PYG{n+nn}{pyplot} \PYG{k}{as} \PYG{n+nn}{plt}
\PYG{g+go}{\PYGZgt{}\PYGZgt{}\PYGZgt{}}
\PYG{g+gp}{\PYGZgt{}\PYGZgt{}\PYGZgt{} }\PYG{n}{hdf} \PYG{o}{=} \PYG{l+s+s2}{\PYGZdq{}}\PYG{l+s+s2}{mymodel.hdf5}\PYG{l+s+s2}{\PYGZdq{}}
\PYG{g+gp}{\PYGZgt{}\PYGZgt{}\PYGZgt{} }\PYG{n}{mp} \PYG{o}{=} \PYG{n}{ColloidOutput}\PYG{o}{.}\PYG{n}{ModelPlot}\PYG{p}{(}\PYG{n}{hdf}\PYG{p}{)}
\PYG{g+gp}{\PYGZgt{}\PYGZgt{}\PYGZgt{} }\PYG{c+c1}{\PYGZsh{} model plot accepts matplotlib args and kwargs!!!}
\PYG{g+gp}{\PYGZgt{}\PYGZgt{}\PYGZgt{} }\PYG{n}{mp}\PYG{o}{.}\PYG{n}{plot}\PYG{p}{(}\PYG{l+s+s1}{\PYGZsq{}}\PYG{l+s+s1}{edl\PYGZus{}x}\PYG{l+s+s1}{\PYGZsq{}}\PYG{p}{,} \PYG{n}{cmap}\PYG{o}{=}\PYG{l+s+s1}{\PYGZsq{}}\PYG{l+s+s1}{viridis}\PYG{l+s+s1}{\PYGZsq{}}\PYG{p}{)}
\PYG{g+gp}{\PYGZgt{}\PYGZgt{}\PYGZgt{} }\PYG{n}{plt}\PYG{o}{.}\PYG{n}{show}\PYG{p}{(}\PYG{p}{)}
\end{sphinxVerbatim}
\index{ADE (class in lb\_colloids.Colloids.Colloid\_output)}

\begin{fulllineitems}
\phantomsection\label{\detokenize{index:lb_colloids.Colloids.Colloid_output.ADE}}\pysiglinewithargsret{\sphinxbfcode{class }\sphinxcode{lb\_colloids.Colloids.Colloid\_output.}\sphinxbfcode{ADE}}{\emph{filename}, \emph{nbin=1000}}{}
Class to calculate macroscopic advection dispersion
equation parameters for field scale model parameterization

Class needs to be re-named and updated to CDE equation
\begin{quote}\begin{description}
\item[{Parameters}] \leavevmode\begin{itemize}
\item {} 
\sphinxstyleliteralstrong{filename} (\sphinxstyleliteralemphasis{str}) \textendash{} ascii output file name from colloid model

\item {} 
\sphinxstyleliteralstrong{nbin} (\sphinxstyleliteralemphasis{int}) \textendash{} number of timesteps to bin a pdf for calculation

\end{itemize}

\end{description}\end{quote}
\paragraph{Methods}
\index{reset\_pdf() (lb\_colloids.Colloids.Colloid\_output.ADE method)}

\begin{fulllineitems}
\phantomsection\label{\detokenize{index:lb_colloids.Colloids.Colloid_output.ADE.reset_pdf}}\pysiglinewithargsret{\sphinxbfcode{reset\_pdf}}{\emph{nbin}, \emph{normalize=False}}{}
User method to reset values based on changing
the pdf bin values
\begin{quote}\begin{description}
\item[{Parameters}] \leavevmode\begin{itemize}
\item {} 
\sphinxstyleliteralstrong{nbin} (\sphinxstyleliteralemphasis{int}) \textendash{} number of timesteps to bin a pdf for calculation

\item {} 
\sphinxstyleliteralstrong{normalize} (\sphinxstyleliteralemphasis{bool}) \textendash{} flag to calculate pdf by residence time or end time

\end{itemize}

\end{description}\end{quote}

\end{fulllineitems}

\index{solve\_jury\_1991() (lb\_colloids.Colloids.Colloid\_output.ADE method)}

\begin{fulllineitems}
\phantomsection\label{\detokenize{index:lb_colloids.Colloids.Colloid_output.ADE.solve_jury_1991}}\pysiglinewithargsret{\sphinxbfcode{solve\_jury\_1991}}{\emph{D=0.01}, \emph{R=0.01}, \emph{ftol=1e-10}, \emph{max\_nfev=1000}, \emph{**kwargs}}{}
Scipy optimize method to solve least sqares
for jury 1991. Pulse flux.
\begin{quote}\begin{description}
\item[{Parameters}] \leavevmode\begin{itemize}
\item {} 
\sphinxstyleliteralstrong{D} (\sphinxstyleliteralemphasis{float}) \textendash{} Diffusivity initial guess. Cannot be 0

\item {} 
\sphinxstyleliteralstrong{R} (\sphinxstyleliteralemphasis{float}) \textendash{} Retardation initial guess. Cannot be 0

\item {} 
\sphinxstyleliteralstrong{ftol} (\sphinxstyleliteralemphasis{float}) \textendash{} scipy function tolerance for solution

\item {} 
\sphinxstyleliteralstrong{max\_nfev} (\sphinxstyleliteralemphasis{int}) \textendash{} maximum number of function iterations

\item {} 
\sphinxstyleliteralstrong{**kwargs} \textendash{} 
scipy least squares kwargs


\end{itemize}

\end{description}\end{quote}
\begin{quote}\begin{description}
\item[{Returns}] \leavevmode
scipy least squares dictionary. Answer in dict{[}‘x’{]}

\end{description}\end{quote}

\end{fulllineitems}

\index{solve\_van\_genuchten\_1986() (lb\_colloids.Colloids.Colloid\_output.ADE method)}

\begin{fulllineitems}
\phantomsection\label{\detokenize{index:lb_colloids.Colloids.Colloid_output.ADE.solve_van_genuchten_1986}}\pysiglinewithargsret{\sphinxbfcode{solve\_van\_genuchten\_1986}}{\emph{D=0.01}, \emph{R=0.01}, \emph{ftol=1e-10}, \emph{max\_nfev=1000}, \emph{**kwargs}}{}
Scipy optimize method to solve least squares
for van genuchten 1986. Miscable displacement.
\begin{quote}\begin{description}
\item[{Parameters}] \leavevmode\begin{itemize}
\item {} 
\sphinxstyleliteralstrong{D} (\sphinxstyleliteralemphasis{float}) \textendash{} Diffusivity initial guess. Cannot be 0

\item {} 
\sphinxstyleliteralstrong{R} (\sphinxstyleliteralemphasis{float}) \textendash{} Retardation initial guess. Cannot be 0

\item {} 
\sphinxstyleliteralstrong{ftol} (\sphinxstyleliteralemphasis{float}) \textendash{} scipy function tolerance for solution

\item {} 
\sphinxstyleliteralstrong{max\_nfev} (\sphinxstyleliteralemphasis{int}) \textendash{} maximum number of function iterations

\item {} 
\sphinxstyleliteralstrong{**kwargs} \textendash{} 
scipy least squares kwargs


\end{itemize}

\end{description}\end{quote}
\begin{quote}\begin{description}
\item[{Returns}] \leavevmode
scipy least squares dictionary. Answer in dict{[}‘x’{]}

\end{description}\end{quote}

\end{fulllineitems}


\end{fulllineitems}

\index{ASCIIReader (class in lb\_colloids.Colloids.Colloid\_output)}

\begin{fulllineitems}
\phantomsection\label{\detokenize{index:lb_colloids.Colloids.Colloid_output.ASCIIReader}}\pysiglinewithargsret{\sphinxbfcode{class }\sphinxcode{lb\_colloids.Colloids.Colloid\_output.}\sphinxbfcode{ASCIIReader}}{\emph{filename}}{}
Class to read in text based output files \textless{}endpoint, timestep, pathline\textgreater{}
to a pandas dataframe
\begin{quote}\begin{description}
\item[{Parameters}] \leavevmode
\sphinxstyleliteralstrong{filename} (\sphinxstyleliteralemphasis{str}) \textendash{} output filename (ie. endpoint, timestep, or pathline)

\end{description}\end{quote}
\paragraph{Methods}
\index{read\_ascii() (lb\_colloids.Colloids.Colloid\_output.ASCIIReader method)}

\begin{fulllineitems}
\phantomsection\label{\detokenize{index:lb_colloids.Colloids.Colloid_output.ASCIIReader.read_ascii}}\pysiglinewithargsret{\sphinxbfcode{read\_ascii}}{\emph{filename}}{}
Method to read endpoint file data from from ascii files for LB-Colloids
Sets data to pandas dataframe
\begin{quote}\begin{description}
\item[{Parameters}] \leavevmode
\sphinxstyleliteralstrong{filename} (\sphinxstyleliteralemphasis{str}) \textendash{} colloid model output filename (ie. endpoint, timestep, or pathline)

\end{description}\end{quote}

\end{fulllineitems}

\index{read\_header() (lb\_colloids.Colloids.Colloid\_output.ASCIIReader method)}

\begin{fulllineitems}
\phantomsection\label{\detokenize{index:lb_colloids.Colloids.Colloid_output.ASCIIReader.read_header}}\pysiglinewithargsret{\sphinxbfcode{read\_header}}{\emph{filename}}{}
Method to read the header from ascii output files for LB-Colloids
\begin{quote}\begin{description}
\item[{Parameters}] \leavevmode
\sphinxstyleliteralstrong{filename} (\sphinxstyleliteralemphasis{str}) \textendash{} colloid model output filename (ie. endpoint, timestep, or pathline)

\end{description}\end{quote}

\end{fulllineitems}


\end{fulllineitems}

\index{Breakthrough (class in lb\_colloids.Colloids.Colloid\_output)}

\begin{fulllineitems}
\phantomsection\label{\detokenize{index:lb_colloids.Colloids.Colloid_output.Breakthrough}}\pysiglinewithargsret{\sphinxbfcode{class }\sphinxcode{lb\_colloids.Colloids.Colloid\_output.}\sphinxbfcode{Breakthrough}}{\emph{filename}}{}
Class to prepare and plot breakthrough curve data from endpoint
files.
\begin{quote}\begin{description}
\item[{Parameters}] \leavevmode
\sphinxstyleliteralstrong{filename} (\sphinxstyleliteralemphasis{str}) \textendash{} \textless{}\textgreater{}.endpoint file

\end{description}\end{quote}
\begin{quote}\begin{description}
\item[{Variables}] \leavevmode\begin{itemize}
\item {} 
\sphinxstyleliteralstrong{df} \textendash{} (pandas DataFrame): dataframe of endpoint data

\item {} 
{\hyperref[\detokenize{index:lb_colloids.LB.LB_2Dpermeability.LB2DModel.resolution}]{\sphinxcrossref{\sphinxstyleliteralstrong{resolution}}}} \textendash{} (float): model resolution

\item {} 
\sphinxstyleliteralstrong{timestep} \textendash{} (float): model timestep

\item {} 
\sphinxstyleliteralstrong{continuous} \textendash{} (int): interval of continuous release, 0 means pulse

\item {} 
\sphinxstyleliteralstrong{ncol} \textendash{} (float): number of colloids per release in simulation

\item {} 
\sphinxstyleliteralstrong{total\_ncol} \textendash{} (int): total number of colloids in simulation

\end{itemize}

\end{description}\end{quote}
\paragraph{Attributes}
\paragraph{Methods}
\index{breakthrough\_curve (lb\_colloids.Colloids.Colloid\_output.Breakthrough attribute)}

\begin{fulllineitems}
\phantomsection\label{\detokenize{index:lb_colloids.Colloids.Colloid_output.Breakthrough.breakthrough_curve}}\pysigline{\sphinxbfcode{breakthrough\_curve}}
Property method that performs a dynamic
calculation of breakthrough curve data

\end{fulllineitems}

\index{plot() (lb\_colloids.Colloids.Colloid\_output.Breakthrough method)}

\begin{fulllineitems}
\phantomsection\label{\detokenize{index:lb_colloids.Colloids.Colloid_output.Breakthrough.plot}}\pysiglinewithargsret{\sphinxbfcode{plot}}{\emph{time=True}, \emph{*args}, \emph{**kwargs}}{}
Convience method to plot data into a matplotlib
chart.
\begin{quote}\begin{description}
\item[{Parameters}] \leavevmode\begin{itemize}
\item {} 
\sphinxstyleliteralstrong{time} (\sphinxstyleliteralemphasis{bool}) \textendash{} if true x-axis is time, false is nts

\item {} 
\sphinxstyleliteralstrong{*args} \textendash{} 
matplotlib args for 1d charts


\item {} 
\sphinxstyleliteralstrong{**kwargs} \textendash{} 
matplotlib keyword arguments for 1d charts


\end{itemize}

\end{description}\end{quote}

\end{fulllineitems}

\index{plot\_pv() (lb\_colloids.Colloids.Colloid\_output.Breakthrough method)}

\begin{fulllineitems}
\phantomsection\label{\detokenize{index:lb_colloids.Colloids.Colloid_output.Breakthrough.plot_pv}}\pysiglinewithargsret{\sphinxbfcode{plot\_pv}}{\emph{*args}, \emph{**kwargs}}{}
Method to plot breakthrough data with pore
volumes (non-dimensional time)
\begin{quote}\begin{description}
\item[{Parameters}] \leavevmode\begin{itemize}
\item {} 
\sphinxstyleliteralstrong{*args} \textendash{} 
matplotlib args for 1d plotting


\item {} 
\sphinxstyleliteralstrong{**kwargs} \textendash{} 
matplotlib kwargs for 1d plotting


\end{itemize}

\end{description}\end{quote}

\end{fulllineitems}

\index{pore\_volume\_conversion() (lb\_colloids.Colloids.Colloid\_output.Breakthrough method)}

\begin{fulllineitems}
\phantomsection\label{\detokenize{index:lb_colloids.Colloids.Colloid_output.Breakthrough.pore_volume_conversion}}\pysiglinewithargsret{\sphinxbfcode{pore\_volume\_conversion}}{}{}
Method to retrieve the pore volume calculation
conversion for plotting colloids.

\end{fulllineitems}


\end{fulllineitems}

\index{CCModelPlot (class in lb\_colloids.Colloids.Colloid\_output)}

\begin{fulllineitems}
\phantomsection\label{\detokenize{index:lb_colloids.Colloids.Colloid_output.CCModelPlot}}\pysiglinewithargsret{\sphinxbfcode{class }\sphinxcode{lb\_colloids.Colloids.Colloid\_output.}\sphinxbfcode{CCModelPlot}}{\emph{hdf5}}{}
Class to query colloid-colloid interactions
and plot data as 1d or as a meshgrid object
More sophisticated than standard ModelPlot
\begin{quote}\begin{description}
\item[{Parameters}] \leavevmode
\sphinxstyleliteralstrong{hdf5} (\sphinxstyleliteralemphasis{str}) \textendash{} hdf5 file name

\end{description}\end{quote}
\paragraph{Attributes}
\paragraph{Methods}
\index{get\_data() (lb\_colloids.Colloids.Colloid\_output.CCModelPlot method)}

\begin{fulllineitems}
\phantomsection\label{\detokenize{index:lb_colloids.Colloids.Colloid_output.CCModelPlot.get_data}}\pysiglinewithargsret{\sphinxbfcode{get\_data}}{\emph{key}}{}
Method to return data by key
\begin{quote}\begin{description}
\item[{Parameters}] \leavevmode
\sphinxstyleliteralstrong{key} (\sphinxstyleliteralemphasis{str}) \textendash{} valid model key

\end{description}\end{quote}

\end{fulllineitems}

\index{get\_data\_by\_path() (lb\_colloids.Colloids.Colloid\_output.CCModelPlot method)}

\begin{fulllineitems}
\phantomsection\label{\detokenize{index:lb_colloids.Colloids.Colloid_output.CCModelPlot.get_data_by_path}}\pysiglinewithargsret{\sphinxbfcode{get\_data\_by\_path}}{\emph{path}}{}
Method to return data by hdf5 path
\begin{quote}\begin{description}
\item[{Parameters}] \leavevmode
\sphinxstyleliteralstrong{path} (\sphinxstyleliteralemphasis{str}) \textendash{} valid HDF5 data path

\end{description}\end{quote}

\end{fulllineitems}

\index{keys (lb\_colloids.Colloids.Colloid\_output.CCModelPlot attribute)}

\begin{fulllineitems}
\phantomsection\label{\detokenize{index:lb_colloids.Colloids.Colloid_output.CCModelPlot.keys}}\pysigline{\sphinxbfcode{keys}}
Property method to return valid keys to obtain data

\end{fulllineitems}

\index{plot() (lb\_colloids.Colloids.Colloid\_output.CCModelPlot method)}

\begin{fulllineitems}
\phantomsection\label{\detokenize{index:lb_colloids.Colloids.Colloid_output.CCModelPlot.plot}}\pysiglinewithargsret{\sphinxbfcode{plot}}{\emph{key}, \emph{*args}, \emph{**kwargs}}{}
Plotting method for 1d colloid-colloid dlvo profiles
\begin{quote}\begin{description}
\item[{Parameters}] \leavevmode\begin{itemize}
\item {} 
\sphinxstyleliteralstrong{key} (\sphinxstyleliteralemphasis{str}) \textendash{} valid data key

\item {} 
\sphinxstyleliteralstrong{*args} \textendash{} 
matplotlib plotting args


\item {} 
\sphinxstyleliteralstrong{**kwargs} \textendash{} 
matplotlib plotting kwargs


\end{itemize}

\end{description}\end{quote}

\end{fulllineitems}

\index{plot\_mesh() (lb\_colloids.Colloids.Colloid\_output.CCModelPlot method)}

\begin{fulllineitems}
\phantomsection\label{\detokenize{index:lb_colloids.Colloids.Colloid_output.CCModelPlot.plot_mesh}}\pysiglinewithargsret{\sphinxbfcode{plot\_mesh}}{\emph{key}, \emph{*args}, \emph{**kwargs}}{}
Plotting method for 2d representation of colloid-colloid
dlvo profiles.
\begin{quote}\begin{description}
\item[{Parameters}] \leavevmode\begin{itemize}
\item {} 
\sphinxstyleliteralstrong{key} (\sphinxstyleliteralemphasis{str}) \textendash{} valid data key

\item {} 
\sphinxstyleliteralstrong{*args} \textendash{} 
matplotlib plotting args


\item {} 
\sphinxstyleliteralstrong{**kwargs} \textendash{} 
matplotlib plotting kwargs


\end{itemize}

\end{description}\end{quote}

\end{fulllineitems}


\end{fulllineitems}

\index{ColloidVelocity (class in lb\_colloids.Colloids.Colloid\_output)}

\begin{fulllineitems}
\phantomsection\label{\detokenize{index:lb_colloids.Colloids.Colloid_output.ColloidVelocity}}\pysiglinewithargsret{\sphinxbfcode{class }\sphinxcode{lb\_colloids.Colloids.Colloid\_output.}\sphinxbfcode{ColloidVelocity}}{\emph{filename}}{}
Method to return colloid velocity and statistics
relating to colloid velocity for a simulation. Class
needs to be rebuilt to work with timeseries and pathline
files for a more precise velocity measurement
\begin{quote}\begin{description}
\item[{Parameters}] \leavevmode
\sphinxstyleliteralstrong{filename} (\sphinxstyleliteralemphasis{str}) \textendash{} endpoint file name

\end{description}\end{quote}
\paragraph{Attributes}
\paragraph{Methods}
\index{cv (lb\_colloids.Colloids.Colloid\_output.ColloidVelocity attribute)}

\begin{fulllineitems}
\phantomsection\label{\detokenize{index:lb_colloids.Colloids.Colloid_output.ColloidVelocity.cv}}\pysigline{\sphinxbfcode{cv}}
\sphinxstyleemphasis{return} \textendash{} coeficient of variance of colloid velocities

\end{fulllineitems}

\index{max (lb\_colloids.Colloids.Colloid\_output.ColloidVelocity attribute)}

\begin{fulllineitems}
\phantomsection\label{\detokenize{index:lb_colloids.Colloids.Colloid_output.ColloidVelocity.max}}\pysigline{\sphinxbfcode{max}}
\sphinxstyleemphasis{return} \textendash{} maximum colloid velocity

\end{fulllineitems}

\index{mean (lb\_colloids.Colloids.Colloid\_output.ColloidVelocity attribute)}

\begin{fulllineitems}
\phantomsection\label{\detokenize{index:lb_colloids.Colloids.Colloid_output.ColloidVelocity.mean}}\pysigline{\sphinxbfcode{mean}}
\sphinxstyleemphasis{return} \textendash{} mean colloid velocity

\end{fulllineitems}

\index{min (lb\_colloids.Colloids.Colloid\_output.ColloidVelocity attribute)}

\begin{fulllineitems}
\phantomsection\label{\detokenize{index:lb_colloids.Colloids.Colloid_output.ColloidVelocity.min}}\pysigline{\sphinxbfcode{min}}
\sphinxstyleemphasis{return} \textendash{} minimum colloid velocity

\end{fulllineitems}

\index{plot() (lb\_colloids.Colloids.Colloid\_output.ColloidVelocity method)}

\begin{fulllineitems}
\phantomsection\label{\detokenize{index:lb_colloids.Colloids.Colloid_output.ColloidVelocity.plot}}\pysiglinewithargsret{\sphinxbfcode{plot}}{\emph{*args}, \emph{**kwargs}}{}
Method to plot distribution of velocities by
colloid for array of velocity.
\begin{quote}\begin{description}
\item[{Parameters}] \leavevmode
\sphinxstylestrong{:param *args: matplotlib plotting args}

\sphinxstylestrong{:param **kwargs: matplotlib plotting kwargs}

\end{description}\end{quote}

\end{fulllineitems}

\index{plot\_histogram() (lb\_colloids.Colloids.Colloid\_output.ColloidVelocity method)}

\begin{fulllineitems}
\phantomsection\label{\detokenize{index:lb_colloids.Colloids.Colloid_output.ColloidVelocity.plot_histogram}}\pysiglinewithargsret{\sphinxbfcode{plot\_histogram}}{\emph{nbin=10}, \emph{width=0.01}, \emph{*args}, \emph{**kwargs}}{}
User method to plot a histogram of velocities using
a bar chart.
\begin{quote}\begin{description}
\item[{Parameters}] \leavevmode\begin{itemize}
\item {} 
\sphinxstyleliteralstrong{nbin} (\sphinxstyleliteralemphasis{int}) \textendash{} number of specific bins for plotting

\item {} 
\sphinxstyleliteralstrong{width} (\sphinxstyleliteralemphasis{float}) \textendash{} matplotlib bar width.

\item {} 
\sphinxstyleliteralstrong{*args} \textendash{} 
matplotlib plotting args


\item {} 
\sphinxstyleliteralstrong{**kwargs} \textendash{} 
matplotlib plotting kwargs


\end{itemize}

\end{description}\end{quote}

\end{fulllineitems}

\index{stdev (lb\_colloids.Colloids.Colloid\_output.ColloidVelocity attribute)}

\begin{fulllineitems}
\phantomsection\label{\detokenize{index:lb_colloids.Colloids.Colloid_output.ColloidVelocity.stdev}}\pysigline{\sphinxbfcode{stdev}}
\sphinxstyleemphasis{return} \textendash{} standard deviation of colloid velocities

\end{fulllineitems}

\index{var (lb\_colloids.Colloids.Colloid\_output.ColloidVelocity attribute)}

\begin{fulllineitems}
\phantomsection\label{\detokenize{index:lb_colloids.Colloids.Colloid_output.ColloidVelocity.var}}\pysigline{\sphinxbfcode{var}}
\sphinxstyleemphasis{return} \textendash{} variance of colloid velocities

\end{fulllineitems}


\end{fulllineitems}

\index{DistributionFunction (class in lb\_colloids.Colloids.Colloid\_output)}

\begin{fulllineitems}
\phantomsection\label{\detokenize{index:lb_colloids.Colloids.Colloid_output.DistributionFunction}}\pysiglinewithargsret{\sphinxbfcode{class }\sphinxcode{lb\_colloids.Colloids.Colloid\_output.}\sphinxbfcode{DistributionFunction}}{\emph{filename}, \emph{nbin=1000}}{}
Class to plot a probablity distribution function of colloid breakthrough
from endpoint files.
\begin{quote}\begin{description}
\item[{Parameters}] \leavevmode\begin{itemize}
\item {} 
\sphinxstyleliteralstrong{filename} (\sphinxstyleliteralemphasis{str}) \textendash{} \textless{}\textgreater{}.endpoint file name

\item {} 
\sphinxstyleliteralstrong{nbin} (\sphinxstyleliteralemphasis{int}) \textendash{} number of bins for pdf calculation

\end{itemize}

\end{description}\end{quote}
\begin{quote}\begin{description}
\item[{Variables}] \leavevmode\begin{itemize}
\item {} 
\sphinxstyleliteralstrong{df} \textendash{} (pandas DataFrame): dataframe of endpoint data

\item {} 
{\hyperref[\detokenize{index:lb_colloids.LB.LB_2Dpermeability.LB2DModel.resolution}]{\sphinxcrossref{\sphinxstyleliteralstrong{resolution}}}} \textendash{} (float): model resolution

\item {} 
\sphinxstyleliteralstrong{timestep} \textendash{} (float): model timestep

\item {} 
\sphinxstyleliteralstrong{continuous} \textendash{} (int): interval of continuous release, 0 means pulse

\item {} 
\sphinxstyleliteralstrong{ncol} \textendash{} (float): number of colloids per release in simulation

\item {} 
\sphinxstyleliteralstrong{total\_ncol} \textendash{} (int): total number of colloids in simulation

\item {} 
\sphinxstyleliteralstrong{pdf} \textendash{} (np.recarray) colloid probability distribution function

\end{itemize}

\end{description}\end{quote}
\paragraph{Methods}
\index{plot() (lb\_colloids.Colloids.Colloid\_output.DistributionFunction method)}

\begin{fulllineitems}
\phantomsection\label{\detokenize{index:lb_colloids.Colloids.Colloid_output.DistributionFunction.plot}}\pysiglinewithargsret{\sphinxbfcode{plot}}{\emph{time=True}, \emph{*args}, \emph{**kwargs}}{}
Method to plot data into a matplotlib chart.
\begin{quote}\begin{description}
\item[{Parameters}] \leavevmode\begin{itemize}
\item {} 
\sphinxstyleliteralstrong{time} (\sphinxstyleliteralemphasis{bool}) \textendash{} if true x-axis is time, false is nts

\item {} 
\sphinxstyleliteralstrong{*args} \textendash{} 
matplotlib args for 1d charts


\item {} 
\sphinxstyleliteralstrong{**kwargs} \textendash{} 
matplotlib keyword arguments for 1d charts


\end{itemize}

\end{description}\end{quote}

\end{fulllineitems}

\index{plot\_pv() (lb\_colloids.Colloids.Colloid\_output.DistributionFunction method)}

\begin{fulllineitems}
\phantomsection\label{\detokenize{index:lb_colloids.Colloids.Colloid_output.DistributionFunction.plot_pv}}\pysiglinewithargsret{\sphinxbfcode{plot\_pv}}{\emph{*args}, \emph{**kwargs}}{}
Method to plot pdf data with pore volumes (non-dimensional time)
\begin{quote}\begin{description}
\item[{Parameters}] \leavevmode\begin{itemize}
\item {} 
\sphinxstyleliteralstrong{*args} \textendash{} 
matplotlib args for 1d plotting


\item {} 
\sphinxstyleliteralstrong{**kwargs} \textendash{} 
matplotlib kwargs for 1d plotting


\end{itemize}

\end{description}\end{quote}

\end{fulllineitems}

\index{pore\_volume\_conversion() (lb\_colloids.Colloids.Colloid\_output.DistributionFunction method)}

\begin{fulllineitems}
\phantomsection\label{\detokenize{index:lb_colloids.Colloids.Colloid_output.DistributionFunction.pore_volume_conversion}}\pysiglinewithargsret{\sphinxbfcode{pore\_volume\_conversion}}{}{}
Method to retrieve the pore volume calculation
conversion for plotting colloids.

\end{fulllineitems}

\index{reset\_pdf() (lb\_colloids.Colloids.Colloid\_output.DistributionFunction method)}

\begin{fulllineitems}
\phantomsection\label{\detokenize{index:lb_colloids.Colloids.Colloid_output.DistributionFunction.reset_pdf}}\pysiglinewithargsret{\sphinxbfcode{reset\_pdf}}{\emph{nbin}, \emph{normalize=False}}{}
Method to generate a probability distribution function
based upon user supplied bin size.
\begin{quote}\begin{description}
\item[{Parameters}] \leavevmode\begin{itemize}
\item {} 
\sphinxstyleliteralstrong{nbin} (\sphinxstyleliteralemphasis{int}) \textendash{} number of time steps to base bin on

\item {} 
\sphinxstyleliteralstrong{normalize} (\sphinxstyleliteralemphasis{bool}) \textendash{} method to calculate pdf by residence time or end time

\end{itemize}

\end{description}\end{quote}

\end{fulllineitems}


\end{fulllineitems}

\index{Hdf5Reader (class in lb\_colloids.Colloids.Colloid\_output)}

\begin{fulllineitems}
\phantomsection\label{\detokenize{index:lb_colloids.Colloids.Colloid_output.Hdf5Reader}}\pysiglinewithargsret{\sphinxbfcode{class }\sphinxcode{lb\_colloids.Colloids.Colloid\_output.}\sphinxbfcode{Hdf5Reader}}{\emph{hdf5}}{}
Reader object to read in HDF5 stored outputs
from colloid models. Contains a data\_paths dictionary
which allows the user to use keys to access data
\begin{quote}\begin{description}
\item[{Parameters}] \leavevmode
\sphinxstyleliteralstrong{hdf5} (\sphinxstyleliteralemphasis{str}) \textendash{} LB-Colloid hdf5 file name

\end{description}\end{quote}
\paragraph{Attributes}
\paragraph{Methods}
\index{get\_data() (lb\_colloids.Colloids.Colloid\_output.Hdf5Reader method)}

\begin{fulllineitems}
\phantomsection\label{\detokenize{index:lb_colloids.Colloids.Colloid_output.Hdf5Reader.get_data}}\pysiglinewithargsret{\sphinxbfcode{get\_data}}{\emph{key}}{}
Method to retrieve hdf5 data by dict. key
\begin{quote}\begin{description}
\item[{Parameters}] \leavevmode
\sphinxstyleliteralstrong{key} (\sphinxstyleliteralemphasis{str}) \textendash{} valid dictionary key from self.keys

\end{description}\end{quote}
\begin{quote}\begin{description}
\item[{Returns}] \leavevmode
data \textless{}varies\textgreater{}

\end{description}\end{quote}

\end{fulllineitems}

\index{get\_data\_by\_path() (lb\_colloids.Colloids.Colloid\_output.Hdf5Reader method)}

\begin{fulllineitems}
\phantomsection\label{\detokenize{index:lb_colloids.Colloids.Colloid_output.Hdf5Reader.get_data_by_path}}\pysiglinewithargsret{\sphinxbfcode{get\_data\_by\_path}}{\emph{path}}{}
Method to retrieve hdf5 data by specific hdf5 path
\begin{quote}\begin{description}
\item[{Parameters}] \leavevmode
\sphinxstyleliteralstrong{path} (\sphinxstyleliteralemphasis{str}) \textendash{} hdf5 directory path to data

\end{description}\end{quote}
\begin{quote}\begin{description}
\item[{Returns}] \leavevmode
data \textless{}varies\textgreater{}

\end{description}\end{quote}

\end{fulllineitems}

\index{keys (lb\_colloids.Colloids.Colloid\_output.Hdf5Reader attribute)}

\begin{fulllineitems}
\phantomsection\label{\detokenize{index:lb_colloids.Colloids.Colloid_output.Hdf5Reader.keys}}\pysigline{\sphinxbfcode{keys}}
\sphinxstyleemphasis{return} \textendash{} list of valid hdf5 data keys

\end{fulllineitems}


\end{fulllineitems}

\index{LBOutput (class in lb\_colloids.Colloids.Colloid\_output)}

\begin{fulllineitems}
\phantomsection\label{\detokenize{index:lb_colloids.Colloids.Colloid_output.LBOutput}}\pysiglinewithargsret{\sphinxbfcode{class }\sphinxcode{lb\_colloids.Colloids.Colloid\_output.}\sphinxbfcode{LBOutput}}{\emph{hdf5}}{}
Class to anaylze LB fluid/solid properties
\begin{quote}\begin{description}
\item[{Parameters}] \leavevmode
\sphinxstyleliteralstrong{hdf} (\sphinxstyleliteralemphasis{str}) \textendash{} hdf5 output filename

\end{description}\end{quote}
\paragraph{Attributes}
\paragraph{Methods}
\index{get\_data() (lb\_colloids.Colloids.Colloid\_output.LBOutput method)}

\begin{fulllineitems}
\phantomsection\label{\detokenize{index:lb_colloids.Colloids.Colloid_output.LBOutput.get_data}}\pysiglinewithargsret{\sphinxbfcode{get\_data}}{\emph{key}}{}
Method to select data from hdf5 file based on key, instead
of data path
\begin{quote}\begin{description}
\item[{Parameters}] \leavevmode
\sphinxstyleliteralstrong{key} (\sphinxstyleliteralemphasis{str}) \textendash{} lattice boltzmann data key

\end{description}\end{quote}
\begin{quote}\begin{description}
\item[{Returns}] \leavevmode
data

\end{description}\end{quote}

\end{fulllineitems}

\index{keys (lb\_colloids.Colloids.Colloid\_output.LBOutput attribute)}

\begin{fulllineitems}
\phantomsection\label{\detokenize{index:lb_colloids.Colloids.Colloid_output.LBOutput.keys}}\pysigline{\sphinxbfcode{keys}}
\sphinxstyleemphasis{return} \textendash{} Lattice boltzmann data keys

\end{fulllineitems}


\end{fulllineitems}

\index{ModelPlot (class in lb\_colloids.Colloids.Colloid\_output)}

\begin{fulllineitems}
\phantomsection\label{\detokenize{index:lb_colloids.Colloids.Colloid_output.ModelPlot}}\pysiglinewithargsret{\sphinxbfcode{class }\sphinxcode{lb\_colloids.Colloids.Colloid\_output.}\sphinxbfcode{ModelPlot}}{\emph{hdf5}}{}
Class to retrieve Colloid force arrays
and plot for data analysis.
\begin{quote}\begin{description}
\item[{Parameters}] \leavevmode
\sphinxstyleliteralstrong{hdf5} (\sphinxstyleliteralemphasis{str}) \textendash{} hdf5 file name

\end{description}\end{quote}
\paragraph{Attributes}
\paragraph{Methods}
\index{get\_data() (lb\_colloids.Colloids.Colloid\_output.ModelPlot method)}

\begin{fulllineitems}
\phantomsection\label{\detokenize{index:lb_colloids.Colloids.Colloid_output.ModelPlot.get_data}}\pysiglinewithargsret{\sphinxbfcode{get\_data}}{\emph{key}}{}
Get data method to view and analyze colloid
force arrays
\begin{quote}\begin{description}
\item[{Parameters}] \leavevmode
\sphinxstyleliteralstrong{key} (\sphinxstyleliteralemphasis{str}) \textendash{} valid dictionary key from self.keys

\end{description}\end{quote}
\begin{quote}\begin{description}
\item[{Returns}] \leavevmode
data \textless{}varies\textgreater{}

\end{description}\end{quote}

\end{fulllineitems}

\index{get\_data\_by\_path() (lb\_colloids.Colloids.Colloid\_output.ModelPlot method)}

\begin{fulllineitems}
\phantomsection\label{\detokenize{index:lb_colloids.Colloids.Colloid_output.ModelPlot.get_data_by_path}}\pysiglinewithargsret{\sphinxbfcode{get\_data\_by\_path}}{\emph{path}}{}
Method to retrieve hdf5 data by specific path
\begin{quote}\begin{description}
\item[{Parameters}] \leavevmode
\sphinxstyleliteralstrong{path} (\sphinxstyleliteralemphasis{str}) \textendash{} hdf5 directory path to data

\end{description}\end{quote}
\begin{quote}\begin{description}
\item[{Returns}] \leavevmode
data \textless{}varies\textgreater{}

\end{description}\end{quote}

\end{fulllineitems}

\index{plot() (lb\_colloids.Colloids.Colloid\_output.ModelPlot method)}

\begin{fulllineitems}
\phantomsection\label{\detokenize{index:lb_colloids.Colloids.Colloid_output.ModelPlot.plot}}\pysiglinewithargsret{\sphinxbfcode{plot}}{\emph{key}, \emph{*args}, \emph{**kwargs}}{}
Hdf array plotting using Hdf5Reader keys
\begin{quote}\begin{description}
\item[{Parameters}] \leavevmode\begin{itemize}
\item {} 
\sphinxstyleliteralstrong{key} (\sphinxstyleliteralemphasis{str}) \textendash{} valid dictionary key from self.keys

\item {} 
\sphinxstyleliteralstrong{*args} \textendash{} 
matplotlib plotting args


\item {} 
\sphinxstyleliteralstrong{**kwargs} \textendash{} 
matplotlib plotting kwargs


\end{itemize}

\end{description}\end{quote}

\end{fulllineitems}

\index{plot\_velocity\_magnitude() (lb\_colloids.Colloids.Colloid\_output.ModelPlot method)}

\begin{fulllineitems}
\phantomsection\label{\detokenize{index:lb_colloids.Colloids.Colloid_output.ModelPlot.plot_velocity_magnitude}}\pysiglinewithargsret{\sphinxbfcode{plot\_velocity\_magnitude}}{\emph{nbin=10}, \emph{*args}, \emph{**kwargs}}{}
Method to create a quiver plot to display the
magnitude and direction of velocity vectors within
the system.
\begin{quote}\begin{description}
\item[{Parameters}] \leavevmode\begin{itemize}
\item {} 
\sphinxstyleliteralstrong{nbin} (\sphinxstyleliteralemphasis{int}) \textendash{} refinement for quiver plotting

\item {} 
\sphinxstyleliteralstrong{*args} \textendash{} 
matplotlib plotting args


\item {} 
\sphinxstyleliteralstrong{**kwargs} \textendash{} 
matplotlib plotting kwargs


\end{itemize}

\end{description}\end{quote}

\end{fulllineitems}


\end{fulllineitems}

\phantomsection\label{\detokenize{index:module-lb_colloids.nam_file}}\index{lb\_colloids.nam\_file (module)}\index{NamFile (class in lb\_colloids.nam\_file)}

\begin{fulllineitems}
\phantomsection\label{\detokenize{index:lb_colloids.nam_file.NamFile}}\pysiglinewithargsret{\sphinxbfcode{class }\sphinxcode{lb\_colloids.nam\_file.}\sphinxbfcode{NamFile}}{\emph{nam\_file}}{}
Class to read the nam file instance of
a lattice boltzmann model and delegate the IO
reading to the applicable lb\_colloid config
reading classes
\begin{quote}\begin{description}
\item[{Parameters}] \leavevmode
\sphinxstyleliteralstrong{nam\_file} (\sphinxstyleliteralemphasis{str}) \textendash{} nam file name

\end{description}\end{quote}

\end{fulllineitems}



\section{LB Colloid Setup (background classes)}
\label{\detokenize{index:lb-colloid-setup-background-classes}}
The colloid setup module contains basic utilities to create and enforce boundary conditions. The computation of grid distance to the nearest solid interface is performed within this module. The normal user should not call these objects, as they are called internally by the ColloidModel instance.


\subsection{API documentation}
\label{\detokenize{index:id54}}\phantomsection\label{\detokenize{index:module-lb_colloids.Colloids.Colloid_Setup}}\index{lb\_colloids.Colloids.Colloid\_Setup (module)}
Colloid\_Setup contains background classes and methods to prepare a
model domain for colloid simulation. The user should not need to
import or call methods directly from this module.
\index{GridArray (class in lb\_colloids.Colloids.Colloid\_Setup)}

\begin{fulllineitems}
\phantomsection\label{\detokenize{index:lb_colloids.Colloids.Colloid_Setup.GridArray}}\pysiglinewithargsret{\sphinxbfcode{class }\sphinxcode{lb\_colloids.Colloids.Colloid\_Setup.}\sphinxbfcode{GridArray}}{\emph{arr}, \emph{gridres}, \emph{gridsplit}, \emph{solid=True}}{}
Gridarray class creates arrays of distances from pore spaces, corrects for
interpolation effects at pore boundaries, and creates vector arrays that are
later used to give direction to forces.
\begin{quote}\begin{description}
\item[{Parameters}] \leavevmode\begin{itemize}
\item {} 
\sphinxstyleliteralstrong{bool}\sphinxstyleliteralstrong{) }\sphinxstyleliteralstrong{arr} (\sphinxstyleliteralemphasis{(}\sphinxstyleliteralemphasis{np.array}\sphinxstyleliteralemphasis{,}) \textendash{} Array of boolean porous media (segmented image array)

\item {} 
\sphinxstyleliteralstrong{gridres} (\sphinxstyleliteralemphasis{float}) \textendash{} model resolution in meters

\item {} 
\sphinxstyleliteralstrong{gridsplit} (\sphinxstyleliteralemphasis{int}) \textendash{} interpolation factor for refining grid mesh

\item {} 
\sphinxstyleliteralstrong{solid} (\sphinxstyleliteralemphasis{bool}) \textendash{} solid phase boolean identifier, default=True

\end{itemize}

\end{description}\end{quote}
\paragraph{Attributes}
\index{gridx (lb\_colloids.Colloids.Colloid\_Setup.GridArray attribute)}

\begin{fulllineitems}
\phantomsection\label{\detokenize{index:lb_colloids.Colloids.Colloid_Setup.GridArray.gridx}}\pysigline{\sphinxbfcode{gridx}}
\sphinxstyleemphasis{return} \textendash{} (np.array, np.float) Array of distances from nearest solid phase in the x-direction

\end{fulllineitems}

\index{gridy (lb\_colloids.Colloids.Colloid\_Setup.GridArray attribute)}

\begin{fulllineitems}
\phantomsection\label{\detokenize{index:lb_colloids.Colloids.Colloid_Setup.GridArray.gridy}}\pysigline{\sphinxbfcode{gridy}}
\sphinxstyleemphasis{return} \textendash{} (np.array, np.float) Array of distances from nearest solid phase in the y-direction

\end{fulllineitems}

\index{vector\_x (lb\_colloids.Colloids.Colloid\_Setup.GridArray attribute)}

\begin{fulllineitems}
\phantomsection\label{\detokenize{index:lb_colloids.Colloids.Colloid_Setup.GridArray.vector_x}}\pysigline{\sphinxbfcode{vector\_x}}
\sphinxstyleemphasis{return} \textendash{} (np.array, np.float) Array of specific vector directions in the x-direction (-1 == left, 1 == right)

\end{fulllineitems}

\index{vector\_y (lb\_colloids.Colloids.Colloid\_Setup.GridArray attribute)}

\begin{fulllineitems}
\phantomsection\label{\detokenize{index:lb_colloids.Colloids.Colloid_Setup.GridArray.vector_y}}\pysigline{\sphinxbfcode{vector\_y}}
\sphinxstyleemphasis{return} \textendash{} (np.array, np.float) Array of specific vector directions in the y-direction (-1 == down, 1 == up)

\end{fulllineitems}


\end{fulllineitems}

\index{Hdf5Reader (class in lb\_colloids.Colloids.Colloid\_Setup)}

\begin{fulllineitems}
\phantomsection\label{\detokenize{index:lb_colloids.Colloids.Colloid_Setup.Hdf5Reader}}\pysiglinewithargsret{\sphinxbfcode{class }\sphinxcode{lb\_colloids.Colloids.Colloid\_Setup.}\sphinxbfcode{Hdf5Reader}}{\emph{hdf\_name}}{}
Hdf5 reader class to grab results from lattice Boltzmann model runs
to parameterize the colloid simulation. Consider moving this class to
the Colloid\_IO module
\begin{quote}\begin{description}
\item[{Parameters}] \leavevmode
\sphinxstyleliteralstrong{HDF\_name} (\sphinxstyleliteralemphasis{str}) \textendash{} lattice boltzmann hdf file name.

\item[{Variables}] \leavevmode\begin{itemize}
\item {} 
\sphinxstyleliteralstrong{imarray} (\sphinxstyleliteralemphasis{np.ndarray}) \textendash{} binary image array defining model boundaries

\item {} 
\sphinxstyleliteralstrong{uarry} (\sphinxstyleliteralemphasis{np.ndarray}) \textendash{} velocity array of {[}y, x{]}

\item {} 
\sphinxstyleliteralstrong{yu} (\sphinxstyleliteralemphasis{np.ndarray}) \textendash{} y velocity array from lattice Boltzmann simulation

\item {} 
\sphinxstyleliteralstrong{xu} (\sphinxstyleliteralemphasis{np.ndarray}) \textendash{} x velocity array from lattice Boltzmann simulation

\item {} 
\sphinxstyleliteralstrong{mean\_yu} (\sphinxstyleliteralemphasis{float}) \textendash{} mean velocity in the y direction

\item {} 
\sphinxstyleliteralstrong{mean\_xu} (\sphinxstyleliteralemphasis{float}) \textendash{} mean velocity in the x direction

\item {} 
\sphinxstyleliteralstrong{velocity\_factor} (\sphinxstyleliteralemphasis{float}) \textendash{} velocity dimensionalization factor

\end{itemize}

\end{description}\end{quote}

\end{fulllineitems}

\index{InterpV() (in module lb\_colloids.Colloids.Colloid\_Setup)}

\begin{fulllineitems}
\phantomsection\label{\detokenize{index:lb_colloids.Colloids.Colloid_Setup.InterpV}}\pysiglinewithargsret{\sphinxcode{lb\_colloids.Colloids.Colloid\_Setup.}\sphinxbfcode{InterpV}}{\emph{LBv}, \emph{gridsplit}, \emph{img=False}}{}
Interpolation method for the lattice Boltzmann velocity array
\begin{quote}\begin{description}
\item[{Parameters}] \leavevmode\begin{itemize}
\item {} 
\sphinxstyleliteralstrong{LBv} (\sphinxstyleliteralemphasis{np.ndarray}) \textendash{} lattice boltzmann velocity array with pore boundaries enforced

\item {} 
\sphinxstyleliteralstrong{gridsplit} (\sphinxstyleliteralemphasis{float}) \textendash{} interpolation factor

\item {} 
\sphinxstyleliteralstrong{img} (\sphinxstyleliteralemphasis{bool}) \textendash{} flag to indicate boolean image interpolation or velocity interpolation

\end{itemize}

\end{description}\end{quote}
\begin{quote}\begin{description}
\item[{Returns}] \leavevmode
(np.ndarray) interpolated velocity array

\end{description}\end{quote}

\end{fulllineitems}

\index{LBVArray() (in module lb\_colloids.Colloids.Colloid\_Setup)}

\begin{fulllineitems}
\phantomsection\label{\detokenize{index:lb_colloids.Colloids.Colloid_Setup.LBVArray}}\pysiglinewithargsret{\sphinxcode{lb\_colloids.Colloids.Colloid\_Setup.}\sphinxbfcode{LBVArray}}{\emph{LBv}, \emph{img}}{}
Method to create a velocity array for use in the colloid simulation model
\begin{quote}\begin{description}
\item[{Parameters}] \leavevmode\begin{itemize}
\item {} 
\sphinxstyleliteralstrong{LBv} (\sphinxstyleliteralemphasis{np.ndarray}) \textendash{} lattice boltzmann velocity array

\item {} 
\sphinxstyleliteralstrong{img} (\sphinxstyleliteralemphasis{np.ndarray}) \textendash{} boolean image array

\end{itemize}

\end{description}\end{quote}
\begin{quote}\begin{description}
\item[{Returns}] \leavevmode
np.ndarray of velocity

\end{description}\end{quote}

\end{fulllineitems}



\chapter{Penetrable Sphere (PSHPERE)}
\label{\detokenize{index:penetrable-sphere-pshpere}}\phantomsection\label{\detokenize{index:module-lb_colloids}}\index{lb\_colloids (module)}\index{PSphere (class in lb\_colloids)}

\begin{fulllineitems}
\phantomsection\label{\detokenize{index:lb_colloids.PSphere}}\pysiglinewithargsret{\sphinxbfcode{class }\sphinxcode{lb\_colloids.}\sphinxbfcode{PSphere}}{\emph{radius=20}, \emph{porosity=0.5}, \emph{dimension=256}, \emph{sensitivity=0.08}}{}
Pshpere is a class that allows for the automated generation of
synthetic porous media in two-dimensions. This approach can be
expanded to three dimensions with some effort.
\begin{quote}\begin{description}
\item[{Parameters}] \leavevmode\begin{itemize}
\item {} 
\sphinxstyleliteralstrong{radius} (\sphinxstyleliteralemphasis{int}) \textendash{} grain size radius

\item {} 
\sphinxstyleliteralstrong{porosity} (\sphinxstyleliteralemphasis{float}) \textendash{} target porosity for porous media

\item {} 
\sphinxstyleliteralstrong{dimension} (\sphinxstyleliteralemphasis{int}) \textendash{} the x and y dimension in pixels for the domain

\item {} 
\sphinxstyleliteralstrong{sensitivity} (\sphinxstyleliteralemphasis{float}) \textendash{} a porosity sensitivity target. This is the allowable range of error for PShpere

\end{itemize}

\end{description}\end{quote}
\paragraph{Methods}

\end{fulllineitems}

\phantomsection\label{\detokenize{index:module-lb_colloids.utilities.psphere}}\index{lb\_colloids.utilities.psphere (module)}
The Pshpere module contains a class named PShpere which allows the user to
generate synthetic porous media, and to get information about that porous
media

A user can instantiate and use the PSphere() object as follows:

\begin{sphinxVerbatim}[commandchars=\\\{\}]
\PYG{g+gp}{\PYGZgt{}\PYGZgt{}\PYGZgt{} }\PYG{k+kn}{from} \PYG{n+nn}{lb\PYGZus{}colloids} \PYG{k}{import} \PYG{n}{PSphere}
\PYG{g+gp}{\PYGZgt{}\PYGZgt{}\PYGZgt{} }\PYG{n}{img} \PYG{o}{=} \PYG{n}{Psphere}\PYG{p}{(}\PYG{n}{dimension}\PYG{o}{=}\PYG{l+m+mi}{200}\PYG{p}{,} \PYG{n}{radius}\PYG{o}{=}\PYG{l+m+mi}{20}\PYG{p}{,} \PYG{n}{porosity}\PYG{o}{=}\PYG{l+m+mf}{0.375}\PYG{p}{,} \PYG{n}{sensitivity}\PYG{o}{=}\PYG{l+m+mf}{0.01}\PYG{p}{)}
\PYG{g+gp}{\PYGZgt{}\PYGZgt{}\PYGZgt{} }\PYG{c+c1}{\PYGZsh{} hydraulic radius can be calculated}
\PYG{g+gp}{\PYGZgt{}\PYGZgt{}\PYGZgt{} }\PYG{n}{rh} \PYG{o}{=} \PYG{n}{img}\PYG{o}{.}\PYG{n}{calculate\PYGZus{}hydraulic\PYGZus{}radius}\PYG{p}{(}\PYG{n}{resolution}\PYG{o}{=}\PYG{l+m+mf}{1e\PYGZhy{}06}\PYG{p}{)}
\PYG{g+gp}{\PYGZgt{}\PYGZgt{}\PYGZgt{} }\PYG{c+c1}{\PYGZsh{} to get a copy of the porous media use}
\PYG{g+gp}{\PYGZgt{}\PYGZgt{}\PYGZgt{} }\PYG{n}{matrix} \PYG{o}{=} \PYG{n}{img}\PYG{o}{.}\PYG{n}{matrix}
\PYG{g+gp}{\PYGZgt{}\PYGZgt{}\PYGZgt{} }\PYG{c+c1}{\PYGZsh{} save the image}
\PYG{g+gp}{\PYGZgt{}\PYGZgt{}\PYGZgt{} }\PYG{n}{img}\PYG{o}{.}\PYG{n}{save}\PYG{p}{(}\PYG{l+s+s2}{\PYGZdq{}}\PYG{l+s+s2}{test\PYGZus{}image.png}\PYG{l+s+s2}{\PYGZdq{}}\PYG{p}{)}
\end{sphinxVerbatim}
\index{PSphere (class in lb\_colloids.utilities.psphere)}

\begin{fulllineitems}
\phantomsection\label{\detokenize{index:lb_colloids.utilities.psphere.PSphere}}\pysiglinewithargsret{\sphinxbfcode{class }\sphinxcode{lb\_colloids.utilities.psphere.}\sphinxbfcode{PSphere}}{\emph{radius=20}, \emph{porosity=0.5}, \emph{dimension=256}, \emph{sensitivity=0.08}}{}
Pshpere is a class that allows for the automated generation of
synthetic porous media in two-dimensions. This approach can be
expanded to three dimensions with some effort.
\begin{quote}\begin{description}
\item[{Parameters}] \leavevmode\begin{itemize}
\item {} 
\sphinxstyleliteralstrong{radius} (\sphinxstyleliteralemphasis{int}) \textendash{} grain size radius

\item {} 
\sphinxstyleliteralstrong{porosity} (\sphinxstyleliteralemphasis{float}) \textendash{} target porosity for porous media

\item {} 
\sphinxstyleliteralstrong{dimension} (\sphinxstyleliteralemphasis{int}) \textendash{} the x and y dimension in pixels for the domain

\item {} 
\sphinxstyleliteralstrong{sensitivity} (\sphinxstyleliteralemphasis{float}) \textendash{} a porosity sensitivity target. This is the allowable range of error for PShpere

\end{itemize}

\end{description}\end{quote}
\paragraph{Methods}
\index{calculate\_hydraulic\_radius() (lb\_colloids.utilities.psphere.PSphere method)}

\begin{fulllineitems}
\phantomsection\label{\detokenize{index:lb_colloids.utilities.psphere.PSphere.calculate_hydraulic_radius}}\pysiglinewithargsret{\sphinxbfcode{calculate\_hydraulic\_radius}}{\emph{resolution}}{}
Calculates the hydraulic radius of the porous medium
\begin{quote}\begin{description}
\item[{Parameters}] \leavevmode
\sphinxstyleliteralstrong{resolution} (\sphinxstyleliteralemphasis{float}) \textendash{} model resolution applied to image

\item[{Returns}] \leavevmode
hydraulic radius of the image

\end{description}\end{quote}

\end{fulllineitems}

\index{check\_percolation() (lb\_colloids.utilities.psphere.PSphere method)}

\begin{fulllineitems}
\phantomsection\label{\detokenize{index:lb_colloids.utilities.psphere.PSphere.check_percolation}}\pysiglinewithargsret{\sphinxbfcode{check\_percolation}}{}{}
Modified sweep line technique that checks the porous media percolation
Pshpere automatically calls this during porous media creation.

\end{fulllineitems}

\index{generate\_plane() (lb\_colloids.utilities.psphere.PSphere method)}

\begin{fulllineitems}
\phantomsection\label{\detokenize{index:lb_colloids.utilities.psphere.PSphere.generate_plane}}\pysiglinewithargsret{\sphinxbfcode{generate\_plane}}{}{}
Main method used to generate a porous media plane by PSphere,
this should not be called by the user

\end{fulllineitems}

\index{iround() (lb\_colloids.utilities.psphere.PSphere method)}

\begin{fulllineitems}
\phantomsection\label{\detokenize{index:lb_colloids.utilities.psphere.PSphere.iround}}\pysiglinewithargsret{\sphinxbfcode{iround}}{\emph{val}}{}
Rounding routine to set index locations.
\begin{quote}\begin{description}
\item[{Parameters}] \leavevmode
\sphinxstyleliteralstrong{val} (\sphinxstyleliteralemphasis{float}) \textendash{} floating point value

\item[{Return int}] \leavevmode
\end{description}\end{quote}

\end{fulllineitems}

\index{patch() (lb\_colloids.utilities.psphere.PSphere method)}

\begin{fulllineitems}
\phantomsection\label{\detokenize{index:lb_colloids.utilities.psphere.PSphere.patch}}\pysiglinewithargsret{\sphinxbfcode{patch}}{\emph{x}, \emph{y}, \emph{radius}}{}
The patch method is used to set grains into a porous media.
Not to be called by the user!
\begin{quote}\begin{description}
\item[{Parameters}] \leavevmode\begin{itemize}
\item {} 
\sphinxstyleliteralstrong{x} (\sphinxstyleliteralemphasis{int}) \textendash{} x index location

\item {} 
\sphinxstyleliteralstrong{y} (\sphinxstyleliteralemphasis{int}) \textendash{} y index location

\item {} 
\sphinxstyleliteralstrong{radius} (\sphinxstyleliteralemphasis{int}) \textendash{} grain radius

\end{itemize}

\end{description}\end{quote}

\end{fulllineitems}

\index{save\_image() (lb\_colloids.utilities.psphere.PSphere method)}

\begin{fulllineitems}
\phantomsection\label{\detokenize{index:lb_colloids.utilities.psphere.PSphere.save_image}}\pysiglinewithargsret{\sphinxbfcode{save\_image}}{\emph{image\_name}}{}
Save method, to save an image to file!
\begin{quote}\begin{description}
\item[{Parameters}] \leavevmode
\sphinxstyleliteralstrong{image\_name} (\sphinxstyleliteralemphasis{str}) \textendash{} image path and name

\end{description}\end{quote}

\end{fulllineitems}

\index{static\_hydraulic\_radius() (lb\_colloids.utilities.psphere.PSphere static method)}

\begin{fulllineitems}
\phantomsection\label{\detokenize{index:lb_colloids.utilities.psphere.PSphere.static_hydraulic_radius}}\pysiglinewithargsret{\sphinxbfcode{static }\sphinxbfcode{static\_hydraulic\_radius}}{\emph{matrix}, \emph{invert=True}}{}
Static method to calculate the hydraulic radius of a given porous medium
\begin{quote}\begin{description}
\item[{Parameters}] \leavevmode\begin{itemize}
\item {} 
\sphinxstyleliteralstrong{matrix} (\sphinxstyleliteralemphasis{np.ndarray}) \textendash{} boolean array corresponding to porous media

\item {} 
\sphinxstyleliteralstrong{invert} (\sphinxstyleliteralemphasis{bool}) \textendash{} inverts model, pore space needs to be set to True

\end{itemize}

\item[{Returns}] \leavevmode
hydraulic radius of the image

\end{description}\end{quote}

\end{fulllineitems}

\index{static\_porosity() (lb\_colloids.utilities.psphere.PSphere static method)}

\begin{fulllineitems}
\phantomsection\label{\detokenize{index:lb_colloids.utilities.psphere.PSphere.static_porosity}}\pysiglinewithargsret{\sphinxbfcode{static }\sphinxbfcode{static\_porosity}}{\emph{matrix}, \emph{invert=True}}{}
Static method to calculate the porosity of a given porous media
\begin{quote}\begin{description}
\item[{Parameters}] \leavevmode\begin{itemize}
\item {} 
\sphinxstyleliteralstrong{matrix} (\sphinxstyleliteralemphasis{np.ndarray}) \textendash{} boolean array corresponding to porous media

\item {} 
\sphinxstyleliteralstrong{invert} (\sphinxstyleliteralemphasis{bool}) \textendash{} inverts model, pore space needs to be set to True

\end{itemize}

\item[{Returns}] \leavevmode
porosity of the image

\end{description}\end{quote}

\end{fulllineitems}

\index{static\_surface\_area() (lb\_colloids.utilities.psphere.PSphere static method)}

\begin{fulllineitems}
\phantomsection\label{\detokenize{index:lb_colloids.utilities.psphere.PSphere.static_surface_area}}\pysiglinewithargsret{\sphinxbfcode{static }\sphinxbfcode{static\_surface\_area}}{\emph{matrix}, \emph{invert=True}}{}
Static method to calculate the non-dimensional surface
area of a given porous medium
\begin{quote}\begin{description}
\item[{Parameters}] \leavevmode\begin{itemize}
\item {} 
\sphinxstyleliteralstrong{matrix} (\sphinxstyleliteralemphasis{np.ndarray}) \textendash{} boolean array corresponding to porous media

\item {} 
\sphinxstyleliteralstrong{invert} (\sphinxstyleliteralemphasis{bool}) \textendash{} inverts model, pore space needs to be set to True

\end{itemize}

\item[{Returns}] \leavevmode
hydraulic radius of the image

\end{description}\end{quote}

\end{fulllineitems}


\end{fulllineitems}



\renewcommand{\indexname}{Python Module Index}
\begin{sphinxtheindex}
\def\bigletter#1{{\Large\sffamily#1}\nopagebreak\vspace{1mm}}
\bigletter{l}
\item {\sphinxstyleindexentry{lb\_colloids}}\sphinxstyleindexpageref{index:\detokenize{module-lb_colloids}}
\item {\sphinxstyleindexentry{lb\_colloids.Colloids.Colloid\_IO}}\sphinxstyleindexpageref{index:\detokenize{module-lb_colloids.Colloids.Colloid_IO}}
\item {\sphinxstyleindexentry{lb\_colloids.Colloids.Colloid\_Math}}\sphinxstyleindexpageref{index:\detokenize{module-lb_colloids.Colloids.Colloid_Math}}
\item {\sphinxstyleindexentry{lb\_colloids.Colloids.Colloid\_output}}\sphinxstyleindexpageref{index:\detokenize{module-lb_colloids.Colloids.Colloid_output}}
\item {\sphinxstyleindexentry{lb\_colloids.Colloids.Colloid\_Setup}}\sphinxstyleindexpageref{index:\detokenize{module-lb_colloids.Colloids.Colloid_Setup}}
\item {\sphinxstyleindexentry{lb\_colloids.Colloids.LB\_Colloid}}\sphinxstyleindexpageref{index:\detokenize{module-lb_colloids.Colloids.LB_Colloid}}
\item {\sphinxstyleindexentry{lb\_colloids.LB.LB\_2Dimage}}\sphinxstyleindexpageref{index:\detokenize{module-lb_colloids.LB.LB_2Dimage}}
\item {\sphinxstyleindexentry{lb\_colloids.LB.LB\_2Dpermeability}}\sphinxstyleindexpageref{index:\detokenize{module-lb_colloids.LB.LB_2Dpermeability}}
\item {\sphinxstyleindexentry{lb\_colloids.LB.LB\_pretty}}\sphinxstyleindexpageref{index:\detokenize{module-lb_colloids.LB.LB_pretty}}
\item {\sphinxstyleindexentry{lb\_colloids.LB.LBIO}}\sphinxstyleindexpageref{index:\detokenize{module-lb_colloids.LB.LBIO}}
\item {\sphinxstyleindexentry{lb\_colloids.nam\_file}}\sphinxstyleindexpageref{index:\detokenize{module-lb_colloids.nam_file}}
\item {\sphinxstyleindexentry{lb\_colloids.utilities.psphere}}\sphinxstyleindexpageref{index:\detokenize{module-lb_colloids.utilities.psphere}}
\end{sphinxtheindex}

\renewcommand{\indexname}{Index}
\IfFileExists{\jobname.ind}
             {\footnotesize\raggedright\printindex}
             {\begin{sphinxtheindex}\end{sphinxtheindex}}
\end{document}